\subsection{List of Overloaded MATLAB Operators and Functions}
\label{sec:overloaded}

The following table lists the set of overloaded MATLAB operators that can be used to implicitly create factors.  The table shows the operator, the corresponding built-in factor (as described in section~\ref{sec:builtInFactors}), the valid variable data types of the inputs and outputs (B = Bit, D = Discrete, or R = Real), and wether or not vectorized inputs are supported.  The use of these operators and functions is described in section~\ref{sec:ImplicitFactorCreation}.

\begin{longtable} {l p{3cm} p{1cm} p{1cm} p{1cm} l p{4cm}}
Operator & Factor & Out & In1 & In2 & Vectorized & Description \\
\hline
\endhead
%
$\&$ & And & B & B & B & \checkmark & Logical AND \\
$|$ & Or & B & B & B & \checkmark & Logical OR \\
xor() & Xor & B & B & B & \checkmark & Logical XOR \\
$\sim$ & Not & B & B & - & \checkmark & Logical NOT \\
$+$ & Sum & D,R\footnote{\label{ftn:outReal}If either input is Real, then the output is Real} & D,R & D,R & \checkmark & Plus \\
$-$ & Subtract & D,R\textsuperscript{\ref{ftn:outReal}} & D,R & D,R & \checkmark & Minus \\
$-$ & Negate & D,R\textsuperscript{\ref{ftn:outReal}} & D,R & - & \checkmark & Unary minus \\
$*$ & Product \newline MatrixVectorProduct & D,R\textsuperscript{\ref{ftn:outReal}} & D,R & D,R & \checkmark\footnote{One of the inputs may be a vector as long as the other is a scalar.} & Scalar multiply, or matrix-vector multiply\footnote{If one input is a vector and the other is a matrix of appropriate dimension, then the MatrixVectorProduct factor will be used.  Otherwise the Product factor will be used.} \\
$.*$ & Product & D,R\textsuperscript{\ref{ftn:outReal}} & D,R & D,R & \checkmark & Point-wise multiply \\
$/$ & Divide & D,R\textsuperscript{\ref{ftn:outReal}} & D,R & D,R & \checkmark\footnote{The dividend may be a vector as long as the divisor is a scalar.} & Scalar divide \\
$./$ & Divide & D,R\textsuperscript{\ref{ftn:outReal}} & D,R & D,R & \checkmark & Point-wise divide \\
$\wedge$ & Power & D,R\textsuperscript{\ref{ftn:outReal}} & D,R & D,R & \checkmark\footnote{The base may be a vector as long as the exponent is a scalar.} & Scalar power \\
$.\wedge$ & Power & D,R\textsuperscript{\ref{ftn:outReal}} & D,R & D,R & \checkmark & Point-wise power \\
$<$ & LessThan & B & D,R & D,R & \checkmark & Less than \\
$>$ & GreaterThan & B & D,R & D,R & \checkmark & Greater than \\
$<=$ & GreaterThan\footnote{Uses GreaterThan factor, reversing the order.} & B & D,R & D,R & \checkmark & Less than or equal to \\
$>=$ & LessThan\footnote{Uses LessThan factor, reversing the order.} & B & D,R & D,R & \checkmark & Greater than or equal to \\
Equals() & Equals & B & B,D,R & B,D,R\footnote{\label{ftn:equals}This function is not limited to two inputs, but can take an arbitrary number of inputs} & \checkmark & Equals\footnote{Equivalent to the $==$ operator, but the $==$ operator is not overloaded for this purpose so that it can instead be used to determine whether or not two variables reference the same Dimple variable.} \\
NotEquals() & NotEquals & B & B,D,R & B,D,R$^{\ref{ftn:equals}}$ & \checkmark & Not equals\footnote{Equivalent to the $\sim=$ operator, but the $\sim=$ operator is not overloaded for this purpose so that it can instead be used to determine whether or not two variables reference the same Dimple variable.} \\
mod() & - & D  & D & D & \checkmark & Modulo function\footnote{Currently, the mod() operator supports discrete variables only, and it uses the MATLAB definition of mod on negative numbers.  This may be subject to change in future versions.} \\
abs() & Abs & D,R\textsuperscript{\ref{ftn:outReal}} & D,R & - & \checkmark & Absolute value \\
sqrt() & Sqrt & R & R & - & \checkmark & Square root \\
log() & Log & R & R & - & \checkmark & Natural log \\
exp() & Exp & R & R & - & \checkmark & Exponential function \\
sin() & Sin & R & R & - & \checkmark & Sine \\
cos() & Cos & R & R & - & \checkmark & Cosine \\
tan() & Tan & R & R & - & \checkmark & Tangent \\
asin() & ASin & R & R & - & \checkmark & Arc-sine \\
acos() & ACos & R & R & - & \checkmark & Arc-cosine \\
atan() & ATan & R & R & - & \checkmark & Arc-tangent \\
sinh() & Sinh & R & R & - & \checkmark & Hyperbolic sine \\
cosh() & Cosh & R & R & - & \checkmark & Hyperbolic cosine \\
tanh() & Tanh & R & R & - & \checkmark & Hyperbolic tangent \\
\end{longtable}



