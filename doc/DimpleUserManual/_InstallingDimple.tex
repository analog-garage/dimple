\section{Installing Dimple}


\subsection{Installing Binaries}

Users can follow these instructions to install Dimple.

\ifmatlab

\begin{enumerate}
\item MATLAB of at least version 2013b is required.
\item Download the latest version of Dimple from http://dimple.probprog.org.
\item Extract the Dimple zip file.
\item Execute the startup.m script in the resulting Dimple directory to 
load Dimple in MATLAB.
\item To avoid having to manually change to this directory and execute 
this script every time you start MATLAB, you will need to add the following lines to MATLAB's startup.m file:
\begin{lstlisting}
cd <Path-to-Dimple>
startup
\end{lstlisting}
Google ``MATLAB startup.m'' for more details regarding startup.m files.
\item Verify the installation:
\begin{enumerate}
\item Start MATLAB
\item  At the MATLAB command prompt type:
\begin{lstlisting} 
testDimple;
\end{lstlisting}
\item Verify the output ends with something like the following (showing that all tests passed): \\
\begin{minipage}{\textwidth}
\begin{lstlisting} 


**********************************************************************
PASSED ALL TESTS
147 of 147 tests passed, 0 failed
**********************************************************************

--testDimple
======================================================================
\end{lstlisting}
\end{minipage}

\end{enumerate}
\end{enumerate}

\fi

\ifjava

\begin{enumerate}
\item Requires Java 7.
\item Download the latest version of Dimple from http://dimple.probprog.org.  The latest binaries can be found at http://dimple.probprog.org/download listed as "Dimple Binaries".
\item Extract the Dimple  zip file
\item If you are developing using Eclipse
\begin{enumerate}
\item If you don't already have a Java project, create one.
\item Open Project-\textgreater Properties.
\item Select Java Build Path
\item Select Libraries
\item Select "Add External JARs..."
\item browse to \textless dimple directory\textgreater /solvers/lib, select all the jars and click open.
\item browse to \textless dimple directory\textgreater /solvers/non-maven-jars, select all the jars and click open.
\item You should now be able to instantiate and use Dimple classes in your project.
\end{enumerate}
\item If you are compiling from the command line, add the following directories to the JAVACLASSPATH
\begin{itemize}
\item \textless dimple directory\textgreater /solvers/lib
\item \textless dimple directory\textgreater /solvers/non-maven-jars
\end{itemize}
\end{enumerate}

\fi

\subsection{Installing from Source}

Developers interested in investigating Dimple source code, helping with bug fixes, or contributing to the source code can install Dimple from source.  Developers only interested in using Dimple should install from binaries (described in the previous section).

\begin{enumerate}
\item Download the source from https://github.com/AnalogDevicesLyricLabs/dimple
\item Install Gradle from http://www.gradle.org/.  (Gradle is a Java build tool that pulls down jars from Maven repositories.)
\item Change to \textless dimple directory\textgreater /solvers/java 
\item Run gradle by typing ``gradle''
\end{enumerate}

If you want to edit java files with Eclipse:

\begin{enumerate}
\item From eclipse, Import-\textgreater Existing Projects Into Workspace
\item Browse to the dimple directory, select sovers/java, and click Finish.
\end{enumerate}



\ifmatlab

\subsection{Adjusting MATLAB's Java Memory Limit}

Each object in Dimple corresponds to underlying Java objects. The amount of heap memory reserved for Java (when called from MATLAB) is limited, and typically low.  In some cases, this can cause Dimple to fail if the memory it requires exceeds this modest limit.  To increase the value of this limit, edit the file java.opts in the MATLAB startup directory, and add the following two lines:

\begin{lstlisting}
-Xmx1024m
-Xms512m
\end{lstlisting}

The value after Xmx is the maximum amount of heap memory allocated to Java, and Xms is the starting value.  You may use wish to use larger values if your system has sufficient memory.  Google ``MATLAB java.opts file'' to determine the specific location of this file on your operating system.

\fi


