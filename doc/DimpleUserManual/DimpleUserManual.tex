\documentclass[11pt, oneside]{article}   	% use "amsart" instead of "article" for AMSLaTeX format
\usepackage{geometry}                		% See geometry.pdf to learn the layout options. There are lots.
\geometry{letterpaper}                   		% ... or a4paper or a5paper or ... 
%\geometry{landscape}                		% Activate for for rotated page geometry
%\usepackage[parfill]{parskip}    		% Activate to begin paragraphs with an empty line rather than an indent
\usepackage{verbatim}
\usepackage{color}
\usepackage{graphicx}				% Use pdf, png, jpg, or eps with pdflatex; use eps in DVI mode
								% TeX will automatically convert eps --> pdf in pdflatex		
\usepackage{amssymb}
\usepackage{graphicx}
\usepackage{listings}
\usepackage{longtable}
\usepackage[titletoc]{appendix}		% titletoc adds 'Appendix' before the letter in the TOC
\usepackage{hyperref}
\usepackage{etoolbox}

\parskip = 0.15in
\parindent = 0.0in
\textwidth = 6 in
\textheight = 9 in
\headheight = 0.0 in
\headsep = 0.0 in


\interfootnotelinepenalty=10000		% Try to keep footnotes from splitting across pages
\setcounter{secnumdepth}{6}			% Number paragraphs and subparagraphs
\newcommand{\para}[1]{\paragraph{#1}$\:$\\}				% Line-break after paragraph heading
\newcommand{\subpara}[1]{\subparagraph{#1}$\:$\\}		% Line-break after sub-paragraph heading
\newcommand{\paraNoToc}[1]{\paragraph*{#1}$\:$\\}		% Line-break after paragraph heading, not in TOC
\newcommand{\subparaNoToc}[1]{\subparagraph*{#1}$\:$\\}	% Line-break after sub-paragraph heading, not in TOC

\definecolor{dkgreen}{rgb}{0,0.6,0}
\definecolor{gray}{rgb}{0.5,0.5,0.5}
\definecolor{mauve}{rgb}{0.58,0,0.82}


% Attempt to define subsubparagraphs, which does not properly create the TOC used for navigation in a PDF reader
% Commented out until it can be made to work
%% Subsubparagraphs
%\makeatletter
%\newcounter{subsubparagraph}[subparagraph]
%\renewcommand\thesubsubparagraph{%
%  \thesubparagraph.\@arabic\c@subsubparagraph}
%\newcommand\subsubparagraph{%
%  \@startsection{subsubparagraph}    % counter
%    {6}                              % level
%    {\parindent}                     % indent
%    {3.25ex \@plus 1ex \@minus .2ex} % beforeskip
%    {-1em}                           % afterskip
%    {\normalfont\normalsize\bfseries}}
%\newcommand\l@subsubparagraph{\@dottedtocline{6}{13em}{6em}}
%\newcommand{\subsubparagraphmark}[1]{}
%\makeatother
%\newcommand{\subsubpara}[1]{\subsubparagraph{#1}$\:$\\}			% Line-break after sub-paragraph heading
%\newcommand{\subsubparaNoToc}[1]{\subsubparagraph*{#1}$\:$\\}	% Line-break after sub-paragraph heading, not in TOC


\lstset{ %
  language=Matlab,                % the language of the code
  basicstyle=\footnotesize\ttfamily,           % the size of the fonts that are used for the code
  numbers=none,                   % where to put the line-numbers
  % numbers=left,                   % where to put the line-numbers
  numberstyle=\tiny\color{gray},  % the style that is used for the line-numbers
  stepnumber=1,                   % the step between two line-numbers. If it's 1, each line 
                                  % will be numbered
  numbersep=5pt,                  % how far the line-numbers are from the code
  backgroundcolor=\color{white},      % choose the background color. You must add \usepackage{color}
  showspaces=false,               % show spaces adding particular underscores
  showstringspaces=false,         % underline spaces within strings
  showtabs=false,                 % show tabs within strings adding particular underscores
  frame=single,                   % adds a frame around the code
  rulecolor=\color{black},        % if not set, the frame-color may be changed on line-breaks within not-black text (e.g. comments (green here))
  tabsize=2,                      % sets default tabsize to 2 spaces
  captionpos=t,                   % sets the caption-position to bottom
  breaklines=true,                % sets automatic line breaking
  breakatwhitespace=false,        % sets if automatic breaks should only happen at whitespace
  title=\lstname,                   % show the filename of files included with \lstinputlisting;
                                  % also try caption instead of title
  keywordstyle=\color{blue},          % keyword style
  commentstyle=\color{dkgreen},       % comment style
  stringstyle=\color{mauve},         % string literal style
  escapeinside={\%*}{*},            % if you want to add LaTeX within your code
  morekeywords={*,...}               % if you want to add more keywords to the set
}


\ifx\version\undefined
  \def\version{TBD}
\fi

\newif\ifjava
\newif\ifmatlab

\ifdefined\forjava
	\javatrue
\else
	\ifdefined\formatlab
		\matlabtrue
	\else		
                %modify this flag to matlabtrue or javatrue when editing with TexShop
                %When building from the commandline this will be ignored.
%		\javatrue
		\matlabtrue
	\fi
\fi

\ifjava
\lstset{language=Java}
\fi

\ifjava
\title{\huge \bfseries Dimple User Manual \linebreak Java Interface \\ \normalfont \large Version \version}
\fi

\ifmatlab
\title{\huge \bfseries Dimple User Manual \linebreak MATLAB Interface \\ \normalfont \large Version \version}
\fi

%\date{}							% Activate to display a given date or no date

\begin{document}


\maketitle



% Attempt to have a brief contents overview prior to the full contents
% Commented out until it can be made to work
%\clearpage
%\renewcommand{\contentsname}{Brief Contents}
%\setcounter{tocdepth}{1}				% Include only the top-level sections in the brief TOC
%\tableofcontents

\clearpage
%\renewcommand{\contentsname}{Contents}
\setcounter{tocdepth}{5}				% Include paragraphs and subparagraphs in TOC
\tableofcontents

\clearpage

\section{What is Dimple?}

Dimple is an open-source API for probabilistic modeling and inference. Dimple allows the user to specify probabilistic models in the form of graphical models (factor graphs, Bayesian networks, or Markov networks), and performs inference on the model using a variety of supported algorithms.

Probabilistic graphical models unify a great number of models from machine learning, statistical text processing, vision, bioinformatics, and many other fields concerned with the analysis and understanding of noisy, incomplete, or inconsistent data. Graphical models reduce the complexity inherent in complex statistical models by dividing them into a series of logically (and statistically) independent components. By factoring the problem into sub-problems with known and simple interdependencies, and by adopting a common language to describe each sub-problem, one can considerably simplify the task of creating complex probabilistic models.  A brief tutorial on graphical models can be found in Appendix~\ref{sec:IntroductionToFactorGraphs}.

An important attribute of Dimple is that it allows the user to construct probabilistic models in a form that is largely independent of the algorithm used to perform inference on the model. This modular architecture benefits those who create probabilistic models by freeing them from the complexities of the inference algorithms, and it benefits those who develop new inference algorithms by allowing these algorithms to be implemented independently from any particular model or application.

Key features of Dimple:
\begin{itemize}
\item Supports both undirected and directed graphs.
\item Supports a variety of \emph{solvers} for performing inference, including sum-product and Gaussian belief propagation (BP), min-sum BP, particle BP, discrete junction tree, linear programming (LP), and Gibbs sampling.
\item Supports both discrete and continuous variables.
\item Supports arbitrary factor functions as well as a growing library of standard distributions and mathematical functions.
\item Supports nested graphs.
\item Supports rolled-up graphs (repeated HMM-like structures).
\item Growing support for parameter estimation (including the EM algorithm).
\item Supports both MATLAB\footnote{MATLAB is a registered trademark of The Mathworks, Inc.} and Java\footnote{Java is a registered trademark of Oracle and/or its affiliates.} API.
\end{itemize}

\ifmatlab
Using MATLAB to program Dimple requires an existing licensed version of MATLAB, of version at least 2013b.
\fi

\section{Installing Dimple}


\subsection{Installing Binaries}

Users can follow these instructions to install Dimple.

\ifmatlab

\begin{enumerate}
\item MATLAB of at least version 2013b is required.
\item Download the latest version of Dimple from http://dimple.probprog.org.
\item Extract the Dimple zip file.
\item Execute the startup.m script in the resulting Dimple directory to 
load Dimple in MATLAB.
\item To avoid having to manually change to this directory and execute 
this script every time you start MATLAB, you will need to add the following lines to MATLAB's startup.m file:
\begin{lstlisting}
cd <Path-to-Dimple>
startup
\end{lstlisting}
Google ``MATLAB startup.m'' for more details regarding startup.m files.
\item Verify the installation:
\begin{enumerate}
\item Start MATLAB
\item  At the MATLAB command prompt type:
\begin{lstlisting} 
testDimple;
\end{lstlisting}
\item Verify the output ends with something like the following (showing that all tests passed): \\
\begin{minipage}{\textwidth}
\begin{lstlisting} 


**********************************************************************
PASSED ALL TESTS
147 of 147 tests passed, 0 failed
**********************************************************************

--testDimple
======================================================================
\end{lstlisting}
\end{minipage}

\end{enumerate}
\end{enumerate}

\fi

\ifjava

\begin{enumerate}
\item Requires Java 7.
\item Download the latest version of Dimple from http://dimple.probprog.org.  The latest binaries can be found at http://dimple.probprog.org/download listed as "Dimple Binaries".
\item Extract the Dimple  zip file
\item If you are developing using Eclipse
\begin{enumerate}
\item If you don't already have a Java project, create one.
\item Open Project-\textgreater Properties.
\item Select Java Build Path
\item Select Libraries
\item Select "Add External JARs..."
\item browse to \textless dimple directory\textgreater /solvers/lib, select all the jars and click open.
\item browse to \textless dimple directory\textgreater /solvers/non-maven-jars, select all the jars and click open.
\item You should now be able to instantiate and use Dimple classes in your project.
\end{enumerate}
\item If you are compiling from the command line, add the following directories to the JAVACLASSPATH
\begin{itemize}
\item \textless dimple directory\textgreater /solvers/lib
\item \textless dimple directory\textgreater /solvers/non-maven-jars
\end{itemize}
\end{enumerate}

\fi

\subsection{Installing from Source}

Developers interested in investigating Dimple source code, helping with bug fixes, or contributing to the source code can install Dimple from source.  Developers only interested in using Dimple should install from binaries (described in the previous section).

\begin{enumerate}
\item Download the source from https://github.com/AnalogDevicesLyricLabs/dimple
\item Install Gradle from http://www.gradle.org/.  (Gradle is a Java build tool that pulls down jars from Maven repositories.)
\item Change to \textless dimple directory\textgreater /solvers/java 
\item Run gradle by typing ``gradle''
\end{enumerate}

If you want to edit java files with Eclipse:

\begin{enumerate}
\item From eclipse, Import-\textgreater Existing Projects Into Workspace
\item Browse to the dimple directory, select sovers/java, and click Finish.
\end{enumerate}



\ifmatlab

\subsection{Adjusting MATLAB's Java Memory Limit}

Each object in Dimple corresponds to underlying Java objects. The amount of heap memory reserved for Java (when called from MATLAB) is limited, and typically low.  In some cases, this can cause Dimple to fail if the memory it requires exceeds this modest limit.  To increase the value of this limit, edit the file java.opts in the MATLAB startup directory, and add the following two lines:

\begin{lstlisting}
-Xmx1024m
-Xms512m
\end{lstlisting}

The value after Xmx is the maximum amount of heap memory allocated to Java, and Xms is the starting value.  You may use wish to use larger values if your system has sufficient memory.  Google ``MATLAB java.opts file'' to determine the specific location of this file on your operating system.

\fi



\section{Getting Started: Basic Examples}

The following sections demonstrate Dimple with four basic examples. The first example is a simple hidden Markov model. The second models a 4-bit XOR constraint. The third demonstrates how to use nested graphs. The final example is a simple LDPC code.  

\ifmatlab
See /demo/12\_Tutorial for the code.  
\fi

\ifjava
All java examples referenced in this section can be found in /solvers/java/src/main/java/com/analog/lyric/dimple/examples if Dimple has been installed from source.
\fi

\subsection{A Hidden Markov Model}

We consider a very simple hidden Markov model (HMM), the Rainy/Sunny HMM illustrated in the Wikipedia article about HMMs. Two friends who live far apart, Alice and Bob, have a daily phone conversation during which Bob mentions what he has been doing during the day. Alice knows that Bob�s activity on a given day depends solely on the weather on that day, and knows some general trends about the evolution of weather in Bob�s area.

Alice believes she can model the weather in Bob�s area as a Markov chain with two states �Sunny� and �Rainy�. She remembers hearing that on the first day of last week it was quite likely (70\% chance) that it was sunny in Bob�s town. She also knows that a sunny day follows another sunny day with 80\% chance, while a sunny day succeeds a rainy day with 50\% chance.

She knows Bob pretty well, and knows that Bob only ever does one of three things: stay inside to read a book, go for a walk, or cook. She knows that if it is sunny, Bob will go for a walk with 70\% probability, cook with 20\% probability, and stay inside with 10\% probability. Conversely, if it is rainy, Bob will go for a walk with 20\% probability, cook with 40\% probability, and stay inside with 40\% probability. 

Bob told Alice that he went for a walk on Monday, Tuesday, and Thursday, cooked on Wednesday and Friday, and read a book on Saturday and Sunday.

Alice wants to know what the most likely weather is for every day of last week. The above naturally defines an HMM which can easily be modeled within Dimple



\subsubsection*{Creating the Factor Graph}
The first command in a Dimple program, is to create a factor graph. This is easily done with the FactorGraph() command. 

\ifmatlab

See \/demo/12\_Tutorial/DimpleTutorial\_HMM.m

\fi

\ifjava

See solvers/java/src/main/java/com/analog/lyric/dimple/examples/HMM.java.

\fi


\ifmatlab

\begin{lstlisting}
HMM = FactorGraph();
\end{lstlisting}

\fi


\ifjava

\begin{lstlisting}
FactorGraph HMM = new FactorGraph();
\end{lstlisting}

\fi

\subsubsection*{Declaring Variables}
Once a Factor Graph has been defined, we can define the variables of the factor graph.  In our case, there will be seven variables, MondayWeather� to SundayWeather. The command to create a variable is Discrete(domain,dimensions). In our case, the domain will consist of the two distinct values �Sunny� and �Rainy�. For now, we will not specify dimensions (this will be covered in 4 Bit XOR example). The domain should either be a cell, or a matrix of numbers.

\ifmatlab

\begin{lstlisting}
Domain={'sunny','rainy'};
MondayWeather=Discrete(Domain);
TuesdayWeather=Discrete(Domain);
WednesdayWeather=Discrete(Domain);
ThursdayWeather=Discrete(Domain);
FridayWeather=Discrete(Domain);
SaturdayWeather=Discrete(Domain);
SundayWeather=Discrete(Domain);
\end{lstlisting}

IMPORTANT: In the above, had we used the declaration Domain=['sunny','rainy'] instead (square brackets instead of curly brackets), the domain would have consisted of 10 letters instead of 2 strings (i.e., the variables would have been 10-ary instead of binary).

\fi

\ifjava

\begin{lstlisting}
Discrete MondayWeather = new Discrete(domain);
Discrete TuesdayWeather = new Discrete(domain);
Discrete WednesdayWeather = new Discrete(domain);
Discrete ThursdayWeather = new Discrete(domain);
Discrete FridayWeather = new Discrete(domain);
Discrete SaturdayWeather = new Discrete(domain);
Discrete SundayWeather = new Discrete(domain);
\end{lstlisting}

\fi

\subsubsection*{Adding Factors}

We now add the different factors of the factor graph. We will first add the factors corresponding to the Markov Chain structure. This is done with the command addFactor, which is a method of the factor graph previously defined. 

\ifmatlab

The command has syntax addFactor(funchandle,arguments), where funchandle is the handle of a regular MATLAB function (which can be specified by the user), and the arguments are the variables involved in the factor being defined (in the same order as the inputs of the MATLAB function). The number of inputs of the function referred to by the function handle has to be equal to the number of arguments of addFactor.


In our case, we define a simple MC\_transition\_Tutorial(state1,state2) as follows (See /demos/12\_Tutorial/MC\_transition\_Tutorial.m):


\begin{lstlisting}
function [probability]=MC_transition_Tutorial(state1,state2)
 
switch state1
    case 'sunny'
        if state2=='sunny'
            probability=0.8;
        else
            probability=0.2;
            
        end
        
    case 'rainy'
        probability =0.5;
end 
\end{lstlisting}

\fi

\ifjava

The command has syntax addFactor(factorFunction,arguments), where factorFunction is an instance of a class that inherits from FactorFunction.  The arguments are the variables involved in the factor being defined (in the same order as the inputs of the veal function of the Factor Function). The number of inputs of the function referred to by the function handle has to be equal to the number of arguments of addFactor.

In our case, we define a class called TransitionFactorFunction.

\begin{lstlisting}
	public static class TransitionFactorFunction extends FactorFunction
	{
		public TransitionFactorFunction()
		{
			super("TransitionFactorFunction");
		}
		
		@Override
		public double eval(Object ... args)
		{
			String state1 = (String)args[0];
			String state2 = (String)args[1];
			
			if (state1.equals("sunny"))
			{
				if (state2.equals("sunny"))
				{
					return 0.8;
				}
				else
				{
					return 0.2;
				}
				
			}
			else
			{
				return 0.5;
			}
			
		}
	}
\end{lstlisting}

\fi

We can now add the factor to the factor graphs by using the addFactor method:

\ifmatlab

\begin{lstlisting}
HMM.addFactor(@MC_transition_Tutorial,MondayWeather,TuesdayWeather);
HMM.addFactor(@MC_transition_Tutorial,TuesdayWeather,WednesdayWeather);
HMM.addFactor(@MC_transition_Tutorial,WednesdayWeather,ThursdayWeather);
HMM.addFactor(@MC_transition_Tutorial,ThursdayWeather,FridayWeather);
HMM.addFactor(@MC_transition_Tutorial,FridayWeather,SaturdayWeather);
HMM.addFactor(@MC_transition_Tutorial,SaturdayWeather,SundayWeather);
\end{lstlisting}

\fi

\ifjava

\begin{lstlisting}
TransitionFactorFunction trans = new TransitionFactorFunction();
HMM.addFactor(trans, MondayWeather,TuesdayWeather);
HMM.addFactor(trans, TuesdayWeather,WednesdayWeather);
HMM.addFactor(trans, WednesdayWeather,ThursdayWeather);
HMM.addFactor(trans, ThursdayWeather,FridayWeather);
HMM.addFactor(trans, FridayWeather,SaturdayWeather);
HMM.addFactor(trans, FridayWeather,SundayWeather);
\end{lstlisting}

The java code first instantiates the factor function before connecting all of the variables.  It would be possible to create a new instance of the factor function for each addFactor call, but this would use unnecessary memory.

\fi

We now need to add factors corresponding to the observations of each day. As it happens, when using an addFactor command, the arguments need not be all random variables---some can be declared as constants. We see now how to use this fact to easily add the observations to each day. We first declare an observation function 

\ifmatlab

see /demo/12\_Tutorial/observation\_function\_Tutorial.m:


\begin{lstlisting}
function [probability]=observation_function_Tutorial(state,observation)
 
switch state    
    case 'sunny'
        
        switch observation
            
            case 'walk'
                probability=0.7;
            case 'book'
                probability=0.1;
            case 'cook'
                probability=0.2;
        end
        
    case 'rainy'
        
        switch observation
           
            case 'walk'
                probability=0.2;
            case 'book'
                probability=0.4;
            case 'cook'
                probability=0.4;
        end
end
\end{lstlisting}

\fi

\ifjava

see the ObservationFactorFunction from solvers/java/src/main/java/com/analog/lyric/dimple/examples/HMM.java.

\begin{lstlisting}
	public static class ObservationFactorFunction extends FactorFunction
	{
		public ObservationFactorFunction()
		{
			super("ObservationFactorFunction");
		}
		
		@Override
		public double eval(Object ... args)
		{
			String state = (String)args[0];
			String observation = (String)args[1];
			
			if (state.equals("sunny"))
				if (observation.equals("walk"))
					return 0.7;
				else if (observation.equals("book"))
					return 0.1;
				else // cook
					return 0.2;
			else
				if (observation.equals("walk"))
					return 0.2;
				else if (observation.equals("book"))
					return 0.4;
				else // cook
					return 0.4;
		}
	}
\end{lstlisting}

\fi

Adding the observations is then very easy:

\ifmatlab
\begin{lstlisting}
HMM.addFactor(@observation_function_Tutorial,MondayWeather,'walk');
HMM.addFactor(@observation_function_Tutorial,TuesdayWeather,'walk');
HMM.addFactor(@observation_function_Tutorial,WednesdayWeather,'cook');
HMM.addFactor(@observation_function_Tutorial,ThursdayWeather,'walk');
HMM.addFactor(@observation_function_Tutorial,FridayWeather,'cook');
HMM.addFactor(@observation_function_Tutorial,SaturdayWeather,'book');
HMM.addFactor(@observation_function_Tutorial,SundayWeather,'book');
\end{lstlisting}
\fi

\ifjava

\begin{lstlisting}
ObservationFactorFunction obs = new ObservationFactorFunction();
HMM.addFactor(obs,MondayWeather,"walk");
HMM.addFactor(obs,TuesdayWeather,"walk");
HMM.addFactor(obs,WednesdayWeather,"cook");
HMM.addFactor(obs,ThursdayWeather,"walk");
HMM.addFactor(obs,FridayWeather,"cook");
HMM.addFactor(obs,SaturdayWeather,"book");
HMM.addFactor(obs,SundayWeather,"book");
\end{lstlisting}

\fi

As we can see, though the function itself depends on two variables, each factor only depends on one random variable (the other argument being set as a constant during the addFactor call). This in effect creates a factor connected only to one variable of the factor graph.

\ifmatlab

There is a cost incurred every time addFactor is called in MATLAB.  For large Factor Graphs with many factors of the same type, it can be advantageous to use the MATLAB vectorized version of addFactor.  The following code will create a transition factor between each pair of consecutive variables and will execute much more quickly than a for loop with addFactor.


\begin{lstlisting}
N = 1000;
Weather = Discrete(Domain,N,1);
HMM.addFactorVectorized(@MC_transition_Tutorial,Weather(1:(end-1)),Weather(2:end));
\end{lstlisting}

This works with continuous variables and Nested Graphs as well.  In addition, it's possible to specify which dimensions to vectorize over:


\begin{lstlisting}
N = 20;
b = Bit(3,N);
fg = FactorGraph();
fg.addFactorVectorized(@xorDelta,{b,2});
\end{lstlisting}

The previous code will create N xor factors across each of the N sets of 3 variables.

\fi

\subsubsection*{Specifying Inputs}


The last step consists in adding the prior for the MondayWeather variable. We could, as above, use a factor with a single variable. Let us introduce a new command to easily add a single variable factor---the input method (on variables).

For a vector of probabilities (i.e., nonnegative numbers which sums up to one), the method \ifmatlab Variable.Input \fi \ifjava Variable.setInput \fi adds a single factor which depends on this variable only, with values equal to those given by the vector.

In our case, we do:

\ifmatlab

\begin{lstlisting}
MondayWeather.Input=[0.7 0.3];
\end{lstlisting}

\fi

\ifjava

\begin{lstlisting}
MondayWeather.setInput(0.7,0.3);
\end{lstlisting}

\fi

The Input command can be used in several different ways. Some notes of interest regarding the Input command:

\begin{itemize}
\item The Input command can typically be used for prior probabilities of the root variables in a Bayes Net, or for the initial node of a Markov Chain or HMM.
\item The Input command can also be used for any factors with only one variable, for instance, for observation factors (see the introduction to factor graphs on how to remove observations and turn them into single node factors).
\item The \ifmatlab observation\_function\_Tutorial \fi \ifjava ObservationFactorFunction \fi in the above example was not entirely required (see below)---we could have used the input command instead.
\item IMPORTANT: Unlike the addFactor command, the Input command can be used only once for each variable. That is, once you have specified an input for a variable, re-specifying this input will destroy the previous factor and create a new one. In the example above, using only the Input command, it would not have been possible to separately incorporate both the prior on Monday�s weather and the factor corresponding to Monday�s observation. However, this feature is very useful when Input is used to specify external information, and when the user wants to see the effect of external information. Say for instance that Bob mentions to Alice that it rained on Wednesday. Alice can simply use the command \ifmatlab WednesdayWeather.Input=[0 1] \fi \ifjava WednesdayWeather.setInput(0,1)\fi. If Bob corrects himself and says he misremembered, and that it actually was sunny that day, Alice can correct the information using again the command \ifmatlab WednesdayWeather.Input=[1 0] \fi \ifjava WednesdayWeather.setInput(1,0)\fi.
\end{itemize}


\subsubsection*{Solving the Factor Graph}

Finally, we explain how to solve the factor graph by using the solve, iterate, and NumIterations factor graph methods.

The typical way of solving a factor graph will be by choosing the number of iterations and then running the solver for the desired number of iterations. In our case, this is simply done by typing the following commands:

\ifmatlab

\begin{lstlisting}
HMM.NumIterations=20;
HMM.solve;
\end{lstlisting}

\fi

\ifjava

\begin{lstlisting}
HMM.getSolver().setNumIterations(20);
HMM.solve();
\end{lstlisting}

\fi

IMPORTANT:

By default, the solver will use either a Flooding Schedule or a Tree Schedule depending on whether the factor-graph contains cycles.  A loopy graph (one containing at least one cycle) will use a Flooding Schedule by default.  This schedule can be described as:

\begin{itemize}
\item Compute all variable nodes
\item Compute all factor nodes
\item Repeat for a specified number of iterations
\end{itemize}

If the factor-graph is a tree (one that contains no cycles), the solver will automatically detect this and use a Tree Schedule by default.  In this schedule, each node is updated in an order that will result in the correct beliefs being computed after just one iteration.  In this case, NumIterations should be 1, which is its default value.

\subsubsection*{Accessing Results}
Once the solver has finished running, we can access the marginal distribution of each variable by using the Belief command

\ifmatlab

\begin{lstlisting}
belief  = TuesdayWeather.Belief;
\end{lstlisting}

\fi

\ifjava

\begin{lstlisting}
double [] belief = TuesdayWeather.getBelief();
\end{lstlisting}

\fi

This returns a vector of probability of the same length as the domain total size (i.e., the product of its dimensions), with the probability of each domain variable.
Another way to solve the factor graph is to use the Solver.iterate(n) command, which runs the factor graph for n iterations (without arguments, it runs for 1 iteration).

\ifmatlab

\begin{lstlisting}
HMM.Solver.iterate();
HMM.Solver.iterate(5);
\end{lstlisting}

\fi

\ifjava

\begin{lstlisting}
HMM.getSolver().iterate();
HMM.getSolver().iterate(5);
\end{lstlisting}

\fi

IMPORTANT: The iterate command is useful to access intermediate results (i.e, see how beliefs change through the iterations).

IMPORTANT: One distinction between the solve and iterate commands is that all messages and beliefs are reinitialized when starting the solve command. Running solve twice in a row is therefore identical to running it once, unlike iterate.  When calling iterate() for the first time, first call initialize(), which reinitialize all messages.


\subsection{A 4-bit XOR}

The following example creates a simple factor graph with 4 variables tied through a 4-bit XOR, with �priors� (we abuse language here and call �prior� the probability distribution of each random variable if they were not tied through the 4-bit XOR).

Through this example, we will learn how to define arrays of random variables, see how to use MATLAB indexing within Dimple, see an example of a hard constraint in a factor graph, and see how to use the Bit type of random variable.

We consider a set of four random variables (B1,B2,B3,B4) taking values 0 or 1. The joint probability distribution is given by:


\[
 Pr(B_1,B_2,B_3,B_4) = \delta_{B_1+B_2+B_3+B_4=0(mod2)}  P(B_1)P(B_2)P(B_3)P(B_4)  
\]


where the delta function   is equal to 1 if the underlying constraint is 
satisfied, and 0 otherwise (this ensures that illegal assignment have 
probability zero). The $P(B_i)$ are single variable factors which help model 
which values of $ B_i $  are more likely to be taken (we call them �priors�, 
though, once again, this is an abuse of language. Typically, the factor
 will represent an observation of  $ O_i $ of variable  $ B_i $, and the factor 
$ P(B_i) $  is 
set equal to the normalized function  $ P(O_i | B_i ) $ \footnote { Normalizing $ P(O_i|B_i) $ happens to be equal to $P(B_i|O_i)$ in a
factor graph with only the two variables $O_i$ and $B_i$ with a prior on both values of 
$B_i$ being equally likely. }



For our example, we will choose  $P(B_1 =1)=P(B_2=1)=P(B_3=1)=.8$ and  
$P(B_4=1)$ = 0.5.

We will build our factor graph in two different ways. The first directly uses a 4-bit XOR, and uses mostly tools seen in the previous example, while the second introduces the Bit kind of random variable, and how to use an intermediate variable to reduce the degree of a factor graph with parity checks (i.e., XOR function).

The first way of building the factor graph uses an inefficient N-bit XOR function defined as follows 

\ifmatlab

From /demo/12\_Tutorial/BadNBitXorDelta\_Tutorial.m:

\begin{lstlisting}
function [valid]=BadNBitXorDelta_Tutorial(variables)
 
valid=mod(sum(variables),2)==0;

end
\end{lstlisting}

\fi

\ifjava


See solvers/java/src/main/java/com/analog/lyric/dimple/examples/BadNBitXor.java

\begin{lstlisting}
	public static class BadNBitXorFactor extends FactorFunction
	{
		@Override
		public double eval(Object ... args)
		{
			int sum = 0;
			for (int i = 0; i < args.length; i++)
				sum += (Integer)args[i];
			
			return (sum % 2) == 0 ? 1 : 0;
		}
	}
\end{lstlisting}

\fi

Using everything we have learned in the previous example, the sequence of instructions we use is simply 

\ifmatlab 

From /demo/12\_Tutorial/DimpleTutorial\_BadNBitXor.m:


\begin{lstlisting}
FourBitXor=FactorGraph();
Domain=[0;1];
B1=Discrete(Domain);
B2=Discrete(Domain);
B3=Discrete(Domain);
B4=Discrete(Domain);
FourBitXor.addFactor(@BadNBitXorDelta_Tutorial, [B1,B2,B3,B4]);
B1.Input=[0.2 0.8];
B2.Input=[0.2 0.8];
B3.Input=[0.2 0.8];
B4.Input=[0.5 0.5];
FourBitXor.solve;
disp(B1.Value);
disp(B1.Belief);
\end{lstlisting}

Note that the MATLAB BadNBitXorDelta\_Tutorial is a function which takes only ONE argument, but this argument is an array. This is reflected in the declaration FourBitXor.addFactor (@BadNBitXorDelta\_Tutorial, [B1,B2,B3,B4]), where we created an array of random variables using square brackets.

\fi

\ifjava

\begin{lstlisting}
FactorGraph fourBitXor = new FactorGraph();
DiscreteDomain domain = new DiscreteDomain(0,1);
Discrete b1 = new Discrete(domain);
Discrete b2 = new Discrete(domain);
Discrete b3 = new Discrete(domain);
Discrete b4 = new Discrete(domain);
fourBitXor.addFactor(new BadNBitXorFactor(),b1,b2,b3,b4);
b1.setInput(0.2,0.8);
b2.setInput(0.2,0.8);
b3.setInput(0.2,0.8);
b4.setInput(0.5,0.5);
fourBitXor.solve();
System.out.println(b4.getValue());
System.out.println(Arrays.toString(b4.getBelief()));
\end{lstlisting}

\fi


IMPORTANT: We also introduce the Discrete method �Value�, which returns the most likely assignment of that random variable.

\ifmatlab

The first remark above is rather important, as it highlights of the concept of arrays of random variables in the MATLAB Dimple API.  Dimple handles arrays of random variables in a very natural manner, and most array indexing operations of MATLAB are supported in Dimple in a similar fashion. 

To create a multidimensional array of random variables in the MATLAB Dimple API, we use the Variable command where every argument after the first is a dimension of an array:

\begin{lstlisting}
VarArray=Variable(Domain,dimension1,dimension2,.., dimensionk) 
\end{lstlisting}

creates an array of variables with domain �Domain�, and with dimensions [dimension1][dimension2]..[ dimensionk]. 

IMPORTANT: If only one dimension is specified, Dimple creates a square array.


\begin{lstlisting}
VarArray1=Discrete(Domain,3,1)
VarArray2=Discrete(Domain,3)
VarArray3=Discrete(Domain,3,3)
\end{lstlisting}



In the example above, VarArray1 is a vector of 3 random variables with domain Domain, while VarArray2 and VarArray2 are both a 3-by-3 matrix of random variables with domain Domain.

Another way to create an array is, as above, to group the variables using square brackets:


\begin{lstlisting}
RowArray=[ B1 B2 B3 B4];
ColumnArray=RowArray�;
\end{lstlisting}

IMPORTANT: The above command is a simple �grouping� (reference) of variables  ; it does not duplicate them.

Concatenation and subindexing of random arrays work in exactly the same fashion as MATLAB.


\begin{lstlisting}
SubArray1=VarArray2(2,1:2);
SubArray2=VarArray2(1,2:3);
SubArray3=[SubArray1;SubArray2];
\end{lstlisting}

repmat works as well.  The following code snippet will set C to a 10x10 Bit matrix.  Rather than creating new
variables, it simply replicates the existing variables.  

\begin{lstlisting}
b = Bit(10,1);
c = repmat(b,1,10);
\end{lstlisting}



Using the Variable methods Belief, Value or Domain on a random array returns the array of Beliefs (resp. ML values, domains, etc..) for these variables.


\begin{lstlisting}
SubArray2.Value
\end{lstlisting}

returns a (1,2) array containing the most likely values of VarArray2(1,2) and VarArray2(1,3).

IMPORTANT: Since Beliefs or Inputs are arrays themselves, calling them on an array returns an array one dimension larger.

\fi

Often, we will find it useful to have random Bits. For that purpose, one can directly create random Bits with the command Bit. A Bit is simply a Discrete with Domain [0, 1]. Also, in order to simplify Input declarations, for a Bit variable named BitVar, BitVar.Input takes a single number, the probability of 1. 

\ifmatlab
Similarly, BitVar.Belief returns the marginal probability of BitVar being equal to 1.
\fi

The second version of the 4-bit XOR will uses Bit variables to represent our random variables. Furthermore, it will decompose the 4 bits XOR into two 3-bits XOR. It is indeed easy to prove that $ B_1 + B_2 + B_3 + B_4 = 0(mod 2) $ is equivalent to


\begin{eqnarray*}
B_1+B_2+C &=& 0(mod 2) \\
B_3+B_4+C &=& 0(mod 2)
\end{eqnarray*}


for an internal bit C. While the reduction from one 4-bit to two 3-bit XORs might not seem tremendous, it is easy to see that more generally, any N-bit XOR can be reduced to a tree of 3-bit XORs, with depth log(N). Because the complexity of running Belief Propagation is exponential in the degree of factors, using this technique leads to dramatic speed improvements.

Using all the techniques mentioned above, we derive a new factor graph for the 4-bit XOR, equivalent to the one previously defined 

\ifmatlab

From /demo/12\_Tutorial/xorDeltaTutorial.m:


\begin{lstlisting}
function valid = XorDeltaTutorial(x,y,z)       
    valid = bitXor(bitXor(x,y),z) == 0;
end
\end{lstlisting}



From demo/12\_Tutorial/DimpleTutorial\_4BitXor.m:

\begin{lstlisting}
XorGraph = FactorGraph();
b = Bit(4,1);
c = Bit();
XorGraph.addFactor(@XorDeltaTutorial,b(1),b(2),c);
XorGraph.addFactor(@XorDeltaTutorial,b(3),b(4),c);
b.Input = [ .8 .8 .8 .5];
XorGraph.solve();
disp(b.Belief);
disp(b.Value);
\end{lstlisting}

\fi

\ifjava

From solvers/java/src/main/java/com/analog/lyric/dimple/examples/ThreeBitXor.java

\begin{lstlisting}
public class ThreeBitXor extends FactorFunction
{
	@Override
	public double eval(Object ... args)
	{
		int arg0 = (Integer)args[0];
		int arg1 = (Integer)args[1];
		int arg2 = (Integer)args[2];
		
		return (arg0 ^ arg1 ^ arg2) == 0 ? 1 : 0;
	}
}
\end{lstlisting}

From solvers/java/src/main/java/com/analog/lyric/dimple/examples/FourBitXor.java

\begin{lstlisting}

	FactorGraph xorGraph = new FactorGraph();
	Bit [] b = new Bit[4];
	for (int i = 0; i < b.length; i++)
		b[i] = new Bit();
	Bit c = new Bit();
		
	ThreeBitXor xd = new ThreeBitXor();
		
	xorGraph.addFactor(xd,b[0],b[1],c); 
	xorGraph.addFactor(xd,b[2],b[3],c);
		
	double [] inputs = new double [] {.8, .8, .8, .5};
	for (int i = 0; i < inputs.length; i++)
		b[i].setInput(inputs[i]);
		
	xorGraph.solve();
		
	for (int i = 0; i < b.length; i++)
		System.out.println(b[i].getP1());
	for (int i = 0; i < b.length; i++)
		System.out.println(b[i].getValue());

\end{lstlisting}

\fi

The following figure represents the Factor Graph that is created and solved in this example.

\includegraphics{images/FactorGraphTemplate.png}

\subsection{Nested Graphs}
\label{sec:nestedGraphs}
Suppose we wish to use two 4-bit XORs from the previous example to create a 6-bit code.  The following diagram shows our desired Factor Graph.

\includegraphics{images/NestedGraph.png}

Dimple provides a way to replicate multiple copies of a Factor Graph and nest these instances in a containing Factor Graph. A nested Factor Graph can be seen as a special factor function between a set of variables (�connector variables�), which, when �zoomed in,� is in fact another factor graph, with factors involving both the connector variables, and other �internal variables.� In the second version of the XOR example, we created a 4-bit XOR connected to 4 variables, by also creating an extra Bit C. What if we wanted to use that construction as a 4-bit XOR for potentially many different sets of 4 bits, overlapping or not, by replicating the factor graph as needed? Nested factor graphs provide an elegant solution to this problem.

A nestable factor graph is created by specifying first a set   of �connector� random variables, and invoking the command FactorGraph(S).

IMPORTANT: The factor graph defined this way is still a proper factor graph, and in principle, we can run BP on it. However, in practice, it is used as a �helper� factor graph (and   are �helper� random variables), which will mostly be replicated in the actual factor graph of interest.

The following code creates a nestable factor graph, with connector variables $ (B_1,B_2,B_3,B_4) $

\ifmatlab

From /demo/12\_Tutorial/DimpleTutorialNested.m:


\begin{lstlisting}
% Define 4 bit Xor from two 3 bit Xors
%%%%%%%%%%%%%%%%%%%%%%%%%%%%%%%%%%%%%%
b = Bit(4,1);
XorGraph = FactorGraph(b);
c = Bit();
XorGraph.addFactor(@XorDeltaTutorial,b(1),b(2),c);
XorGraph.addFactor(@XorDeltaTutorial,b(3),b(4),c);
\end{lstlisting}

\fi

\ifjava

See solvers/java/src/main/java/com/analog/lyric/dimple/examples/Nested.java

\begin{lstlisting}
////////////////////////////////////// 
// Define 4 bit xor from two 3 bit xors 
////////////////////////////////////// 
Bit [] b = new Bit[4];
for (int i = 0; i < b.length; i++)
	b[i] = new Bit();
		
FactorGraph xorGraph = new FactorGraph(b);
		
Bit c = new Bit(); 
ThreeBitXor xd = new ThreeBitXor();
		
xorGraph.addFactor(xd,b[0],b[1],c); 
xorGraph.addFactor(xd,b[2],b[3],c);
\end{lstlisting}
\fi

IMPORTANT: In principle, variables attached to a Factor Graph can be defined before or after defining the factor graph (but obviously always before the factors they are connected to). However, connector variables naturally need to be defined before the nestable factor graph.

\includegraphics{images/BoundaryVariables.png}

Consider a nestable factor graph NestableFactorGraph(connectvar1, connectvar2,..,vark) with k connector variables. Consider also an actual factor graph of interest, FactorGraph, containing (among others) k variables of interest (var1,..,vark). By using the addGraph command, we can replicate the NestableFactorGraph and use it to connect the variables (var1,..,vark) in the same way the connector variables are connected in the nestable factor graph. The command is: FactorGraph.addGraph(NestableFactorGraph,var1,..,vark).

In essence, the nestable factor graph represents a factor between the dummy connector variables, and the addGraph command is adding the factor to the desired actual variables.

IMPORTANT: Nested factor graphs support arbitrary levels of nesting.  That is, a nested factor graph can be composed of nested factor graphs.

Armed with this tool, we can very simply use our custom 4-bit XOR to implement the factor graph described at the beginning of the section:

\ifmatlab

\begin{lstlisting}
%%%%%%%%%%%%%%%%%%%%%%%%%%%%%%%%%%%%%%
% Create graph for 6 bit code
%%%%%%%%%%%%%%%%%%%%%%%%%%%%%%%%%%%%%%    


d = Bit(6,1);
MyGraph = FactorGraph(d);
MyGraph.addGraph(XorGraph,d([1:3 5]));
MyGraph.addGraph(XorGraph,d([1 2 4 6]));

%%%%%%%%%%%%%%%%%%%%%%%%%%%%%%%%%%%%%%
% Set input and Solve
%%%%%%%%%%%%%%%%%%%%%%%%%%%%%%%%%%%%%%

d.Input = [.75 .6 .9 .1 .2 .9];
MyGraph.NumIterations = 20;
MyGraph.solve();
disp(d.Value');
\end{lstlisting}



IMPORTANT: Note the use of standard MATLAB array indexing in the d([1:3 5]) command above. Also, note that technically, our nestable factor graph declared above only had one argument (not four), an array of four random variables (as opposed to four separately declared random variables). As a result, the addGraph call requires a single argument---an array of four random 

\fi

\ifjava
\begin{lstlisting}
////////////////////////////////////// 
// Create graph for 6 bit code 
////////////////////////////////////// 
Bit [] d = new Bit[6];
for (int i = 0; i < d.length; i++)
	d[i] = new Bit();

FactorGraph myGraph = new FactorGraph(d); 
myGraph.addFactor(xorGraph,d[0],d[1],d[2],d[4]); 
myGraph.addFactor(xorGraph,d[0],d[1],d[3],d[5]);
////////////////////////////////////// 
// Set input and Solve 
////////////////////////////////////// 
double [] inputs = new double [] {.75, .6, .9, .1, .2, .9};
for (int i = 0; i < inputs.length; i++)
	d[i].setInput(inputs[i]);
myGraph.solve();
for (int i = 0; i < d.length; i++)
	System.out.println(d[i].getValue());
\end{lstlisting}
\fi

\ifmatlab

\subsection{An LDPC Code}
For our last example, we will show how to use a library function to very quickly code up a complex factor graph.

We present the code that defines a Factor Graph to decode an LDPC (low-density parity check) code (in the file /demo/11\_SingleCodewordLDPC/createLDPC.m):

\begin{lstlisting}
function [ldpc,x] = createLdpc()
    A = load('matrixout.txt');
    blockLength = size(A,2);
    numCheckEquations = size(A,1);
    ldpc = FactorGraph();   
        x = Bit(blockLength,1);     
        for i = 1:numCheckEquations
            varIndices = find(A(i,:));
            gd = getNBitXorDef(length(varIndices));
            ldpc.addGraph(gd,x(varIndices));
        end       
     ldpc.NumIterations = 100;
end
\end{lstlisting}


At this point, all the code above should be clear, except for the getNBitXorDef call. getNBitXorDef(N) returns a nestable factor graph with an array of N bits as connected variables, representing an N-bit XOR factor between those N bits. It implements the most efficient (in some sense) implementation of an N-bit XOR using a tree of 3-bits XORs.

The check matrix connectivity is loaded from a file into matrix A. The for loop iterates all check equations, creates a nestable n-bit XOR Graph and adds that graph to the LDPC.  The user can set Input on x and the returned variables, solve the ldpc object, and retrieve the results with Value or Beliefs.

\fi

\section{How to Use Dimple}

\subsection{Defining Models}

\subsubsection{Overview of Graph Structures}

While Dimple supports a variety of graphical model forms, including factor graphs, Bayesian networks, or Markov networks, the Dimple programming interface is oriented toward the language of factor graphs.  Factor graphs are normally defined to be bipartite graphs---that is, graphs consisting of two types of nodes: \emph{factor nodes} and \emph{variable nodes}, connected by \emph{edges}.  Factor nodes connect only to variable nodes and vice versa.  A mathematical overview of factor graphs can be found in Appendix~\ref{sec:IntroductionToFactorGraphs}.

The model creation aspect of Dimple primarily involves defining variables and connecting these variables to factors.  While dimple supports basic factor graphs, in which variables and factors are connected directly with no hierarchy, Dimple also supports more complex structures such as \emph{nested} factor graphs, and \emph{rolled-up} factor graphs, described below.

\begin{description}
\item[Basic factor graph:] This is a graph in which all variables and factors are connected directly, with no hierarchy.  Basic graphs are inherently finite in extent.
\item[Nested factor graph:] When a portion of a factor graph is used more than once, it may be convenient to describe that portion once, and then place copies of that sub-graph within a larger graph.  Nested graphs can be used as a form of modularity that allows abstracting the details of a sub-graph from the description of the outer graph.  A sub-graph can be thought of as a single factor in the outer graph that happens to be described as a factor graph itself.  In Dimple, graphs may be nested to an arbitrary depth of hierarchy.  Creation of a nested sub-graph within another graph is described in section~\ref{sec:usingSubGraphs}.
\item[Rolled-up factor graph:] In some cases, a structure in a factor graph is repeated a great many times, or an unbounded or data-dependent number of times.  In such cases, Dimple allows creation of rolled-up factor graphs, where only one copy of the repeated section is described in the model.  The result is a factor graph that is implicitly unrolled when inference is performed.  The amount of unrolling depends on a potentially unbounded set of data sources.  Creation of rolled-up factor graphs is described in section~\ref{sec:rolledUpFactorGraphs}.
\end{description}

\subsubsection{Creating a Graph}

To create a new factor graph, call the FactorGraph constructor and assign it to a variable.  For example,

\ifmatlab

\begin{lstlisting}
myGraph = FactorGraph();
\end{lstlisting}

Note that Dimple makes use of MATLAB objects.  In this case, a FactorGraph is such an object, and ``FactorGraph()'' corresponds to the constructor of that object; in this case, called with no arguments (if there are no arguments, the parentheses may be omitted).  The FactorGraph object has a number of properties and methods that can be called using MATLAB's standard notation for object properties and methods.  For example, ``myGraph.Name'' refers to the Name property of the factor graph.

\fi

\ifjava

\begin{lstlisting}
FactorGraph myGraph = new FactorGraph();
\end{lstlisting}

\fi

The created factor graph does not yet contain any variables or factors.  Once a graph is created, then variables and factors can be added to it, as described in the subsequent sections.

Creating a factor graph in this way corresponds to a basic factor graph, or the top-level graph in a nested-graph hierarchy.  To create a factor graph that will ultimately be used as a sub-graph in a nested graph hierarchy, the constructor must include arguments that refer to the ``boundary variables'' that will connect it to the outer graph.  For example,

\ifmatlab

\begin{lstlisting}
mySubGraph = FactorGraph(varA, varB, varC);
\end{lstlisting}

\fi

\ifjava
\begin{lstlisting}
FactorGraph mySubGraph = new FactorGraph(varA, varB, varC);
\end{lstlisting}

\fi


In this case, the variables listed in the constructor, must already have been created.  Creation of variables is described in section~\ref{sec:CreatingVariables}.

Creation of sub-graphs is described in more detail in section~\ref{sec:usingSubGraphs}.


\subsubsection{Creating Variables}
\label{sec:CreatingVariables}

\para{Types of Variables}

Dimple supports several types of variables:

\begin{description}
\item[Discrete:] A Discrete variable is one that has a finite set of possible states.  The domain of a discrete variable is a list of the possible states.  These states may be anything---numbers, strings, arrays, objects, etc.
\item[Bit:] A Bit is a special-case of a discrete variable that has exactly two states: 0 and 1.  Note that when using Bit variables, there are some differences in the API versus using Discrete variables.  These differences are noted in the appropriate sections of this manual.
\item[Real:] A Real variable is a continuous variable defined on the real line, or some subset of the real line.
\item[RealJoint:] A RealJoint variable is a multidimensional continuous variable, defined on $\mathbb{R^{N}}$ or a subset of $\mathbb{R^{N}}$.  Unlike a vector of Real variables, the components of a RealJoint variable cannot be connected independently in a graph.
\item[Complex:] A Complex variable is a special-case of a RealJoint variable with two dimensions. \ifmatlab In some cases, the values of a Complex variable are expressed as MATLAB complex numbers. \fi
\item[FiniteFieldVariable:] A FiniteFieldVariable is a special-case of a discrete variable that represents a finite field with $N=2^{n}$ elements.  These fields are often used, for example, in error correcting codes.  These variables can be used along with certain custom factors that are implemented more efficiently for sum-product belief propagation than the alternative using discrete variables and factors implemented directly.  See section~\ref{sec:finiteFields} for more information on how these variables are used.
\end{description}

A factor graph may mix any combination of these variable types, though there are some limitations in the use of certain variable types by some solvers.  Specifically, some solvers do not currently support Real variables.

A variable of a given type can be created using the corresponding object constructor.  For example,

\ifmatlab

\begin{lstlisting}
myDiscreteVariable = Discrete(1:10);
myBitVariable = Bit();
myRealVariable = Real();
\end{lstlisting}

\fi

\ifjava
\begin{lstlisting}
Discrete myDiscreteVariable = new Discrete(1,2,3,4,5,6,7,8,9,10);
Bit myBitVariable = new Bit();
Real myRealVariable = new Real();
\end{lstlisting}
\fi

In this case, each of the above lines created a single variable of the corresponding type.  In the case of a Discrete variable, a constructor argument is required, which defines the domain of the variable.  In the above example, the domain is the set of integer numbers 1 through 10.

Note that variables may be created prior to creation of the factor graph that they will ultimately become part of\footnote{For nested graphs, at least the boundary variables \emph{must} be created prior to creation of the factor graph.}.  A variable is not yet associated with any factor graph until it is connected to at least one factor, as described in section~\ref{sec:CreatingFactors}.


\para{Specifying Variable Domains}

\subpara{Discrete Variables}

When creating a Discrete variable, the domain of that variable must be specified\footnote{Note that for Bit variables, the domain is implicit and is not specified in the constructor.}.  Once created, the domain of a variable cannot change.  

\ifmatlab

The domain of the discrete variable is always the first argument of the constructor.  The domain is a list in the form of either a MATLAB numerical array, or a MATLAB cell array.  Each entry in the array corresponds to one possible state of the variable.  In the case of a cell array, the entries are arbitrary--they may be, for example, strings, vectors, arrays, or objects.

Some examples:

\begin{lstlisting}
v1 = Discrete(1:10);
v2 = Discrete([1.5, 1/3, 27.4327, -13.6, sqrt(2), sin(pi/7)]);
v3 = Discrete({'Sun', 'Clouds', 'Rain', 'Snow'});
v4 = Discrete({ [1 2; 3 4], [5 6 7; 8 9 10], 3.7});
\end{lstlisting}

\fi


\ifjava

The domain can be passed as a comma separated list, an object array, or a DiscreteDomain instance.

Some examples:

\begin{lstlisting}
Object [] domain = new Object[10];
for (int i = 1; i <= domain.length; i++)
	domain[i-1] = i;
Discrete var = new Discrete(domain);
Discrete v2 = new Discrete(1.5, 1/3, 27.4327, -13.6, Math.sqrt(2), Math.sin(Math.PI/7));
Discrete v3 = new Discrete("Sun", "Clouds", "Rain", "Snow");
Discrete v4 = new Discrete(new double [][] {
						new double [] {1,2},
						new double [] {3,4}}, 
					new double [][] {
						new double [] {5,6,7},
						new double [] {8,9,10}},
					3.7);
Discrete v5 = new Discrete(DiscreteDomain.range(1,3));
\end{lstlisting}

WARNING:  The following code:

\begin{lstlisting}
Discrete v6 = new Discrete(new int [] {1,2,3});
\end{lstlisting}

Creates a discrete variable with a single domain object which is an array of three integers.

\fi

Even though the domains are discrete, the actual values of the domain states may be anything, including real numbers.  Though, when using real numbers and objects, care must be taken, for example, when making equality comparisons.

The values of a domain need not be defined in-line in the constructor, but may be defined elsewhere in the program.  For example:

\ifmatlab

\begin{lstlisting}
myDomain = 1:10;
...
myVariable = Discrete(myDomain);
\end{lstlisting}

\fi

\ifjava
\begin{lstlisting}
Object [] domain = new Object[10];
...
for (int i = 1; i <= domain.length; i++)
	domain[i-1] = i;
\end{lstlisting}
\fi

Alternatively, the domain can be defined as a DiscreteDomain object, which provides an object wrapper around the domain.  For example:

\ifmatlab
\begin{lstlisting}
myDomain = DiscreteDomain(1:10);
...
myVariable = Discrete(myDomain);
\end{lstlisting}

In this case, the myDomain object has properties that can be queried, such as ``.Elements'', which provides a list of the elements of the domain.

\fi

\ifjava
\begin{lstlisting}
DiscreteDomain myDomain =  DiscreteDomain.range(1,3);
...
Discrete myVariable = new Discrete(myDomain);
\end{lstlisting}


\fi

In this case, the myDomain object has properties that can be queried, such as \ifmatlab ``.Elements'' \fi \ifjava ".getElements()" \fi, which provides a list of the elements of the domain.

To see the domain of a variable, you can query the Domain property.  For a discrete variable, this returns a DiscreteDomain object, which you can query as described above to see the elements of the domain.  For example:

\ifmatlab
\begin{lstlisting}
disp(myVariable.Domain.Elements);
\end{lstlisting}
\fi

\ifjava
\begin{lstlisting}
System.out.println(Arrays.toString(myVariable.getDomain().getElements()));
\end{lstlisting}
\fi

The Domain property could be used, for example, to take one variable and create others that have the same domain.  For example,

\ifmatlab
\begin{lstlisting}
newVariable = Discrete(otherVariable.Domain);
\end{lstlisting}
\fi

\ifjava
\begin{lstlisting}
Discrete newVariable = new Discrete(otherVariable.getDomain());
\end{lstlisting}
\fi


\subpara{Real Variables}

When creating a Real variable, specifying a domain is optional.  If no domain is specified, the domain is assumed to be the entire real line from $-\infty$ to $\infty$.

The domain of a Real variable is an array of two real numbers: the first specifies the lower bound on the variable, and the second specifies the upper bound (the upper bound must be greater than or equal to the lower).  Either or both of the values may be $-\infty$ or $\infty$.

As with discrete variables, the domain may be specified ahead of time, and may be created by defining in this case a RealDomain object, which takes two arguments: the lower and upper bound, respectively.  Some examples:

\ifmatlab
\begin{lstlisting}
r1 = Real();
r2 = Real([ 0 Inf]);
r3 = Real([-pi pi]);
myDomain = RealDomain(-2.6, 12.4);
r4 = Real(myDomain);
\end{lstlisting}

The Domain property of a Real variable returns in this case a RealDomain object, which contains two properties, LB and UB, which correspond to the lower bound and upper bound, respectively.

\fi

\ifjava
\begin{lstlisting}
Real r1 = new Real();
Real r2 = new Real(0.0,Double.POSITIVE_INFINITY);
Real r3 = new Real(-Math.PI,Math.PI);
RealDomain myDomain = RealDomain.create(-2.6,12.4);
Real r4 = new Real(myDomain);
\end{lstlisting}

The Domain property of a Real variable returns in this case a RealDomain object, which contains two properties, getLowerBound() and getUpperBound(), which correspond to the lower bound and upper bound, respectively.
\fi

\ifmatlab

\para{Creating Arrays of Variables}

In many cases, a factor graph contains a set of variables that all have the same type and the same domain.  In this case, it is often convenient to treat these as a single vector or multidimensional array of variables.  An array of variables can be created all at once by specifying the dimensions of the array as the final arguments in the constructor.  For example,

\begin{lstlisting}
v1 = Bit(10,10);
v2 = Discrete(myDomain, 3, 4, 5, 2, 10);
v3 = Real([-pi pi], 20, 1);
\end{lstlisting}

The first example creates a 10x10 matrix, the second a multidimensional array with 5 dimensions, each with the specified length, and the third a column vector.

The dimensions follow the MATLAB convention of row, followed by column, followed by any additional dimensions.  Also like other MATLAB functions, specifying a single dimension of length N implies a square NxN matrix.

Once a variable array has been created, individual variables or sub-arrays may be referenced using standard MATLAB notation for array indexing.  For example:

\begin{lstlisting}
oneVariable = v1(3,2);
subArray = v2(:,:,:,1,8:end);
subVector = v3(5:10);
\end{lstlisting}

\fi

\para{Naming Variables}

Variables can be named, which can be useful in debugging as well as displaying a factor graph visually.  A single variable can be named by setting the Name property:

\ifmatlab
\begin{lstlisting}
myVariable.Name = 'My variable';
\end{lstlisting}
\fi

\ifjava
\begin{lstlisting}
myVariable.setName("My variable");
\end{lstlisting}

\fi

The variable name can be retrieved by referencing the Name property.

\ifmatlab

An array of variables can be named, each with distinct names by using the setNames method\footnote{For large arrays, using this method is significantly faster than naming each variable individually in a loop.}:

\begin{lstlisting}
myVariableArray.setNames('X');
\end{lstlisting}

This results in a unique name assigned to each variable with the specified prefix, followed by a unique vector variable index number.

\fi

All variables in a factor graph must have unique names.  Sometimes it's desirable to give multiple variables the same label for plotting or displaying.  In this case, users can set the Label property.  If the Label property is not set, plotting and displaying uses the name.  If the Label property is set, the Label is used when displaying and plotting.

\ifmatlab
\begin{lstlisting}
myVar.Label = 'myvar';
\end{lstlisting}
\fi

\ifjava
\begin{lstlisting}
myVar.setLabel("myvar");
\end{lstlisting}
\fi

\subsubsection{Creating Factors and Connections to Variables}
\label{sec:CreatingFactors}

\para{Basic Factor Creation}

Creation of a factor graph involves defining the variables of that graph, and then successively creating factors that connect to these variables.

Before creating a factor, the following must already have been created:
\begin{itemize}
\item The factor-graph into which the factor will be placed.
\item All of the variables that the factor will connect to.
\end{itemize}

The most basic way of creating a factor is by calling the ``addFactor'' method on the factor graph object.  For example:

\begin{lstlisting}
myGraph.addFactor(factorSpecification, myVariable1, myVariable2);
\end{lstlisting}

Resulting from this method call, the specified factor is created, the factor is connected to the variables listed in the arguments, and the factor and all connected variables become part of the designated factor graph (if they are not already).

The factor specification is one or more arguments that define the factor.  Dimple supports a variety of ways of specifying a factor, each of which is described in more detail in subsequent sections.  A factor specification may be one of the following:

\ifmatlab

\begin{itemize}
\item A MATLAB function handle
\item A factor-table
\item A built-in factor
\item The name of a built-in factor
\item A sub-graph
\item A built-in overloaded MATLAB operator\footnote{This method of factor creation does not use addFactor, but instead uses a different syntax to create the factor implicitly.}
\item A custom Java FactorFunction object\footnote{The mechanism to create and use custom Java FactorFunction objects is described in Appendix~\ref{sec:userJava}.}
\end{itemize}

\fi

\ifjava

\begin{itemize}
\item A factor-table
\item A built-in factor (or any instance of a class that extends FactorFunctions)
\item The name of a built-in factor
\item A sub-graph
\end{itemize}

\fi

After the factor specification argument(s), the remaining arguments are all treated as variables or constants.   \ifmatlab The variables may be individual variables or arrays of variables, or combinations thereof. \fi

The number and type of variables that can be connected to a factor depend on the type of factor being created.  In some cases, a factor is defined flexibly to accommodate an arbitrary number of connected variables, while in other cases there are restrictions.  In most cases, the order of variables matters.  The definition of a factor generally requires a specific order of variables, where each variable or set of variables may be required to be of a specific type.  In such cases, the arguments of the addFactor method call must appear in the required order.

For example, ``Normal'' is a built-in factor that describes a set of normally distributed variables with variable mean and precision.  In this case, the first argument must be the mean, the second must be the precision, and the remaining one or more variables are the normally distributed variables.  In this case, the factor creation could look like the following:

\ifmatlab

\begin{lstlisting}
fg = FactorGraph();
meanValue = Real();
precisionValue = Real([0 Inf]);
normalSamples = Real(100,100);
fg.addFactor('Normal', meanValue, precisionValue, normalSamples);
\end{lstlisting}


In this example, the factor specification argument is the name of the built-in factor, and the normalSamples argument refers to the entire array of 100x100 normally distributed variables.  The number of these sample variables is of arbitrary length only because of the way the ``Normal'' built-in factor was defined. 

\fi

\ifjava

\begin{lstlisting}
FactorGraph fg = new FactorGraph();
Real meanValue = new Real();
Real precisionValue = new Real(0.0,Double.POSITIVE_INFINITY);
Real [][] normalSamples = new Real[100][100];
for (int i = 0; i < 100; i++)
	for (int j = 0; j < 100; j++)
		normalSamples[i][j] = new Real();
		
Object [] vars = new Object[2 + 100*100];
vars[0] = meanValue;
vars[1] = precisionValue;
int index = 2;
for (int i = 0; i < 100; i++)
	for (int j = 0; j < 100; j++)
	{
		vars[index] = normalSamples[i][j];
		index++;
	}
		
fg.addFactor(new Normal(), vars);
\end{lstlisting}

In this example, the factor specification argument is an instance of the Normal factor function, and the normalSamples argument refers to the entire array of 100x100 normally distributed variables.  The number of these sample variables is of arbitrary length only because of the way the ``Normal'' built-in factor was defined.  

\fi

Note that these variables could have been listed explicitly as separate arguments.  For example, if there were only two such variables, we could have written:

\ifmatlab
\begin{lstlisting}
normalSamples = Real(2,1);
fg.addFactor('Normal', meanValue, precisionValue, normalSamples(1), normalSamples(2));
\end{lstlisting}
\fi

\ifjava
\begin{lstlisting}
Real normalSample1 = new Real();
Real normalSample2 = new Real();
fg.addFactor(new Normal(), meanValue, precisionValue, normalSample1, normalSample2);
\end{lstlisting}
\fi

When creating a factor, it is sometimes convenient to supply a constant value to one or more of the factor's arguments instead of a variable.  This can be done simply by substituting a value that is not a Dimple variable.  In this case, the value must be consistent with the possible values the particular factor is designed to accept.  In the above example, if we wished to have a variable mean value, but define a fixed precision of 0.2, we could have written:

\ifmatlab

\begin{lstlisting}
fg.addFactor('Normal', meanValue, 0.2, normalSamples(1), normalSamples(2));
\end{lstlisting}

\fi

\ifjava
\begin{lstlisting}
fg.addFactor(new Normal(), meanValue, 0.2, normalSample1, normalSample2);
\end{lstlisting}

\fi

Note that in this case, the ``Normal'' built-in factor requires the precision to be positive, so if we had provided a constant value of -0.2, for example, this would have resulted in an error.  The particular requirements of a factor are specific to the definition of that factor.

\ifmatlab

\para{Vectorized Factor Creation}
\label{sec:vectorizedFactorCreation}

In many cases, a factor graph has repeated patterns of variables and factors, where the factors are all of the same type.  In this case, Dimple provides a vectorized factor creation method that allows adding an entire array of factors all at once with a single operation.  Using this approach is typically significantly faster than adding factors one-by-one in a loop.

The most straightforward use of this method is when all of the associated variables are all arrays with the same dimensions as the array of factors.  For example, supposed we have a 100x100 array of variables, A, and a 100x100 array of variables, B.  And we wish to add a set of 100x100 factors, where each factor connects the corresponding A and B variables.  In this case, we would write:

\begin{lstlisting}
A = Bit(100,100);
B = Bit(100,100);
fg.addFactorVectorized(factorSpecification, A, B);
\end{lstlisting}

This need not be limited to arrays with two dimensions, but generalizes to an arbitrary set of dimensions.

The vectorized approach can also be used to connect arrays of variables to individual variables or constants.  For example, continuing the above example:

\begin{lstlisting}
C = Real();
fg.addFactorVectorized(factorSpecification, A, B, C);
fg.addFactorVectorized(factorSpecification, A, B, 0.1);
\end{lstlisting}

In the first addFactorVectorized call, above, each of the 100x100 factors created is connected to the corresponding element of A and B, and all of them connect to the variable C.  In the second case, above, all of the 100x100 factors use the constant 0.1 as the third argument.

Note that unlike the previous use of addFactor, having array arguments resulted in many copies of the factors rather than a single factor connecting to many variables.  In some cases, it is desirable to have some combination of these.  Dimple provides a syntax for determining which dimensions of an array should be vectorized (creating multiple factors) and which should be treated as array inputs to a each factor.

Expanding on the example above, say A were a 100x100 array and B were a 100x100x5 array, and for each 100x100 factors we wished to connect one element of A with each of the 5 elements of B corresponding the entire length of its third dimension.  In this case, we could write:

\begin{lstlisting}
A = Bit(100,100);
B = Bit(100,100,5);
fg.addFactorVectorized(factorSpecification, A, {B, [1 2]});
\end{lstlisting}

In the above, the variable argument B is replaced with a cell-array containing the variable followed by an array listing the dimensions for which the ``vectorized'' connection to the array of factors should be made.  In this case, first two dimensions, of size 100x100 correspond to the 100x100 factors that we wish to vectorize factor creation.  Since dimension 3 was not included in this list, then for each value of row and column dimension, the entire length of B variables in the third dimension are connected to the corresponding factor.

All arguments passed to addFactorVectorized that are arrays of variables (rather than single variables or constants) must have the same number of dimensions over which they are vectorized and those dimensions must be of the same size.  For example, in the above example, we could not have specified vectorizing over B's second and third dimensions, which do not have the same size as A.  However, if B were instead 5x100x100, we could have done so.

If arrays of variables that must be connected are not already of the same size, or have the same order of dimensions, the MATLAB functions repmat, permute, reshape, or squeeze may be applied to variable arrays to reorganize them into the appropriate form.  In the case of repmat, Dimple does not actually make multiple copies of the variables, but instead provides repeated references to the same variables.  This allows, for example, a vector of variables to be connected to a grid of variables, where each element of the vector connects to an entire row of factors in the grid.  Below is an example of such a case:

\begin{lstlisting}
A = Bit(100,100);
B = Bit(100,1);
fg.addFactorVectorized(factorSpecification, A, repmat(B,1,100));
\end{lstlisting}



\para{Using MATLAB Factor Functions}

A factor may be specified by defining a MATLAB function, and passing a handle to that function.  The function must accept values that correspond to the possible states of the connected variables (in the same order as specified in the addFactor call), and return a non-negative weight corresponding to the unnormalized value of the factor.

Using a MATLAB function to specify a factor is valid in Dimple only when all of the connected variables are discrete (either Discrete or Bit).  Real values are allowed in other parts of a graph, as long as they are not directly connected to a factor defined this way.

Starting with a very simple example, suppose we wish to create a factor that corresponds to a two-input logical AND function, where the first argument is the output of the AND function, and the second and third are the inputs.  This is a deterministic factor, where the weight of the factor is 1 (or any arbitrary constant) if the output variable equals the logical AND of the two inputs, and 0 otherwise.

To define this factor function, we create a MATLAB function that is accessible via the MATLAB path.  In this example, we will call the function AndExample, in the file AndExample.m.  We can define this function as follows:

\begin{lstlisting}
function factorValue = AndExample(out, in1, in2)
    factorValue = (in1 && in2) == out;
end
\end{lstlisting}

We then create a factor using this factor function, connecting it to previously defined Bit variables, x, y, and z:

\begin{lstlisting}
myFactorGraph.addFactor(@AndExample, x, y, z);	
\end{lstlisting}

The ``@'' symbol indicates a MATLAB function handle, which is a reference to the previously defined function.

It would have been possible to instead define the function in-line as a MATLAB anonymous function.  Though, this does not allow a function to be reused to create multiple factors.  The following is equivalent to the example above:

\begin{lstlisting}
myFactorGraph.addFactor(@(out, in1, in2) (in1 && in2) == out, x, y, z);	
\end{lstlisting}

In this case, the factor function had a fixed number of arguments.  If instead, we would like the number of input arguments to be variable, we can make use of MATLAB's variable argument lists:

\begin{lstlisting}
function factorValue = AndExampleImproved(out, varargin)
    factorValue = prod(cell2mat(varargin)) == out;
end
\end{lstlisting}

We can use this factor with an arbitrary number of input variables:

\begin{lstlisting}
myFactorGraph.addFactor(@AndExampleImproved, x, y, z, a, b, c, d);	
\end{lstlisting}


MATLAB factor functions need not be so simple.  First of all, their outputs can be any non-negative value.  But they can also take more interesting values than just the integers.  For example, suppose we define a factor of two variables over a domain consisting of strings such that the factor output value is proportional to the number of times the second string is found in the first, relative to the length of the first string.

\begin{lstlisting}
function factorValue = StringMatchExample(string, pattern)
    factorValue = numel(strfind(string, pattern)) / length(string);
end
\end{lstlisting}

\fi

\para{Using Factor Tables}

When Dimple creates a factor using a factor function, for some solvers, behind the scenes it translates that factor function into a table by enumerating all possible states of all connected variables.  While steps are taken to make this efficient, including storing only non-zero values and reusing tables for identical factors, the time it takes to create the factor table can in some cases be very large.  In some situations, a user may have specific knowledge of the structure of a factor that would allow them to create the same table much more efficiently.  To accommodate such cases, Dimple allows factors to be specified using user-created factor tables.

A factor table consists of two parts: a two dimensional array of integers and a single dimensional array of doubles.  Each row of the two dimensional table represents a combination of variable values for which the factor value is non-zero.  Each column represents a variable connected to the factor. The values of this table specify an index into the discrete domain of a variable. Each row of the two dimensional table corresponds to one entry of the array of doubles, where that entry contains the value of the factor corresponding to the corresponding set of variable values.

Once the user has created the table, they can create a factor using this table in one of two ways. The first is to provide the two dimensional array of indices and vector of values directly as the first two arguments of the addFactor call, respectively:

\ifmatlab

\begin{lstlisting}
   fg = FactorGraph();
   b = Bit(2,1);
   indices = [0 0; 1 1];
   values = [2 1];
   fg.addFactor(indices, values, b);
\end{lstlisting}
   
In the following example we first create a factor table object and then create a factor using that table. This has the advantage of using less overall memory if this same factor table will be used in multiple factors.

\begin{lstlisting}
   fg = FactorGraph();
   b = Bit(2,1);   
   indices = [0 0; 1 1];
   values = [1 1];
   myFactorTable = FactorTable(indices, values, b.Domain, b.Domain);	
   fg.addFactor(myFactorTable, b);
\end{lstlisting}
   
\fi

\ifjava
\begin{lstlisting}
FactorGraph fg = new FactorGraph();
Bit a = new Bit();
Bit b = new Bit();
int [][] indices = new int [][] {
	new int [] {0,0},
	new int [] {1,1}
};
double [] weights = new double [] {2.0,1.0};
FactorTable ft = FactorTable.create(indices, weights, a,b);
fg.addFactor(ft,a,b);
\end{lstlisting}
\fi
   


\para{Using Sub-Graphs}
\label{sec:usingSubGraphs}

In a nested graph, a factor at one layer of the graph hierarchy can correspond to an entire sub-graph.  To add a sub-graph as a factor to another graph, first the sub-graph must have already been created.  A sub-graph is created almost the same as any ordinary graph, with the exception of defining a subset of its variables to be ``boundary'' variables.  These indicate how the sub-graph will connect to other variables in the outer graph.

To understand how sub-graph creation works, we first note that when a sub-graph is added to an outer graph, a new copy of the sub-graph is made, with entirely new variables and factors.  The original sub-graph is used only as a template for creating the copies.  This means that the actual variables used in the sub-graph are never directly used in the final nested graph.  Internal variables within the sub-graph are created new when the sub-graph is added. Boundary variables, on the other hand, are connected to variables in the outer graph, which might already exist in that graph.

When a sub-graph is created, its boundary variables must be defined in the graph constructor.  The boundary variables listed in the constructor must be of the identical type and have the identical domain (in the case of discrete variables) as the variables they will later connect when added to the outer graph.  Additionally, the order of variables listed in creation of the sub-graph must match exactly the order of variables listed when adding the sub-graph to an outer graph.

For example, we define a subgraph as follows:

\ifmatlab

\begin{lstlisting}
a = Discrete(1:10);
b = Bit;
x = Bit;
mySubGraph = FactorGraph(a, b);
mySubGraph.addFactor(exampleFactor1, a, b);
mySubGraph.addFactor(exampleFactor2, b, x);
\end{lstlisting}

\fi

\ifjava
\begin{lstlisting}
Discrete a = new Discrete(1,2,3);
Bit b = new Bit();
Bit x = new Bit();
FactorGraph mySubGraph = new FactorGraph(a, b);
mySubGraph.addFactor(exampleFactor1, a, b);
mySubGraph.addFactor(exampleFactor2, b, x);
\end{lstlisting}
\fi

To add this subgraph to an outer graph, we use addFactor \ifmatlab (or addFactorVectorized)\fi, specifying the factor using the subgraph object.

\ifmatlab

\begin{lstlisting}
N = 5;
fg = FactorGraph;
P = Discrete(1:10, N, 1);
Q = Bit(N,1);
for i = 1:N
    fg.addFactor(mySubGraph, P(i), Q(i));
end
\end{lstlisting}

Equivalently, this can be vectorized using:

\begin{lstlisting}
fg.addFactorVectorized(mySubGraph, P, Q);
\end{lstlisting}

\fi

\ifjava
\begin{lstlisting}
int N = 5;
FactorGraph fg = new FactorGraph();
Discrete [] P = new Discrete[N];
Discrete [] Q = new Discrete[N];
for (int i = 0; i < P.length; i++)
{
	P[i] = new Discrete(1,2,3);
	Q[i] = new Bit();
}

for (int i = 0; i < N; i++)
    fg.addFactor(mySubGraph, P[i],Q[i]);

\end{lstlisting}
\fi


\para{Using Built-In Factors}
\label{sec:usingBuiltInFactors}

Dimple supports a set of built-in factors that can be specified when adding a factor to a graph.  The complete list of available built-in factors can be found in section~\ref{sec:builtInFactors}.

\ifmatlab

In many cases, a built-in factor may be specified simply by referring to the name of the factor as a string (which is case sensitive).  As an example,

\begin{lstlisting}
Mean = Real();
Precision = Real();
Values = Real(1,100);
MyGraph.addFactor('Normal', Mean, Precision, Values);
\end{lstlisting}


Built-in factors may be specified by name only if no constructor arguments are needed.  If constructor arguments are needed, then there are two ways to specify the factor.  The preferred way is to create a FactorFunction object, which takes the name of the factor followed by the constructor arguments for that factor.  For example:

\begin{lstlisting}
MyGraph.addFactor(FactorFunction('Gamma', 1, 1), X);
\end{lstlisting}

The same FactorFunction can be used more than once, which avoids creating additional copies of the FactorFunction object.  For example:

\begin{lstlisting}
myFactorFunction = FactorFunction('Gamma', 1, 1);
MyGraph.addFactor(myFactorFunction, X1);
MyGraph.addFactor(myFactorFunction, X2);
\end{lstlisting}

A short-hand notation may alternatively be used, in which the name of the factor function and its constructor arguments are contained in a cell array.  For example:

\begin{lstlisting}
MyGraph.addFactor({'Gamma', 1, 1}, X);
\end{lstlisting}

\fi

\ifjava

For example:

\begin{lstlisting}
FactorFunction myFactorFunction = new Gamma(1,1);
MyGraph.addFactor(myFactorFunction, X1);
MyGraph.addFactor(myFactorFunction, X2);
\end{lstlisting}


\fi

\ifmatlab

\para{Implicit Factor Creation Using Overloaded Operators and Functions}
\label{sec:ImplicitFactorCreation}

Dimple supports a set of built-in factors that can be added implicitly using overloaded MATLAB operators or functions.  For example,

\begin{lstlisting}
fg = FactorGraph();
a = Discrete(1:4);
b = Discrete(1:10);
c = a + b;
\end{lstlisting}

The last line of this example will automatically create a new variable, c, and a 'Sum' factor connected to variables c, a, and b.  The domain of c will be defined appropriately given the domains of the input variables.  In this example, the domain of c would automatically be set to the range [2:14].

These operations can be compounded in a single line of code, and variables of different data types as well as constants can be intermingled (as long as the type is supported by the specific operator).  In this case, intermediate anonymous variables will be created in the graph associated with intermediate results of the operation.  For example,

\begin{lstlisting}
z = (a + b) * c^d - sqrt(-e);
\end{lstlisting}

Like using the addFactor method, adding factors implicitly can include constants.  Specifically, for binary operators, one of the inputs may be a constant instead of a variable.  For example:

\begin{lstlisting}
x = a^2;
y = (a + b + 2) * 3;
\end{lstlisting}

Since adding a factor implicitly does not specifically refer to the factor graph, the graph to which these factors are added is also implicit.  In particular, these implicitly defined factors are added to the last factor graph that was created.  So, in the first example above, the factor would be added to fg, regardless of whether other factor graphs had previously been created.

Adding built-in factors implicitly using overloaded MATLAB operators or functions can also be vectorized, with some limitations.  Specifically, if each of the input variables are vectors of the same dimension, then the result will be to create a vector of output variables of the same dimension, along with a vector of factors relating the inputs and outputs.

In some cases, to be consistent with MATLAB notation, there is a distinction made between the vectorized and non-vectorized operator.  Specifically, Dimple uses MATLAB's notation for pointwise product and power operators to indicate a vectorized operation.  For example, if variables a through e are vectors of variables of identical size, then the following would create a variable vector z, and a series of factors relating these variables.

\begin{lstlisting}
z = (a .* b) + c.^d - sqrt(-e);
\end{lstlisting}

For binary operators, one of the inputs may be a scalar variable or a scalar constant instead of a variable vector.  For a scalar variable, the result is that scalar variable connecting to each instance of the factors that are created.  For a constant, each instance of the factor uses the same constant for that input (vectors of distinct constants are not currently supported).  As an example:

\begin{lstlisting}
a = Real();
b = Discrete(domain, 1, 10);
z = a + b;					
\end{lstlisting}

In cases where the user wishes to use operator overloading with nested graphs, it may be necessary to use the FactorGraph's addBoundaryVariables method.  For instance, if they create a nested graph such that $y = a+b$ and the user requires y to be a boundary variable, the addBoundaryVariables method must be used.  For example:

\begin{lstlisting}
ng = FactorGraph();
a = Bit();
b = Bit();
y = a+b;
ng.addBoundaryVariables(y,a,b);

fg = FactorGraph();
a2 = Bit();
b2 = Bit();
y2 = Discrete([0 1 2]);
fg.addFactor(ng,y2,a2,b2);
\end{lstlisting}

The specific set of operators supported is given in section~\ref{sec:overloaded}.

\fi


\para{Naming Factors}

Just like variables, factors can be named, which can be useful in debugging as well as displaying a factor graph visually.  To name a factor, the factor object must be accessible via a variable.  When using addFactor, the result is the factor object, which can be assigned to a variable.  A single factor can be named by setting the Name property of the factor object.

\ifmatlab

\begin{lstlisting}
myFactor = fg.addFactor(exampleFactor, a, b, c);
myFactor.Name = 'My factor';
\end{lstlisting}

\fi

\ifjava
\begin{lstlisting}
Factor myFactor = fg.addFactor(exampleFactor, a, b, c);
myFactor.setName("My factor");
\end{lstlisting}

\fi

And the factor name can be retrieved by referencing the \ifmatlab Name property. \fi \ifjava getName() method. \fi

\ifmatlab

An array of factors can be named, each with distinct names by using the setNames method:

\begin{lstlisting}
myFactorArray = addFactorVectorized(exampleFactor, a, b, c);
myFactorArray.setNames('F');
\end{lstlisting}

This results in a unique name assigned to each factor with the specified prefix, followed by a unique vector variable index number.

\fi

Just like Variables, Factors also support the notion of a Label, which can be used to allow multiple factors to share the same label.

\subsubsection{Modifying an Existing Graph}

\para{Removing a Factor}

It is possible to remove a Factor from an existing FactorGraph:

\ifmatlab

\begin{lstlisting}
fg = FactorGraph();
   b = Bit(3,1);
   fg.addFactor(@xorDelta,b(1:2));
   f = fg.addFactor(@xorDelta,b(2:3));
   b.Input = [.8 .8 .6];
   fg.NumIterations = 2;
   fg.solve();
   assertElementsAlmostEqual([.96 .96 .96]',b.Belief);
   fg.removeFactor(f);
   fg.solve();
   p1 = .8*.8;
   p0 = .2*.2;
   total = p1+p0;
   p1 = p1/total;
   p0 = p0/total;
   assertElementsAlmostEqual([p1 p1 .6]',b.Belief);
\end{lstlisting}

\fi

\ifjava

\begin{lstlisting}
FactorGraph fg = new FactorGraph();
 Bit [] b = new Bit[3];
 for (int i = 0; i < b.length; i++)
	 b[i] = new Bit();
 Equals equals = new Equals();
 fg.addFactor(equals,b[0],b[1]);
 Factor f = fg.addFactor(equals,b[1],b[2]);
 double [] inputs = new double [] {0.8,0.8,0.6};
 for (int i = 0; i < inputs.length ; i++)
	 b[i].setInput(inputs[i]);
 fg.solve();
 for (int i = 0; i < b.length; i++)
	 System.out.println(b[i].getP1());
 fg.remove(f);
 fg.solve();
 for (int i = 0; i < b.length; i++)
	 System.out.println(b[i].getP1());
\end{lstlisting}
\fi

\para{Splitting Variables}

It can be useful to make a copy of a variable and relate it to the old variable with an equals factor. The following code shows how to do this.

\ifmatlab

\begin{lstlisting}
   a = Bit();
   a.Name = 'a';
   b = Bit();
   b.Name = 'b';
  
   fg = FactorGraph();
  
   f = fg.addFactor(@(x,y) x~=y,a,b);
   f.Name = 'unequal';
   
   b2 = fg.split(b);
   b2.Name = 'b2';
   a2 = fg.split(a,f);
   a2.Name = 'a2';
   
   fg.plot(1);
\end{lstlisting}

We've added code to name all the variables and factors so that the following plot is informative.

\fi

\ifjava
\begin{lstlisting}

Bit a = new Bit();
Bit b = new Bit();
 
FactorGraph fg = new FactorGraph();
NotEquals ne = new NotEquals();
Factor f = fg.addFactor(ne,a,b);
		   
Variable b2 = fg.split(b);
Variable a2 = fg.split(a,f);
\end{lstlisting}

\fi

\includegraphics[width=.75\textwidth]{images/SplitGraph.png}
 
Note that the split method takes a list of factors as the second through nth argument. This is the list of factors that will be moved from the original variable to the copied variable. All unspecified factors will remain connected to the initial variable.


\para{Joining Variables}

It is possible to join variables in an existing graph, which will create a new joint variable and modify all factors connected to the original variables to reconnect to the new joint variable. This can be useful in eliminating loops in a graph. The following code creates a loopy graph and then uses join to remove the loop.

\ifmatlab

\begin{lstlisting}
a = Bit(); 
a.Name = 'a'; 
b = Bit();
b.Name  = 'b';
c = Bit();
c.Name = 'c';
d = Bit();
d.Name = 'd';

fg = FactorGraph();
f1 = fg.addFactor(@xorDelta,a,b,c);
f1.Name = 'xor';
f2 = fg.addFactor(@(x,y,z) (x|y)==z ,a,b,d);
f2.Name = 'or';
 
newvar = fg.join(a,b);
newvar.Name = 'a,b';

fg.plot(1);
\end{lstlisting}

\fi

\ifjava
\begin{lstlisting}
Bit a = new Bit(); 
Bit b = new Bit();
Bit c = new Bit();
Bit d = new Bit();

FactorGraph fg = new FactorGraph();
ThreeBitXor xor = new ThreeBitXor();
XorYequalsZ xoryequalsz = new XorYequalsZ();
Factor f1 = fg.addFactor(xor,a,b,c);
Factor f2 = fg.addFactor(xoryequalsz ,a,b,d);
Variable newvar = fg.join(a,b);
\end{lstlisting}
\fi

The following is the loopy graph:

\includegraphics{images/LoopyGraph.png}
  
And after joining the variables we have:

\includegraphics{images/NonLoopyGraph.png}


\para{Joining Factors}

It is possible to remove loops by joining factors as well as by joining variables. (\textit{NOTE: an easier way to eliminate loops is to use the Junction Tree solver (see \ref{sec:JunctionTreeSolverAPI}), which will produce a transformed version of the graph without altering the original graph.})

\ifmatlab
\begin{lstlisting}
   b = Bit(4,1);
   for i = 1:4
       b(i).Name = sprintf('b%d',i);
   end
   fg = FactorGraph();
   f1 = fg.addFactor(@xorDelta,b(1:3));
   f2 = fg.addFactor(@xorDelta,b(2:4));
   
   f3 = fg.join(f1,f2);
   f3.Name = 'twoxors';
   
   b.Input = input;
   fg.solve();
   actualBelief = b.Belief;
   
   fg.plot(1);
\end{lstlisting}
\fi

\ifjava
\begin{lstlisting}
Bit [] b = new Bit[4];
for (int i = 0; i < b.length; i++)
	   b[i] = new Bit();
		   
FactorGraph fg = new FactorGraph();
ThreeBitXor xor = new ThreeBitXor();
Factor f1 = fg.addFactor(xor, b[0],b[1],b[2]);
Factor f2 = fg.addFactor(xor, b[1],b[2],b[3]);
		   
Factor f3 = fg.join(f1,f2);
		   
\end{lstlisting}
\fi

The following plot shows the graph with the loops:  
 
\includegraphics{images/LoopyGraph2.png}

And the following plot shows the graph after the factor is joined: 
 
\includegraphics{images/NonLoopyGraph2.png}

To join factors, Dimple does the following:

\begin{itemize}
\item Find the variables in common between two factors.
\item Take the cartesian product of the tables but discard rows where the common variable indices differ.
\item Consolidate the columns with common variables.
\item Multiply the values for each row.
\end{itemize}


\para{Changing Factor Tables}

For factors connected only to discrete variables, the factors are stored in the form of a factor table (regardless of how the factor was originally specified).  It is possible to modify some or all of the entries of that factor table in an existing factor graph.

\ifmatlab

Modifying a single entry or set of entries in a factor table can be done by getting the FactorTable property and calling the set method.  For example, the following changes a single entry in a factor.  In this example, the factor has two edges, the first with domain containing strings, the second with a domain containing numbers.  The arguments of the set method prior to the last one specify the domain settings for which the factor value is to be changed.  The final argument is the new factor value for the particular domain setting, which will replace the previous value.

\begin{lstlisting}
factor.FactorTable.set('red', 1, 0.4);
\end{lstlisting}

To change multiple elements at once, the sets of arguments may be included in a comma-separated list of cell arrays:

\begin{lstlisting}
factor.FactorTable.set({'red', 1, 0.4}, {'blue', 1, 0.25}, {'blue', 2, 0});
\end{lstlisting}

More detail on the use of the set method can be found in section~\ref{sec:FactorTable.set}.

Alternatively, the entire factor table can be replaced with another one.  For this, the change method would be used instead:

\begin{lstlisting}
indexList = [0 0 0; 0 1 1; 1 1 0; 1 0 1];
weightList = [0.2 0.15 0.4 0.9];
factor.FactorTable.change(indexList, weightList);
\end{lstlisting}

\fi

\ifjava
\begin{lstlisting}
int [][] indexList = new int [][] {
	new int[] {0, 0, 0},
	new int[][ {0, 1, 1},
	new int[]{ 1, 1, 0},
	new int[]{1, 0, 1}};
double [] weightList = new double []{0.2, 0.15, 0.4, 0.9};
factor.getFactorTable().change(indexList, weightList);
\end{lstlisting}
\fi

Here the indexList is is an array where each row represents a set of zero-based indices into the list of domain elements for each successive domain in the set of domains associated with the variables connected to this factor.  The weightList is a one-dimensional array of real-valued entries in the factor table, where each entry corresponds to the indices given by the corresponding row of the indexList.

IMPORTANT: Identical factor tables are automatically shared between factors.  If changing the factor table for a factor and if that factor is shared by other factors, then this changes it globally for all such factors.


\subsubsection{Plotting a Graph}

\ifmatlab

\label{sec:PlottingAGraph}

When debugging Factor Graphs, it is sometimes useful to be able to plot a Factor Graph.  The FactorGraph class provides a plot method that can be used to visualize a Factor Graph.

The following code describes how to use the plot function in various ways.

\begin{lstlisting}
%First we build a Factor Graph to use for plotting examples
fg = FactorGraph();
b = Bit(6,1);
for i = 1:6
    b(i).Name = sprintf('b%d',i);
end
 
%We use Label rather than Name for the factors so that we can assign
%them the same Label.  Name must be a unique identifier,
%Label is just used for printing/plotting.
f1 = fg.addFactor(@xorDelta,b(1:4));
f1.Label = 'f';
f2 = fg.addFactor(@xorDelta,b(4:6));
f2.Label = 'f';
 
 
pause_time = 1;
 
%Calling plot with no arguments shows no labels.  It draws variables as
%circles and factors as squares.
fg.plot();
\end{lstlisting}

This will result in the following graph:


\includegraphics{images/plot1.png}

Note that factors are displayed as squares and variables as circles.

\begin{lstlisting}
pause(pause_time);
 
%The following is equivalent to the previous plot command.  We are simply
%explicitly turning off labels.
fg.plot('labels',false);
\end{lstlisting}

Results in the same plot as above.

\begin{lstlisting}
pause(pause_time);
 
%Now we turn on the labels.  Now we see the names we assigned to the
%various nodes and variables of the Factor Graph.
fg.plot('labels',true);
\end{lstlisting}

Results in the following:
 

 
\includegraphics{images/plot2.png}

If the user has specified a Label, those will be displayed, otherwise the object's Name's will be displayed.


\begin{lstlisting}
pause(pause_time)
 
%We can specify a subset of nodes to plot
fg.plot('labels',1,'nodes',{b(1:2),f1});
\end{lstlisting}

Results in the following:



\includegraphics{images/plot3.png}

Only the specified nodes and their connectivity were included.

\begin{lstlisting}
pause(pause_time)
 
%By can set a global color for all the nodes in the graph.
fg.plot('labels',1,'color','g');
\end{lstlisting}

Setting a global color:



\includegraphics{images/plot4.png}

\begin{lstlisting}
pause(pause_time)
 
%We can specify a color for one node in the graph.
fg.plot('labels',1,'nodes',{b(1:2),f1},'color',b(1),'g');
\end{lstlisting}

Setting a color for a specific node:
 
\includegraphics{images/plot5.png}
 
\begin{lstlisting}
pause(pause_time)
 
%We can specify colors for multiple nodes in the graph.
fg.plot('labels',1,'nodes',{b(1:2),f1},'color',{b(2),f1},{'r','c'});
\end{lstlisting}

Setting colors for multiple nodes:

\includegraphics{images/plot6.png}

\begin{lstlisting}
pause(pause_time)
 
%We can mix setting a global color, colors for a single node mutliple
%times, and colors for multiple nodes.  The global color is used in all
%cases where a color has not explicitly been set for a node.
fg.plot('labels',1,'color',b(1),'g','color',b(2),'r','color',{b(3),b(4)},{'y','c'},'color','b');
\end{lstlisting}

Mixing and matching the various coloring options:

\includegraphics{images/plot7.png}

\begin{lstlisting}
pause(pause_time)
 
%We can also specify a root node on which we perform a depth first search
%up to a specific depth and then only print nodes up to that depth.
for depth = 0:5
 
    %Furthermore, we color the root node green so we know which is the root
    %node.
    fg.plot('labels',1,'depth',b(1),depth,'color',b(1),'g','color','b');
    
    pause(pause_time);
end
\end{lstlisting}

Specifying a depth will display a specified root node and all nodes that are N steps away.  The following shows the result of calling plot with `depth' of b(1) and 2:
 

\includegraphics[width=.75\textwidth]{images/plot8.png}

We also colored the root node green to make it clear which was the root node.

\begin{lstlisting}
%The following shows how using the depth feature we might be able to find
%out interesting information.  Here we increase the depth until we visually
%see a loop.
[ldpc,vars] = createLDPC();
v = vars(1);
 
for depth = 0:6
    ldpc.plot('depth',v,depth);
    
    pause(pause_time);
end
\end{lstlisting}

Here, we show how we can use plot to find interesting information about a large graph.  The final plot with a depth of 6 shows a cycle in an LDPC Factor Graph:

\includegraphics{images/plot9.png}

 
\para{Plotting Nested Graphs}

By default, the plotting method ignores hierarchy and plots the flattened graph.  If the user specifies the 'nesting' parameter, however, they can specify how deep to descend into the hierarchy before considering NestedGraphs to be Factors and plotting them as such.

When labels are turned off, nested graphs are displayed as triangles

First let's build a graph with three levels of nesting.

\begin{lstlisting}
b = Bit(2,1);
template1 = FactorGraph(b);
iv = Bit();
template1.addFactor(@xorDelta,b(1),iv);
template1.addFactor(@xorDelta,b(2),iv);
 
b = Bit(2,1);
template2 = FactorGraph(b);
iv = Bit();
template2.addFactor(template1,b(1),iv);
template2.addFactor(template1,b(2),iv);
 
template2.plot();
 
b = Bit(2,1);
fg = FactorGraph(b);
iv = Bit();
fg.addFactor(template2,b(1),iv);
fg.addFactor(template2,b(2),iv);
\end{lstlisting}
 
Here we show the graph plotted with various levels of nesting.

fg.plot('nesting',0);

\includegraphics{images/plot10.png}


Notice the Nested Graphs show up as triangles.

\begin{lstlisting}
fg.plot('nesting',1);
\end{lstlisting}

\includegraphics{images/plot11.png}
 
\begin{lstlisting}
fg.plot('nesting',2);
\end{lstlisting}


\includegraphics{images/plot12.png}
 
Note that once we have reached the bottom, we are actually seeing the Factors plotted.
 
We can retrieve an instance of a nested graph and plot that with nesting set.

\begin{lstlisting}
fg.NestedGraphs{1}.plot('nesting',0);
pause(pause_time);
\end{lstlisting}



\includegraphics{images/plot13.png}
 
 
When plotting graphs, boundary variables show up as stars.

Next we mix depth first search with nesting


\begin{lstlisting}
fg.plot('nesting',0,'depth',iv,0);
\end{lstlisting}

\includegraphics{images/plot14.png}
 

 
\begin{lstlisting}
fg.plot('nesting',0,'depth',iv,1);
\end{lstlisting}

\includegraphics{images/plot15.png}
 

 
\begin{lstlisting}
fg.plot('nesting',0,'depth',iv,2);
\end{lstlisting}


\includegraphics{images/plot16.png}

\fi

\ifjava
Plotting graphs is not currently supported in Java Dimple.  (MATLAB Dimple does support plotting graphs).  
\fi


\subsubsection{Structuring Software for Model Portability}

Because the specification of a model in Dimple is expressed in code, this code can be intermingled with other code that makes use of the model.  However, it is recommended that code for defining a model be clearly separated from code that makes use of the model or performs other functions.  Specifically, we recommend a structure whereby a graph (or subgraph) is encapsulated in a function that creates the graph and returns the graph object, optionally along with some or all of the variables in the graph.  Depending on the application, the graph creation function may take some application-specific arguments.  For example:

\ifmatlab

\begin{lstlisting}
function [fg, vars] = createMyGraph(xDim, yDim)
	fg = FactorGraph();
	v = Bit(xDim, yDim);
	p = Discrete(1:10);
	vars = {v, p};
	fg.addFactorVectorized(@exampleFactor, v(:, 1:end-1), v(:, 2:end), p);
	fg.addFactorVectorized(@exampleFactor, v(1:end-1, :), v(2:end, :), p);
end
\end{lstlisting}

This function can then be used by calling, for example:

\begin{lstlisting}
[fg,vars] = createMyGraph(4,4);
\end{lstlisting}

\fi

\ifjava
\begin{lstlisting}
import com.analog.lyric.dimple.model.Bit;
import com.analog.lyric.dimple.model.FactorGraph;

public class MyGraphCreator {

	public static class Return
	{
		FactorGraph graph;
		Bit a;
		Bit b;
		Bit c;
	}
	
	public static Return createGraph()
	{
		Return r = new Return();
		
		r.graph = new FactorGraph();
		r.a = new Bit();
		r.b = new Bit();
		r.c = new Bit();
		r.graph.addFactor(new ThreeBitXor(),r.a,r.b,r.c);
		return r;
	}
	
	public static void main(String[] args) 
	{
		Return r = createGraph();
		
		r.a.setInput(0.8);
		r.b.setInput(0.9);
		r.graph.solve();
		System.out.println(r.c.getP1());

	}

}
\end{lstlisting}
\fi

While it is possible to use Dimple to retrieve variables from the factor graph object itself, returning variables in the graph creation function allows them to be more easily managed and manipulated.  The set of returned variables should include, at least, the variables that will subsequently be conditioned on input data and the variables that will be queried after performing inference.  But since which variables will be used for these or other purposes may not be known ahead of time, it is often useful to simply return all variables.  \ifmatlab If more than one variable or variable array is to be returned, this could be done either by returning each as a separate return value, or combining them all into a single cell array, as shown in the above example.  \fi   \ifjava If more than one variable is to be returned, this could be done either by returning each as a separate return value, or combining them all into a single class, as shown in the above example.  \fi

In a nested graph, it is often preferable to use a structure like that shown above at each layer of nesting.  In this way, a sub-graph creation function might be reused in more than one different outer graph.

In general, functions that create a Dimple model should not include operations that are related to performing inference.  This includes choosing the solver, setting parameters that affect the solver behavior.  There may in some cases be exceptions.  For example, while conditioning variables on input data would normally not be considered part of the model, in some situations, it might be appropriate to consider this part of the model and to include it in the model creation code.




\subsection{Performing Inference}

Once a model has been created in Dimple, the user may then perform inference on that model.  This typically involves first conditioning on input data, then performing the inference computation, and then querying the model to retrieve information such as beliefs (marginals), maximum \emph{a posteriori} value, or samples from the posterior distribution.

\subsubsection{Choosing a Solver}
\label{sec:Solvers}

To perform the inference computation, Dimple supports a variety of \emph{solver} algorithms that the user can choose from.  For each solver, there are a number of options and configuration parameters that the user may specify to customize the behavior of the solver.

At a high level, Dimple supports two categories of solver:
%
\begin{itemize}
\item Belief propagation (BP)
\item Gibbs sampling
\end{itemize}

The BP solvers further segment into sum-product and min-sum solvers.  The sum-product solvers are primarily for computing the marginals of individual variables in the model, while min-sum can be used to compute the maximum \emph{a posteriori} value.  In both cases, the result may be approximate, depending on the specific model.  In the case of sum-product, the solvers further divide based on how real variables are dealt with.  The basic SumProduct solver doesn't support real variables at all, while the Gaussian and ParticleBP solvers use Gaussian and particle representations, respectively, for messages passed to and from real variables.  In the current version of Dimple, the MinSum solver supports only discrete variables\footnote{This restriction may be lifted in a future version.}.

The Gibbs solver supports all of the primary variable types, and may be used to generate samples from the posterior distribution, or to determine marginals or approximate the maximum \emph{a posteriori} value.

The Solver is a property of a factor graph.  To set the Solver for a given graph, this property is set to the name of the solver.  For example:

\begin{lstlisting}
myGraph.Solver = 'Gibbs';
\end{lstlisting}

The solver name is case insensitive.  The current set of valid solver names are:

\begin{itemize}
\item SumProduct
\item MinSum
\item Gaussian
\item ParticleBP
\item Gibbs
\end{itemize}

More detail on each of these solvers is provided later in this document.

If no solver is specified for a graph, Dimple will use the SumProduct solver by default.


\subsubsection{Conditioning on Input Data}

In many cases, the model created in Dimple represents the prior, before conditioning on the data.  In this case, then assuming inference on the posterior model is desired, then the user must condition the model on the data before performing inference.

There are two primary ways to condition on input data.  In the first approach, the values actually measured are not included in the model, and instead the effect of the data is specified via a likelihood function each variable that is indirectly influenced by the data.  The second approach, the variables that will be measured are included in the model, and the value of each is fixed to the actual measured data value.

\para{Using a Likelihood Function as Input}
\label{sec:LikelihoodInput}

Suppose a variable to be measured, $y$, depends only on another variable, $x$, and the conditional distribution $p(y|x)$ is known.  Then conditioned on the measured value, $y = Y$, then the likelihood of $x$ is given by $L(x) = p(y = Y | x)$.  If our model includes the variable $x$, but does not include $y$, then we can indicate the effect of measuring $y = Y$ by specifying the likelihood function $L(x)$ as the ``input'' to the variable $x$ using the Input property of the variable.

The particular form of the Input property depends on the type of variable.  For a Discrete variable type, the Input property is a vector with length equal to the size of the variable domain.  The values represent the (not necessarily normalized) value of the likelihood function for each element of the domain.  For example,

\begin{lstlisting}
v = Discrete(1:4);
v.Input = [1.2, 0.6, 0, 0.8];
\end{lstlisting}

Notice that values in the Input vector may be greater than one---the Input is assumed to be arbitrarily scaled.  All values, however, must be non-negative.

For a Bit variable, the Input property is specified differently.  In this case, the Input is set to a scalar that represents a normalized version of the likelihood of the value 1.  That is,
%
\[
\frac{L(x=1)}{L(x=0) + L(x=1)}
\]

For example,

\begin{lstlisting}
b = Bit();
b.Input = 0.3;
\end{lstlisting}

In this case, the value must be between 0 and 1, inclusive.

For a Real variable, the form of the Input property depends on the particular solver.  For the Gibbs and ParticleBP solvers, the Input property must be a built-in factor function, as described in section~\ref{sec:usingBuiltInFactors}.  Typically, it would be one of the standard distributions included in the list of available built-in factor functions.  In this case, it must be one in which all the parameters can be fixed to pre-defined constants.  For example:

\begin{lstlisting}
r = Real();
r.Input = FactorFunction('Normal', measuredMean, measurementPrecision);
\end{lstlisting}

For the Gaussian solver, the Input property is a two-element array, where the first element is the mean, and the second element is the standard deviation.  For example:

\begin{lstlisting}
r = Real();
r.Input = [measuredMean, measurementStandardDeviation];
\end{lstlisting}

If the inputs are to be applied to an array of variables, this can generally be done all at once by setting the Input property of the entire array.  For a Discrete variable array, the Input is set to an array with the same dimensions as the variable array (or subarray), but with an extra dimensions corresponding to the Input vector for each variable in the array.  (If a dimension in the array is 1, that dimension is not included.) For example:

\begin{lstlisting}
v = Discrete(1:4, 2, 1);
v.Input = [1.2, 0.6, 0, 0.8 ; 0.4, 0, 1.1, 0.9];
\end{lstlisting}

For an array of Bit variables, the Input is an array with the same dimensions as the Bit array.  For example:

\begin{lstlisting}
b = Bit(2,3);
b.Input = [0.3 0.7 0.1; 0.0 0.8 0.6];
\end{lstlisting}

In the current version of Dimple, Inputs on Real variable arrays must be set one at a time, or all set to a single common value\footnote{This restriction may be removed in a future version.}.



\para{Fixing a Variable to a Known Value}
\label{sec:FixingAVariableValue}

In some cases, the variable that will be measured is included in the model.  In this case, once the value becomes known, the variable may be fixed to that specific value so that the remainder of the model becomes conditioned on that value.  The FixedValue property is used to set a variable to a fixed value.  For a single variable, the FixedValue is set to any value in the domain of that variable.  For example:

\begin{lstlisting}
v = Discrete(1:4);
v.FixedValue = 2;
\end{lstlisting}

\begin{lstlisting}
v = Discrete([1.2, 5.6, 2.7, 6.94]);
v.FixedValue = 5.6;
\end{lstlisting}

\begin{lstlisting}
b = Bit();
b.FixedValue = 0;
\end{lstlisting}

\begin{lstlisting}
r = Real([-pi, pi]);
r.FixedValue = 1.7;
\end{lstlisting}

For Discrete variables, the FixedValue property is currently limited to variables with numeric domains, though the domains need not include only integer values\footnote{This limitation may be removed in a future version.}.

For arrays of variables, the FixedValue property may be set for the entire array by setting it to an array of values of with the same dimensions as the array (or subarray) being set.  For example:

\begin{lstlisting}
b = Bit(2,3);
b.FixedValue = [0 1 0; 1 1 0];
\end{lstlisting}

To see if a FixedValue has been set on a variable, you can use the hasFixedValue method.  For a single variable this method returns a boolean value, and for an array of variables this method returns a boolean array.

Because the Input and FixedValue properties serve similar purposes, setting one of these overrides any previous use of the other.  Setting the Input property removes any fixed value and setting the FixedValue property removes any input\footnote{For implementation reasons, setting the fixed value of a Discrete or Bit variable also sets the Input to a delta function---with the value 0 except in the position corresponding to the fixed value that had been set.}.


\para{Using a Data Source in a Rolled-Up Graph}

In a rolled-up graph, the Input property of a variable can be set using a data source.  Detail on how to do this can be found in section~\ref{sec:rolledUpFactorGraphs}



\subsubsection{Choosing a Schedule}

All of the Dimple solvers operate by successively performing the inference computation on each element in the graph.  In the case of BP solvers, both variable and factor nodes must be updated, and the performance of the inference can depend strongly on the order that these updates occur.  Similarly, for the Gibbs solver, while variables must be updated in an order that maintains the requirements of valid Gibbs sampling, performance may depend on the particular order chosen.

The order of updates in Dimple is called a ``schedule.''  The schedule may either be determined automatically using one of Dimple's built-in ``schedulers,'' or the user may specify a custom schedule.

Each solver has a default scheduler, so if the user does not explicitly choose one, a reasonable choice is made automatically.


\para{Built-in Schedulers}

If no scheduler or custom schedule is specified, a default scheduler will be used.  The default scheduler depends on the selected solver.

Another scheduler may be specified by setting the Scheduler property of a graph:
\begin{lstlisting}
myGraph.Scheduler = 'ExampleScheduler';	
\end{lstlisting}

When setting the scheduler for a graph, the name of the scheduler is case sensitive.

For all of the BP solvers (SumProduct, MinSum, Gaussian, ParticleBP), the following schedulers are available.  More detail on each of these schedulers is provided in section~\ref{sec:FactorGraph.Scheduler}.

\begin{itemize}
\item TreeOrFloodingScheduler
\item TreeOrSequentialScheduler
\item FloodingScheduler
\item SequentialScheduler
\item RandomWithoutReplacementScheduler
\item RandomWithReplacementScheduler
\end{itemize}

In a nested graph, for most of the schedulers listed above (except for the random schedulers), the schedule is applied hierarchically.  In particular, a subgraph is treated as a factor in the nesting level that it appears.  When that subgraph is updated, the schedule for the corresponding subgraph is run in its entirety, updating all factors and variables contained within according to its specified schedule.

It is possible for subgraphs to be designated to use a schedule different from that of its parent graph.  This can be done by specifying either a scheduler or a custom schedule for the subgraph prior to adding it to the parent graph.  For example:
%
\begin{lstlisting}
SubGraph.Scheduler = 'SequentialScheduler';
ParentGraph.addFactor(SubGraph, boundaryVariables);
ParentGraph.Scheduler = 'FloodingScheduler';
\end{lstlisting}

For the TreeOrFloodingScheduler and the TreeOrSequentialScheduler, the choice of schedule is done independently in the outer graph and in each subgraph.  In case that a subgraph is a tree, the tree scheduler will be applied when updating that subgraph even if the parent graph is loopy.  This structure can improve the performance of belief propagation by ensuring that the effect of variables at the boundary of the subgraph fully propagates to all other variables in the subgraph on each iteration.

For the RandomWithoutReplacementScheduler and RandomWithReplacementScheduler, if these are applied to a graph or subgraph, the hierarchy of any lower nesting layers is ignored.  That is, the subgraphs below are essentially flattened prior to schedule creation, and any schedulers or custom schedules specified in lower layers of the hierarchy are ignored.


Because of the differences in operation between the Gibbs solver and the BP based solvers, the Gibbs solver supports a distinct set of schedulers.  For the Gibbs solver, the following schedulers are available.  More detail on each of these schedulers is provided in section~\ref{sec:FactorGraph.Scheduler}.

\begin{itemize}
\item GibbsSequentialScanScheduler
\item GibbsRandomScanScheduler
\end{itemize}

Because of the nature of the Gibbs solver, the nested structure of a graph is ignored in creating the schedule.  That is, the graph hierarchy is essentially flattened prior to schedule creation, and only the scheduler specified on the outermost graph is applied.


\para{Custom Schedules}
\label{sec:CustomSchedules}


Dimple supports user defined custom schedules created in MATLAB.  A custom schedule is specified using the Schedule method.  Specifying a custom schedule overrides any scheduler that the graph would otherwise use.

The following code demonstrates this feature:

\begin{lstlisting}
   eq = @(x,y) x == y;
   fg = FactorGraph();
   a = Bit();
   b = Bit();
   c = Bit();
   eq1 = fg.addFactor(eq,a,b);
   eq2 = fg.addFactor(eq,b,c);
   
   %define schedule
   % update b
   % update eq1->a
   % update eq2->c
   % update a->eq1
   % update c->eq2
   % update eq1->b
   % update eq2->b
   schedule = {
       b,
       {eq1,a},
       {eq2,c},
       {a,eq1},
       {c,eq2},
       {eq1,b},
       {eq2,b}
       };
   
   fg.Schedule = schedule;
   
   %Set priors
   a.Input = .6;
   b.Input = .7;
   c.Input = .8;
   
   %Solve
   fg.NumIterations = 1;
   fg.solve();
\end{lstlisting}

Dimple also supports nesting custom schedules and nesting in general. The following example demonstrates specifying nested graphs in a schedule.

\begin{lstlisting}
   eq = @(x,y) x == y;
   b = Bit(2,1);
   nfg = FactorGraph(b);
   nfg.addFactor(eq,b(1),b(2));
   b = Bit(3,1);
   fg = FactorGraph();
   nf1 = fg.addFactor(nfg,b(1),b(2));
   nf2 = fg.addFactor(nfg,b(2),b(3));
   
   
   fg.Schedule = {b(1),nf1,b(2),nf2,b(3)};
   b(1).Input = .7;
   fg.NumIterations = 1;
   fg.solve();
\end{lstlisting}

And finally we look at nesting a custom schedule:


\begin{lstlisting}
   %Now let's try nesting with a custom schedule on the nested graph.
   
   %Create a graph to nest and give it a funny schedule    
   % nfg: eb(1) - f1 - ib - f2 - eb(2)
   eb = Bit(2,1);
   ib = Bit();
   nfg = FactorGraph(eb);
   f1 = nfg.addFactor(eq,eb(1),ib);
   f2 = nfg.addFactor(eq,ib,eb(2));
   %Set an input and solve
   eb(1).Input = .8;
   
   nfg.NumIterations = 1;
   nfg.solve();
   
   %We expect the output to be equal to the input since the tree
   %scheduler passes the info along.
   assertElementsAlmostEqual(eb(2).Belief,eb(1).Input(1));
   
   %Now we create a schedule that will not propagate the info.
   nfg.Schedule = {ib,{f1,eb(1)},{f2,eb(2)},eb(1),eb(2),f1,f2};
   nfg.solve();
   
   assertElementsAlmostEqual(eb(2).Belief,.5);

   %Nest it and see if the schedule is preserved
   b = Bit(2,1);
   fg = FactorGraph();
   g = fg.addFactor(nfg,b);
   
   fg.Schedule = {b(1),b(2),g};
   
   b(1).Input = .8;
   fg.NumIterations = 1;
   fg.solve();
   assertElementsAlmostEqual(b(2).Belief,.5);
\end{lstlisting}


\subsubsection{Running the Solver}

The once a factor graph has been created and conditioned on any input data, inference may be performed on the graph by calling the solve method:

\begin{lstlisting}
myGraph.solve();
\end{lstlisting}

The solve method performs all necessary computation, leaving the results available to be subsequently accessed.  The behavior of the solve method is determined by the chosen schedule as well as by any solver-specific configuration parameters.

For example, for all of the BP solvers, the number of iterations can be set.  By default, the number of iterations is 1, but for a loopy factor graph, generally multiple iterations should be performed.  To set the number of iterations prior to solving, the NumIterations property of the graph may be used:

\begin{lstlisting}
myGraph.NumIterations = 10;
myGraph.solve();
\end{lstlisting}

For the Gibbs solver, the NumIterations property isn't used, and instead other solver specific may be used instead.  For example, to set the number of Gibbs samples to run before stopping (assuming the solver has already been set to 'Gibbs'):

\begin{lstlisting}
myGraph.Solver.setNumSamples(1000);
myGraph.solve();
\end{lstlisting}

Notice that while NumIterations is a property of a factor graph (even though it isn't used by the Gibbs solver), other solver-specific parameters are generally controlled by calling methods on the solver object.  The solver for a given factor graph is obtained via the Solver property, and solver-specific methods can be called on that.

In general, each solver has a series of custom methods that can be used to customize the behavior of the solver.  A complete list of these can be found in section~\ref{sec:SolversAPI}.

In some cases, it is useful to observe the intermediate behavior of the solver before it has completed.  For the BP solvers, this can be done by using the solver-specific iterate method instead of the solve method.  When called without any arguments, this results in running one iteration.  An optional argument allows specifying the number of iterations to run.  Successive calls to iterate do not reset the state of the solver, allowing it to be called multiple times in succession.  However, before running iterate for the first time, the initialize method must be called in order to reset the state before beginning.  For example, here we run one iteration at a time, displaying the belief of a particular variable after each iteration:

\begin{lstlisting}
myGraph.initialize();
for i=1:numberOfIterations
	myGraph.Solver.iterate();
	disp(someVariable.Belief);
end
\end{lstlisting}

If instead we wanted to run 5 iterations at a time, the iterate call would be replaced with:

\begin{lstlisting}
	myGraph.Solver.iterate(5);
\end{lstlisting}

For the Gibbs solver, a similar method allows running one or a specified number of samples at a time, skipping initialization as well as any burn-in or random restarts.  This is the sample method, which behaves the same as the iterate method:


\subsubsection{Getting the Results of Inference}

Once the solver has been run, the results of inference can be obtained from the elements of the graph.  The kinds of results that may be desired vary with the application, and the kinds of results that are available depend on the particular solver and other factors.

One of the most common types of results are beliefs on individual variables in the graph.  The belief of a variable is an estimate of the marginal distribution of that variable given the graph and any conditioning data.  When available, the belief is accessed via the Belief property of a variable:

\begin{lstlisting}
b = myVariable.Belief;
\end{lstlisting}

The particular form of the Belief property depends on the type of variable, and in some cases on the solver.  For a Discrete variable type, the Belief property is a vector with length equal to the size of the variable domain.  The values represent the normalized value of the approximate marginal probability of each element of the domain.  For a Bit variable, the Belief property is a single number that represents the marginal probability of the value 1.

For Real variables when using the Gaussian solver, the Belief is represented as a mean/standard-deviation pair.  Using the ParticleBP solver, beliefs are available for Real variables, but a different interface is used to obtain the beliefs in a useful form.  This is summarized in section~\ref{sec:ParticleBPRealVariableSpecificMethods}.  Beliefs for Real variables are not currently supported in the Gibbs solver.

Beliefs can be obtained directly from an array of variables using the Belief property of the array.  For a Discrete variable array, the Belief is an array with the same dimensions as the variable array (or subarray), but with an extra dimensions corresponding to the Belief vector for each variable in the array.  (If a dimension in the array is 1, that dimension is not included.)  For an array of Bit variables, the Belief is an array with the same dimensions as the Bit array.

Another useful inference result returned by the Value property of a variable.  This property returns a single value from the domain of the variable that corresponds to the maximum value of the belief:

\begin{lstlisting}
v = myVariable.Value;
\end{lstlisting}

As with the Belief property, this can be used either on individual variables or on a variable array.  Support for this property is currently limited to discrete variables.

When using the Gibbs solver, there are additional inference results that may be useful.  For example, for a given variable, you can request the best sample value that has been seen during inference.

\begin{lstlisting}
bestValue = myVariable.Solver.getBestSample();
\end{lstlisting}

This is defined as the value of that variable associated with the sample over all variables that resulted in the most probably configuration observed given the graph and any conditioning data. Considering the graph as a potential function over the configuration space of all variables, this corresponds to the lowest energy configuration that had been observed.

By default, the Gibbs solver doesn't save all samples, but if so configured for a given variable (or all variables) prior to running the solver, the solver will save all samples, allowing the entire set of samples (post burn-in) to be obtained.

\begin{lstlisting}
myVariable.saveAllSamples();
myGraph.solve();
allSamples = myVariable.Solver.getAllSamples();
\end{lstlisting}

There are a number of other useful results that can be obtained from the Gibbs solver, which are detailed in section~\ref{sec:GibbsSolverAPI}.


It is also possible to retrieve beliefs from factors.  The belief of a factor is defined as the joint belief over all joint states of the variables connected to that factor.  In the current version of Dimple, this works only for factors connected exclusively to discrete variables.  The beliefs can be extracted using one of two properties of a factor:

\begin{lstlisting}
fb = myFactor.Belief;
\end{lstlisting}

or

\begin{lstlisting}
fb = myFactor.FullBelief;
\end{lstlisting}

Using the Belief property results in a compact representation of the belief that leaves out values corresponding to zero values of the factor.  The FullBelief property results in a multidimensional array of beliefs over the range of all possible states of the connected variables.  The more compact representation may be needed where the full representation would result in a data structure too large to be practical.  More information about the Belief and FullBelief properties can be found in sections~\ref{sec:Factor.Belief} and~\ref{sec:Factor.FullBelief}, respectively.

 
\subsubsection{Explicit Scheduling and Retrieving Message Values}

Dimple supports the ability to retrieve and set messages as well as to explicitly update edges, factors and variables.

\begin{lstlisting}
%OK, first we create a simple Factor Graph with a single xor connecting two  
%variables.
fg = FactorGraph(); 
b = Bit(2,1);
f = fg.addFactor(@xorDelta,b);
%We can go ahead and set some inputs
b(1).Input = .8;
b(2).Input = .7;


%we can examine some edges 
disp(f.Ports{1}.InputMsg);
disp(f.Ports{1}.OutputMsg);

%we can even set some edge messages
f.Ports{1}.InputMsg = [.6 .4];

%we can update a node 
b(1).update();
b(2).update();

%or all the variables in a vector.
b.update();

%or a specific edge
b(1).updateEdge(f);
 
%but updating via portNum is quicker
b(1).updateEdge(1);

%of course, if we don't know the portNum, we can get it
portNum = b(1).getPortNum(f);
b(1).updateEdge(portNum);

%We can do the same kind of stuff with factors
f.updateEdge(b(1));
f.updateEdge(f.getPortNum(b(2)));

%Let's look at some messages again
b(1).Ports{1}.InputMsg
b(2).Ports{1}.InputMsg

%and some beliefs
b.Belief
\end{lstlisting}




%
%\subsubsection{Inference on Rolled-Up Graphs}
%
%[??? TO BE COMPLETED ???]
%
%
%                - Data sources
%                - Data sinks
%
%\subsubsection{Accessing Solver-Specific Methods}
%
%[??? TO BE COMPLETED ???]
%
%
%\para{Solver-Specific Methods on Arrays}
%
%[??? TO BE COMPLETED ???]
%
%
%
%
%\subsubsection{Belief Propagation}
%
%\para{Overview}
%
%[??? TO BE COMPLETED ???]
%
%
%\para{Types of Belief Propagation}
%
%[??? TO BE COMPLETED ???]
%
%
%                - �Sum-product
%                    - ...Message representation for Real variables
%                        - ...Gaussian
%                        - ...Particle BP
%                - �Min-sum
%
%\para{Trees Vs. Loopy Graphs}
%
%[??? TO BE COMPLETED ???]
%
%
%\para{Setting the Number of Iterations}
%
%[??? TO BE COMPLETED ???]
%
%
%\para{Schedulers for BP}
%
%[??? TO BE COMPLETED ???]
%
%
%\para{Solving}
%
%[??? TO BE COMPLETED ???]
%
%
%\para{Iterating Step-by-Step}
%
%[??? TO BE COMPLETED ???]
%
%
%\para{Getting the Results of Inference}
%
%[??? TO BE COMPLETED ???]
%
%
%\subpara{Beliefs}
%
%[??? TO BE COMPLETED ???]
%
%
%\subpara{Values}
%
%[??? TO BE COMPLETED ???]
%
%
%\para{Damping}
%
%[??? TO BE COMPLETED ???]
%
%
%\para{K-Best}
%
%[??? TO BE COMPLETED ???]
%
%
%\para{Using Gaussian form of Sum-Product BP}
%
%[??? TO BE COMPLETED ???]
%
%
%\para{Using Particle-BP form of Sum-Product BP}
%
%[??? TO BE COMPLETED ???]
%
%
%\para{Sum-Product with Finite Field Variables}
%
%[??? TO BE COMPLETED ???]
%
%
%
%\subsubsection{Gibbs Sampling}
%
%\para{Overview}
%
%[??? TO BE COMPLETED ???]
%
%
%\para{Specifying Solver Parameters}
%
%[??? TO BE COMPLETED ???]
%
%
%\para{Schedulers for Gibbs Sampling}
%
%[??? TO BE COMPLETED ???]
%
%
%\para{Getting the Results of Inference}
%
%[??? TO BE COMPLETED ???]
%
%
%\subpara{Samples}
%
%[??? TO BE COMPLETED ???]
%
%
%\subpara{Best Joint Sample}
%
%[??? TO BE COMPLETED ???]
%
%
%\subpara{Beliefs}
%
%[??? TO BE COMPLETED ???]
%
%
%\subpara{Values}
%
%[??? TO BE COMPLETED ???]




\subsection{Using Rolled Up Factor Graphs}
\label{sec:rolledUpFactorGraphs}


\subsubsection{Markov Model}

In our first rolled up graph we build a simple Markov model describing an infinite stream of variables.   

\begin{lstlisting}
%%%%%%%%%%%%%%%%%%%%%%%%%%%%%%%%%                                                                                             
%Markov Model                                                                                                                 
%%%%%%%%%%%%%%%%%%%%%%%%%%%%%%%%%                                                                                             

%%%%%%%%%%%%%%%%%%%%%%%%%%%%%%%%%                                                                                             
%Build nested graph                                                                                                           
%%%%%%%%%%%%%%%%%%%%%%%%%%%%%%%%%                                                                                             

%Here we build a graph that connects two variables by an xor 
%equation An xor equation with only two variables is the 
%equivalent of an equals constraint.                                                                                                 

in = Bit();
out = Bit();
ng = FactorGraph(in,out);
ng.addFactor(@xorDelta,in,out);


%%%%%%%%%%%%%%%%%%%%%%%%%%%%%%%%%                                                                                             
%create rolled up graph.                                                                                                      
%%%%%%%%%%%%%%%%%%%%%%%%%%%%%%%%%                                                                                             

%Now we build a FactorGraph that creates an infinite chain of %variables.                                                     
bs = BitStream();
fg = FactorGraph();

%Passing variable streams as arguments to addFactor will result 
%in a rolled up graph.  Passing in a slice of a variable stream 
%specifies a relative offset for where the nested graph should 
%be connected to the variable stream.                       
fs = fg.addFactor(ng,bs,bs.getSlice(2));

%%%%%%%%%%%%%%%%%%%%%%%%%%%%%%%%%                                                                                             
%create data source                                                                                                           
%%%%%%%%%%%%%%%%%%%%%%%%%%%%%%%%%  
data = repmat([.4 .6]',1,10);
ds = DoubleArrayDataSource(data);
dsink = DoubleArrayDataSink();
bs.DataSource = ds;
bs.DataSink = dsink;

%%%%%%%%%%%%%%%%%%%%%%%%%%%%%%%%%                                                                                             
%solve 
%%%%%%%%%%%%%%%%%%%%%%%%%%%%%%%%%                                                                                             
fg.solve();

%%%%%%%%%%%%%%%%%%%%%%%%%%%%%%%%%                                                                                             
%get Beliefs
%%%%%%%%%%%%%%%%%%%%%%%%%%%%%%%%%                                                                                             
while dsink.hasNext()
    dsink.getNext()
end

\end{lstlisting}



Let's talk about some of the aspects of rolled up graphs illustrated in this example in more detail:

\para{Variable Streams and Slices}

Variable Streams represent infinite streams of variables.  When instantiating a VariableStream, users have to specify a domain.  The following are examples of instantiating Discrete VariableStreams:

\begin{itemize}
\item DiscreteStream(DiscreteDomain($\{$0,1,2$\}$));
\item DiscreteStream($\{$0,1,2$\}$); - equivalent to the previous example
\item DiscreteStream($\{$0,1$\}$); - A stream of bits
\item BitStream(); - Shorthand for the previous example.
\end{itemize}

Users can optionally create streams of vectors of variables.  When doing so, users have to specify the dimensions of their variables

\begin{itemize}
\item DiscreteStream([0 1 2],N,1) - For a column vector
\item DiscreteStream([0 1 2],1,N) - For a row vector
\item DiscreteStream([0 1 2],M,N,P) - For a three dimensional tensor
\end{itemize}

When adding a repeated FactorGraph, users need to be able to specify how a nested graph connects to the variable streams.  Slices can be used to indicate where in the stream a nested graph should connect.  Slices are essentially references to a Variable Stream with a specified offset and increment.  Slices have two main methods:

\begin{itemize}
\item hasNext() -- Returns true if the next variable referenced by the slice can be retrieved.
\item getNext() -- Returns the next variable in the  slice if it can be referenced.  Otherwise throws an error.
\end{itemize}

Users typically shouldn't use these methods.  Slices should primarily be used as arguments to addFactor.

First we need to instantiate a variable stream:

\begin{lstlisting}
S = BitStream();
\end{lstlisting}

We can get slices in the following way:

\begin{itemize}

\item S.getSlice(start); -- Start specifies this slices starting point as an index into the variable stream.
\end{itemize}

\para{Buffer Size}

Users can specify a bufferSize for each FactorGraph stream.  A FactorGraphStream is instantiated every time addFactor is called with a nestedGraph and VariableStreams or Slices.  The default bufferSize is 1.  Solving a graph with bufferSize one will result in a forward only algorithm.  The bufferSize indicates how many nested graphs to instantiate for one step.  In our Markov Model example, when buffer size is set to 1 and we plot the graph before solving we see this:

\includegraphics{images/BufferSize1.png}

We see one factor and two instantiated variables.  If we set bufferSize to 5 and plot we get:

\includegraphics{images/BufferSize5.png}


We see five factors and 6 variables.  After the first time we call �advance�, BlastFromThePast factors will be added to the oldest variable.  These factors contain messages from the past. 
There are two ways to set the BufferSize for a FactorGraph stream:

\begin{itemize}
\item Fg.addFactor(ng,bufferSize,sream,slice,etc...); - Specified as second argument to addFactor.
\item Fgs = fg.addFactor(ng,stream,slice,etc...);
fgs.BufferSize = bufferSize; - Set on the FactorGraphStream directly.
\end{itemize}

\para{DataSources}

Users can create data sources and attach them to VariableStreams.  As variables are created, data is retrieved from the DataSources and applied as inputs to the variables.  If a VariableStream is never connected to a DataSource, hasNext() will always return true for that VariableStream.  When a VariableStream is connected to a data source, hasNext() only returns true when there's more data in the DataSource.

DataSources must implement the hasNext and getNext methods.  Methods include:

\begin{itemize}

\item DoubleArrayDataSource(); -- Constructor that creates an empty data source
\item DoubleArrayDataSource(data); -- Constructor that expects a matrix as input.  The data should be formatted such that each column provides input data for a step in the rolled up graph.
\item DoubleArrayDataSource(dimensions); -- Dimensions is a row vector specifying the dimensions of the variable for each step of the repeated graph.
\item DoubleArrayDataSource(dimensions,data); -- Dimensions means the same as above.  The data is added to the double array data source as though add(data) were called.  See the following:
\item Add(data); -- Users can add  more data to the end of the data source.  Data should have the following dimensions: [VarDimensions AizeOfArrayperVariable NumSteps).  So if the user creates an MxN variable and wants to repeat it P times and the variable has domain of length K, the data should have the dimensions: [M N K P]
\end{itemize}

Dimple also has support for MultivariateDataSources.

\begin{itemize}
\item MultivariateDataSource(); -- Creates a data source object and assumes there is a single variable per step.
\item MultivariateDataSource(dimensions) -- Creates a data source objects that can be associated with a variable stream with the given dimensions.
\item add(means,covariance) -- Means should be in the form [VarDimensions NumberMeansPerVar] and Covariance should be of the form [VarDimensions NumberMeansPerVar NumberMeansPerVar] 
\end{itemize}

\para{DataSink}

Users can retrieve their data using data sinks.

\begin{itemize}
\item DoubleArrayDataSink(); -- Create a double array data sink
\item DoubleArrayDataSink(dimensions) -- Creates a double array data sink for a variable with the specified dimensions.
\item MultivariateDataSink() -- Created a multivariate data sink
\item MultivariateDataSink(dimensions) -- Creates a multivariate data sink with the specified dimensions.
\item hasNext() -- Is there more data in the data sink?
\item getNext() -- Retrieve the next chunk of data.  For DoubleArrays, this returns data in the same form as is supplied to data sources.  The MultivariateDataSink object returns MultivariateMsg objects.
\item varStream.DataSink = dataSink; -- Data sinks can be assigned to variable streams.
\end{itemize}

\para{Accessing Variables}
In the absence of data sinks, users need a way to retrieve variables to get beliefs.   The following methods allow the user to do that:

\begin{lstlisting}
Vs = BitStream();
\end{lstlisting}

\begin{itemize}
\item Vs.Size -- Number of variables in the buffer.
\item Vs.get(index) -- Retrieves a variable of the specified index.  This is a 1-based value.
\end{itemize}



\subsubsection{Markov Model with Parameter}


When adding a repeated graph, it is possible to specify some variables as streams and others as individual variables.  We sometimes call these individual variables parameters.  Using this feature is straightforward:

\begin{lstlisting}
ng = FactorGraph(a,b);
ng.addFactor(@xorDelta,a,b);
p = Bit();
s = BitStream();
fg = FactorGraph();
fgs = fg.addFactor(ng,p,s);
fgs.BufferSize = 5;
fg.plot();
\end{lstlisting}

This code results in the following graph:

\includegraphics{images/RolledUpParameter.png}

\subsubsection{Real Variables}

Rolled up graphs work with real variables as well.  Here we create another Markov Model.  We use the Gaussian solver that supports a custom factor called �constmult�.  We create a data source that only has information about the first variable.  The means beliefs are growing by 110\% as we iterate through the stream because the factor provides a constraint that each variable is 110\% of the previous variable.

\begin{lstlisting}
%%%%%%%%%%%%%%%%%%%%%%%%%%%%%%                                                                                                
%Set solver type so we can create reals                                                                                       
%%%%%%%%%%%%%%%%%%%%%%%%%%%%%%                                                                                                
setSolver('Gaussian');

%%%%%%%%%%%%%%%%%%%%%%%%%%%%%%                                                                                                
%Build graph                                                                                                                  
%%%%%%%%%%%%%%%%%%%%%%%%%%%%%%                                                                                                
a = Real();
b = Real();

ng = FactorGraph(a,b);
ng.addFactor(@constmult,b,a,1.1);

fg = FactorGraph();
s = RealStream();

fg.addFactor(ng,s,s.getSlice(2));

%%%%%%%%%%%%%%%%%%%%%%%%%%%%%%                                                                                                
%set data                                                                                                                     
%%%%%%%%%%%%%%%%%%%%%%%%%%%%%%                                                                                                

data = [[1; .1] repmat([0 Inf]',1,10)];
dataSource = DoubleArrayDataSource(data);


s.DataSource = dataSource;
s.DataSink = DoubleArrayDataSink();

%%%%%%%%%%%%%%%%%%%%%%%%%%%%%%                                                                                                
%Solve                                                                                                                        
%%%%%%%%%%%%%%%%%%%%%%%%%%%%%%                                                                                                
fg.solve();


%%%%%%%%%%%%%%%%%%%%%%%%%%%%%%                                                                                                
%get belief                                                                                                                        
%%%%%%%%%%%%%%%%%%%%%%%%%%%%%%                                                                                                
while s.DataSink.hasNext();
    disp(s.DataSink.getNext());
end
\end{lstlisting}

This produces the following output:

\begin{lstlisting}
    1.0000    0.1000

    1.1000    0.1100

    1.2100    0.1210

    1.3310    0.1331

    1.4641    0.1464

    1.6105    0.1611

    1.7716    0.1772

    1.9487    0.1949

    2.1436    0.2144
\end{lstlisting}

\subsubsection{Manually Advancing}

By default, rolled up graphs will advance until there is no data left in a DataSource.  Users may override this behavior by either using FactorGraph.solveOneStep or setting the FactorGraph.NumSteps parameter.  By default, NumSteps is set to Inf.  By setting this to a finite number, N, users can examine the graph at every N steps of the rolled up graph.  This allows the user to pull Beliefs off of any portion of the graph.

\begin{lstlisting}
%%%%%%%%%%%%%%%%%%%%%%%%%%%%%%%%%%%%%%%%%%%%
%simple Markov Model with larger buffer size

%Create the data
N = 10;
data = repmat([.4 .6]',1,N);

%Create a data source
dataSource = DoubleArrayDataSource(data);

%Create a variable stream.
vs = DiscreteStream({0,1});
vs.DataSource = dataSource;

%Create our nested graph
in = Bit();
out = Bit();
ng = FactorGraph(in,out);
ng.addFactor(@(a,b) a==b,in,out);

%Create our main factor graph
fg = FactorGraph();

%Build the repeated graph
bufferSize = 2;
fgs = fg.addFactor(ng, bufferSize, vs,vs.getSlice(2));

%Initialize our messages
fg.initialize();

while 1
    %Solve the current time step
    fg.solveOneStep();
    
    %Get the belief for the first variable
    belief = vs.get(3).Belief;
    disp(belief)

    if fg.hasNext
        fg.advance();
    else
        break;
    end
end
\end{lstlisting}

In this code snippet, the user initializes the graph and calls advance until there is no data left.  At each step, the user retrieves the beliefs from the third instance of the variable in the variable stream.

The user can also progress N steps:

\begin{lstlisting}
%Initialize our messages
fg.initialize();
fg.NumSteps = 2;

while 1
    %Solve the current time step
    fg.continueSolve(); %This method is need to avoid initialization
    
    %Get the belief for the first variable
    belief = vs.get(3).Belief;
    disp(belief)

    if fg.hasNext
        fg.advance();
    else
        break;
    end
end
\end{lstlisting}


\subsection{Using Finite Field Variables}
\label{sec:finiteFields}


\subsubsection{Overview}

Dimple supports a special variable type called a FiniteFieldVariable and a few custom factors for these variables. They represent finite fields with $N=2^{n}$ elements. These fields find frequent use in error correcting codes. Because Dimple can describe any discrete distribution, it is possible to handle finite fields simply by describing their factor tables. However, the native FiniteFieldVariable type is much more efficient. In particular, variable addition and multiplication, which naively require $\mathcal{O}(N^{3})$ operations, are calculated in only $\mathcal{O}(N\log N)$ operations.

\ifmatlab

\subsubsection{Finite Fields Without Optimizations}

As we mentioned previously, a user can construct (non-optimized) finite fields from scratch.

First we create a domain for our variables using the MATLAB gf function (for Galois Field).

\begin{lstlisting}
m = 3; 
numElements = 2^m;
domain = 0:numElements-1;

tmp = gf(domain,m); 
real_domain = cell(length(tmp),1);
for i = 1:length(tmp)
  real_domain{i} = tmp(i); 
end
\end{lstlisting}

Next we create a bunch of variables with that domain.

\begin{lstlisting}
x_slow = Discrete(real_domain);
y_slow = Discrete(real_domain); 
z_slow = Discrete(real_domain);
\end{lstlisting}

Now we create our graph and add the addition constraint.

\begin{lstlisting}
fg_slow = FactorGraph();

addDelta = @(x,y,z) x+y==z;
fg_slow.addFactor(addDelta,x_slow,y_slow,z_slow);
\end{lstlisting}

This code runs in $\mathcal{O}(N^3) $ time since it tries all combinations of x,y, and z.
Next we set some inputs.

\begin{lstlisting}
x_input = rand(size(x_slow.Domain.Elements));
y_input = rand(size(y_slow.Domain.Elements));

x_slow.Input = x_input;
y_slow.Input = y_input;
\end{lstlisting}

Finally we set number of iterations, solve, and look at beliefs.

\begin{lstlisting}
fg_slow.NumIterations = 1;

fg_slow.solve();

z_slow.Belief
\end{lstlisting}

The solver runs in $\mathcal{O}(N^2)$ time since z is determined by x and y, x is determined by z, and y, and y is determined by x and z.

\fi

\subsubsection{Optimized Finite Field Operations}

Rather than building finite field elements from scratch, a user can use a built-in variable type and associated set of function nodes. These native variables are much faster, both for programming and algorithmic reasons. All of these operations are supported with the SumProduct solver.

\para{FiniteFieldVariables}

Dimple supports a FiniteFieldVariable variable type, which takes a primitive polynomial (to be discussed later) and dimensions of the matrix as constructor arguments:

\ifmatlab
\begin{lstlisting}
v = FiniteFieldVariable(prim_poly,3,2);
\end{lstlisting}

This would create a 3x2 matrix of finite field Variables with the given primitive polynomial.

\fi

\ifjava
\begin{lstlisting}
FiniteFieldVariable v = new FiniteFieldVariable(prim_poly);
\end{lstlisting}

This would create a finite field Variable with the given primitive polynomial.

\fi

\para{Addition}

Users can use the following syntax to create an addition factor node with three variables:

\ifmatlab
\begin{lstlisting}
myFactorGraph.addFactor('FiniteFieldAdd',x,y,z);
\end{lstlisting}

using the 'FiniteFieldAdd' built-in factor.
\fi

\ifjava
\begin{lstlisting}
myFactorGraph.addFactor("FiniteFieldAdd",x,y,z);
\end{lstlisting}

The factor function must have the "FiniteFieldAdd" name in order for the SumProduct solver to use the correct custom factor.
\fi

Adding this variable take $\mathcal{O}(1)$ time and solving takes $\mathcal{O}(N\log N)$ time, where N is the size of the finite field domain.

\para{Multiplication}

Similarly, the following syntax can be used to create a factor node with three variables for multiplication:

\ifmatlab
\begin{lstlisting}
myFactorGraph.addFactor('FiniteFieldMult',x,y,z);
\end{lstlisting}
\fi

\ifjava
\begin{lstlisting}
myFactorGraph.addFactor("FiniteFieldMult",x,y,z);
\end{lstlisting}
\fi

Under the hood this will create one of two custom factors, CustomFiniteFieldConstMult or CustomFiniteFieldMult. The former will be created if x or y is a constant and the latter will be created if neither is a constant. This allows Dimple to optimize belief propagation so that it runs in O(N) for multiplication by constants and O(Nlog(N)) in the more general case.

\ifmatlab

\para{NVarFiniteFieldPlus}

Suppose we have the finite field equation $ x1+x2+x3+x4=0 $.

We can not express that using the FiniteFieldAdd factor directly, since it accepts only three arguments. However, we can support larger addition constraints by building a tree of these constraints. We do so using the following function:

\begin{lstlisting}
NumVars = 4;
[graph,vars] = getNVarFiniteFieldPlus(prim_poly,NumVars);
\end{lstlisting}

This function takes a primitive polynomial and the number of variables involved in the constraint and builds up a graph such that  $ x_1+...+x_n = 0 $ .

It returns both the graph and the external variables of the graph. This can be used in one of two ways: setting inputs on the variables and solving the graph directly or using this as a nested sub-graph.

\fi

\para{Projection}

Elements of a finite field with base 2 can be represented as polynomials with binary coefficients. Polynomials with binary coefficients can be represented as strings of bits. For instance, $x^3+x+1$ could be represented in binary as 1011. Furthermore, that number can be represented by the (decimal) integer 11. When using finite fields for decoding, we are often taking bit strings and re-interpreting these as strings of finite field elements. We can use the FiniteFieldProjection built-in factor to relate n bits to a finite field variable with a domain of size $2^n$.

The following code shows how to do that:

\ifmatlab
\begin{lstlisting}
args = cell(n*2,1);
for j = 0:n-1
   args{j*2+1} = j;
   args{j*2+2} = bits(n-j);
end

myFactorGraph.addFactor('FiniteFieldProjection',v,args{:});
\end{lstlisting}
\fi

\ifjava
\begin{lstlisting}
Object [] vars = new Object[n*2];
		
for (int j = 0; j < n; j++)
{
	vars[j*2] = j;
	vars[j*2+1] = new Bit();
}
myFactorGraph.addFactor("FiniteFieldProjection",vars);

\end{lstlisting}
\fi

\subsubsection{Primitive Polynomials}

See Wikipedia for a definition. 

\subsubsection{Algorithmics}


Dimple interprets the domains as integers mapping to bit strings describing the coefficients of polynomials. Internally, the FiniteFieldVariable contains functions to map from this representation to a representation of powers of the primitive polynomial. This operation is known as the discrete log. Similarly Dimple FiniteFieldVariables provide a function to map the powers back to the original representation (i.e., an exponentiation operator).

\begin{itemize}
\item The addition code computes $x+y$ by performing a fast Hadamard transform of the distribution of both $x$ and $y$, pointwise multiplying the transforms, and then performing an inverse fast Hadamard transform.
\item The generic multiplication code computes $xy$ by performing a fast Fourier transform on the distribution of the non-zero elements of the distribution, pointwise multiplying the transforms, performing an inverse fast Fourier transform, and then accounting for the zero elements.
% FIXME
\item The constant multiplication code computes $x$ by converting the distribution of the non-zero values of $x$ to the discrete log domain (which corresponds to reshuffling the array), adding the discrete log of   modulo $N-1$ (cyclically shifting the array), and exponentiating (unshuffling the array back to the original representation).  
\end{itemize}

%\subsection{Parameter Learning}
%
%\subsubsection{Expectation-Maximization on Directed Graphs}
%
%[??? TO BE COMPLETED ???]
%
%\subsubsection{Bayesian Parameter Estimation Using Gibbs Sampling}
%
%[??? TO BE COMPLETED ???]

\subsection{Graph Libraries}

Dimple provides a few graphs that are useful as nested graphs.

\subsubsection{Multiplexer CPDs}
\label{sec:multiplexerCPD}

Suppose you wanted a factor representing a DAG with the following probability distribution:

\[
p(Y=y|a,z_1,z_2,...) \propto \delta(y = z_a)
\]

where all variables are Discrete.

You could code this up in Dimple as follows:

\ifmatlab
\begin{lstlisting}
function weight = myFunc(y,a,z)
    weight = y == z(a);
end

N = 3; %Number of possible sources
M = 2; %Domain of Zs and Y

y = Discrete(1:M);
a = Discrete(1:N);
z = Discrete(1:M,N,1);
 
fg = FactorGraph();
 
fg.addFactor(@myFunc,y,a,z);
\end{lstlisting}
\fi

\ifjava
\begin{lstlisting}
public class Main 
{		
	public static class MyFunc extends FactorFunction
	{
		@Override
		public double eval(Object ... args)
		{
			Object y = args[0];
			int a = (Integer)args[1];
			
			Object z = args[2+a];
			
			return y.equals(z) ? 1.0 : 0.0;
		}
	}
	
	public static void main(String [] args)
	{
		
		Discrete y = new Discrete(1,2);
		Discrete a = new Discrete(0,1,2);
		Discrete z1 = new Discrete(1,2);
		Discrete z2 = new Discrete(1,2);
		Discrete z3 = new Discrete(1,2);
		 
		FactorGraph fg = new FactorGraph();
		 
		fg.addFactor(new MyFunc(),y,a,z1,z2,z3);
	}
}
\end{lstlisting}
\fi

However, to build this FactorTable takes $O(NM^{N+1})$ where N is the number of Zs and M is the domain size of the Zs.  Runtime is almost as bad at $O(NM^N)$.  However, there is an optimization that can result in $O(MN^2)$ runtime and graph building time.  Dimple provides a MultiplexerCPD graph that can be used as a nested graph to achieve this optimization.

\ifmatlab
\begin{lstlisting}
cpd = MultiplexerCPD({1,2},3);
Y = Discrete(1:2);
A = Discrete(1:3);
Z1 = Discrete(1:2);
Z2 = Discrete(1:2);
Z3 = Discrete(1:2);
fg = new FactorGraph();
fg.addFactor(cpd,Y,A,Z1,Z2,Z3);
\end{lstlisting}
\fi


\ifjava
\begin{lstlisting}
MultiplexerCPD cpd = new MultiplexerCPD(new Object [] {1,2},3);
Discrete Y = new Discrete(1,2);
Discrete A = new Discrete(0,1,2);
Discrete Z1 = new Discrete(1,2);
Discrete Z2 = new Discrete(1,2);
Discrete Z3 = new Discrete(1,2);

FactorGraph fg = new FactorGraph();
fg.addFactor(cpd,Y,A,Z1,Z2,Z3);
\end{lstlisting}
\fi

Dimple supports each Z having different domains.  In this case, Y's domain must be the sorted union of all the Z domains

\ifmatlab
\begin{lstlisting}
cpd = MultiplexerCPD({{1,2},{1,2,3},{2,4}});
Y = Discrete(1:4);
A = Discrete(1:3);
Z1 = Discrete(1:2);
Z2 = Discrete(1:3);
Z3 = Discrete([2 4]);
fg = FactorGraph();
fg.addFactor(cpd,Y,A,Z1,Z2,Z3);
\end{lstlisting}
\fi

\ifjava
\begin{lstlisting}
Object [][] domains = new Object [][] {
	new Object [] {1,2},
	new Object [] {1,2,3},
	new Object [] {2,4}
};

MultiplexerCPD cpd = new MultiplexerCPD(domains);
Discrete Y = new Discrete(1,2,3,4);
Discrete A = new Discrete(0,1,2);
Discrete Z1 = new Discrete(1,2);
Discrete Z2 = new Discrete(1,2,3);
Discrete Z3 = new Discrete(2,4);

FactorGraph fg = new FactorGraph();
fg.addFactor(cpd,Y,A,Z1,Z2,Z3);
\end{lstlisting}
\fi

Note that when using the SumProduct solver, a custom implementation of the built-in `Multiplexer' factor function (see section~\ref{sec:builtInFactors}) exists that is even more efficient than using the MultiplexerCPD graph library function.  When using other solvers with discrete variables, the MultiplexerCPD graph library function should be more efficient.  When the Z variables are real rather than discrete, the `Multiplexer' factor function is the only option.


\ifmatlab
\subsubsection{N-Bit Xor Definition}
An N-bit soft xor can be decomposed into a tree of three bit soft xors.  Dimple provides the getNBitXorDef function to generate such a graph.  The following code shows how to use such a graph.

\begin{lstlisting}
fg = FactorGraph();
fg.addFactor(getNBitXorDef(4),Bit(4,1));
\end{lstlisting}

\fi
\section{API Reference}

The following section describes the functions, classes, properties, and methods that comprise the Dimple API for \ifmatlab MATLAB. \fi \ifjava Java. \fi

\subsection{FactorGraph}

The FactorGraph class represents a single factor graph and contains a collection of all factors associated with that factor graph and all variables that are connected by factors to the graph.

\subsubsection{Constructor}

\begin{lstlisting}
FactorGraph([boundaryVariables])
\end{lstlisting}

For a basic factor graph, the constructor is simply the command FactorGraph with no arguments.

For a nested factor graph (one that may be used as a sub-graph within another graph), the constructor must include a list of the boundary variables of the graph.  When used as a sub-graph, the boundary variables are dummy variables with the same specification as the variables in the outer graph that will ultimately connect to the sub-graph.  A graph defined with boundary variables may alternatively be used as a top-level graph, in which case the boundary variables are used directly.

\subsubsection{Properties}

\para{Solver}
\label{sec:FactorGraph.Solver}

Read-write.  Indicates the choice of solver to be used for performing inference on the graph.  The default solver is SumProduct.

When setting the solver, the solver is given by a string representing the name of the solver.  The solver name is case insensitive.

\begin{lstlisting}
fg.Solver = 'SolverName';
\end{lstlisting}

The current set of valid solver names are:

\begin{itemize}
\item SumProduct
\item MinSum
\item Gaussian
\item ParticleBP
\item Gibbs
\end{itemize}

A description of each of these solvers is given in section~\ref{sec:SolversAPI}.

Note that the solver can be modified at any time.  After running the solver on a graph, the solver may be modified and the new solver run using the same graph\footnote{In this case, care must be taken to set any solver-specific parameters to the new values after changing the solver.}.  One exception is that if solver-specific built-in factors (also referred to as ``custom factors'') are used in a graph, it is not possible to switch solvers (a list of solver-specific built-in factors is given at the end of section~\ref{sec:builtInFactors}).


\para{Scheduler}
\label{sec:FactorGraph.Scheduler}

Read-write.  Indicates the scheduler to be used for performing inference on the graph (unless a custom schedule is specified instead).

When setting the scheduler, the scheduler is given by a string representing the name of the scheduler.  The scheduler name is case \emph{sensitive}.

\begin{lstlisting}
fg.Scheduler = 'SchedulerName';
\end{lstlisting}

For all of the BP solvers (SumProduct, MinSum, Gaussian, ParticleBP), the following schedulers are available:

\begin{longtable}{l p{4in}}
\textbf{Name} & \textbf{Description} \\ \hline \hline
%
\textsf{DefaultScheduler} & Same as the TreeOrFloodingScheduler, which is the default if no scheduler or custom schedule is specified. \\ \hline
%
\textsf{TreeOrFloodingScheduler} & The solver will use either a Tree Schedule or a Flooding Schedule depending on whether the factor-graph contains cycles.  In a nested graph, this choice is applied independently in each subgraph.  If the factor-graph is a tree, the scheduler will automatically detect this and use a Tree Schedule.  In this schedule, each node is updated in an order that will result in the correct beliefs being computed after just one iteration.  If the entire graph is a tree, NumIterations should be set to 1, which is its default value.  If the factor-graph is loopy, the solver will instead use a Flooding Schedule (as described below). \\ \hline
%
\textsf{TreeOrSequentialScheduler} & The solver will use either a Tree Schedule (as described above) or a Sequential Schedule (as described below) depending on whether the factor-graph contains cycles.  In a nested graph, this choice is applied independently in each subgraph.  \\ \hline
%
\textsf{FloodingScheduler} & The solver will apply a Flooding Schedule.  For each iteration, all variable nodes are updated, followed by all factor nodes.  Because the graph is bipartite (factor nodes only connect to variable nodes, and vice versa), the order of update within each node type does not affect the result. \\ \hline
%
\textsf{SequentialScheduler} & The solver will apply a Sequential Schedule.  For each factor node in the graph, first, for each variable connected to that factor, the edge connecting the variable to the factor is updated; then the factor node is updated.  The specific order of factors chosen is arbitrary, and depends on the order that factors were added to the graph. \\ \hline
%
\textsf{RandomWithoutReplacementScheduler} & The solver will apply a Sequential Schedule with the order of factors chosen randomly without replacement.  On each subsequent iteration, a new random order is chosen.  Since the factor order is chosen randomly with replacement, on each iteration, each factor will be updated exactly once. \\ \hline
%
\textsf{RandomWithReplacementScheduler} & The solver will apply a Sequential Schedule with the order of factors chosen randomly with replacement.  On each subsequent iteration, a new random order is chosen.  The number of factors updated per iteration is equal to the total number of factors in the graph. However, since the factors are chosen randomly with replacement, not all factors are necessarily updated in a single iteration, and some may be updated more than once. \\ \hline
%
\end{longtable}


In a nested graph, for most of the schedulers listed above (except for the random schedulers), the schedule is applied hierarchically.  In particular, a subgraph is treated as a factor in the nesting level that it appears.  When that subgraph is updated, the schedule for the corresponding subgraph is run in its entirety, updating all factors and variables contained within according to its specified schedule.

It is possible for subgraphs to be designated to use a schedule different from that of its parent graph.  This can be done by specifying either a scheduler or a custom schedule for the subgraph prior to adding it to the parent graph.  For example:
%
\begin{lstlisting}
SubGraph.Scheduler = 'SequentialScheduler';
ParentGraph.addFactor(SubGraph, boundaryVariables);
ParentGraph.Scheduler = 'FloodingScheduler';
\end{lstlisting}

For the TreeOrFloodingScheduler and the TreeOrSequentialScheduler, the choice of schedule is done independently in the outer graph and in each subgraph.  In case that a subgraph is a tree, the tree scheduler will be applied when updating that subgraph even if the parent graph is loopy.  This structure can improve the performance of belief propagation by ensuring that the effect of variables at the boundary of the subgraph fully propagates to all other variables in the subgraph on each iteration.

For the RandomWithoutReplacementScheduler and RandomWithReplacementScheduler, if these are applied to a graph or subgraph, the hierarchy of any lower nesting layers is ignored.  That is, the subgraphs below are essentially flattened prior to schedule creation, and any schedulers or custom schedules specified in lower layers of the hierarchy are ignored.


Because of the differences in operation between the Gibbs solver and the BP based solvers, the Gibbs solver supports a distinct set of schedulers.  For the Gibbs solver, the following schedulers are available:

\begin{tabular}{l p{4in}}
\textbf{Name} & \textbf{Description} \\ \hline \hline
%
\textsf{GibbsDefaultScheduler} & Same as the GibbsSequentialScanScheduler, which is the default when using the Gibbs solver. \\ \hline
%
\textsf{GibbsSequentialScanScheduler} & The solver will apply a Sequential Scan Schedule.  For each scan, each variable is resampled in a fixed order.  The specific order of variables chosen is arbitrary, and depends on the order that variables were added to the graph. \\ \hline
%
\textsf{GibbsRandomScanScheduler} & The solver will apply a Random Scan Schedule.  Each successive variable to be resampled is chosen randomly with replacement.  The number of variables resampled per scan is equal to the total number of variables in the graph, but not all variables are necessarily resampled in a given scan, and some may be resampled more than once. \\ \hline
%
\end{tabular}

Because of the nature of the Gibbs solver, the nested structure of a graph is ignored in creating the schedule.  That is, the graph hierarchy is essentially flattened prior to schedule creation, and only the scheduler specified on the outermost graph is applied.



\para{Schedule}

Read-write.  Specifies a custom schedule to be used for performing inference.

\para{NumIterations}

Read-write.  The NumIterations property sets the number of iterations BP will to run when using the solve method.  This only applies to solvers that use BP, which are the SumProduct, MinSum, Gaussian, and ParticleBP solvers.

The default value is 1.  For a factor graph with a tree-structure, when using the default scheduler, one iteration is appropriate.  Otherwise, it would normally be appropriate to set the number of iterations to a larger value.

\para{NumSteps}

\para{Name}

Read-write.  When read, retrieves the current name of the factor graph.  When set, modifies the name of the factor graph to the corresponding value.  The value set must be a string.

\begin{lstlisting}
fg.Name = 'string';
\end{lstlisting}


%\para{QualifiedName}
%
%\para{ExplicitName}
%
%\para{Label}
%
%\para{QualifiedLabel}
%
%\para{UUID}
%
%\para{Nodes}
%
%\para{NodesFlat}
%
%\para{NodesTop}
%
%\para{Variables}
%
%\para{VariablesFlat}
%
%\para{VariablesTop}
%
%\para{Factors}
%
%\para{FactorsFlat}
%
%\para{FactorsTop}
%
%\para{NonGraphFactors}
%
%\para{NonGraphFactorsFlat}
%
%\para{NonGraphFactorsTop}
%
%\para{NestedGraphs}
%
%\para{FactorGraphStreams}
%
%\para{AdjacencyMatrix}

\para{Score}

\para{BetheFreeEnergy}

\para{BetheEntropy}

\para{InternalEnergy}


\subsubsection{Methods}

\para{initialize}

\begin{lstlisting}
MyGraph.initialize();
\end{lstlisting}

The initialize method resets the state of the factor graph and its associated solver.  When performing inference incrementally, for example using the iterate method, the initialize method must be called before the first iterate call.  When using the solve method to perform inference, there is no need to call initialize first.  The initialize method takes not arguments.

\para{solve}

\begin{lstlisting}
MyGraph.solve();
\end{lstlisting}

The solve method runs the solver on the factor graph for the specified duration.  Calling solve initializes the graph prior to solving.


\para{addFactor}

The addFactor (and related) methods are used to add a factor to a factor-graph, connecting that factor to a specified set of variables.  There are several ways of specifying the particular factor:

\begin{itemize}
\item A MATLAB function handle.
\item A sub-graph.
\item A factor-table.
\item A built-in factor.
\item A built-in overloaded MATLAB operator.
\item A custom Java FactorFunction object.
\end{itemize}

Each of these methods is summarized in the subsequent sections.

\subpara{MATLAB Function Handle}

A factor may be specified by defining a MATLAB function, and passing a handle to that function.  The function must accept values that correspond to the state of the connected variables (in the same order as specified in the addFactor call), and return a non-negative weight corresponding to the unnormalized value of the factor.

In this case, the �addFactor� method takes the following arguments:

\begin{itemize}
\item A function handle. 
\item A variable length list of Variables or matrices.
\end{itemize}

For example,

\begin{lstlisting}
MyGraph.addFactor(@XorDeltaTutorial,x,y,z);	
\end{lstlisting}

The ``@'' symbol indicates a function handle, where the function ``XorDeltaTutorial'' is assumed to exist somewhere in the current MATLAB path.

Using a MATLAB function to specify a factor is valid in Dimple only when all of the connected variables are discrete (either Discrete or Bit).  Real values are allowed in other parts of a graph, as long as they are not directly connected to this factor.

Some of the variables in the call may be replaced by constants.  In this case, no variable is created, but instead the specified constant value is used for the value of that input to the factor function.  The following shows an example of mixing variables and constants as arguments to the addFactor method.

\begin{lstlisting}
b = Bit(2,1);
MyGraph.addFactor(@XorDeltaTutorial,b(1),1, b(2));	
\end{lstlisting}

The constants can be any n-dimensional real or complex matrix.  

Adding a Factor in this way involves creating an array known as a factor table, which contains all assignments of the variables resulting in a positive value of the function, along with a value vector, containing all the corresponding values of the function. This table is then passed to the solver. Creating that table takes time (and typically space) proportional to the product of the size of the domains of all variables connected to that factor, thus requiring an exponential time in the degree of the factor.  The addFactor method attempts to save memory and time by re-using existing factors when addFactor is called with the same factor function.  The caching mechanism uses the name of the function (using MATLAB�s func2str) as well as the domain values for all of the arguments to addFactor when deciding if a FactorTable can be re-used.  If this behavior is undesirable, the addFactorNoCache method can be used instead of addFactor (see below).

\subpara{Nested Graphs}

In a nested graph, a sub-graph can be added to its parent graph using the addFactor method.  In this case, the first argument of the addFactor call is the sub-graph.  For example:

\begin{lstlisting}
MyGraph.addFactor(mySubGraph, a, b, c);
\end{lstlisting}

The number and order of the variables must correspond to the number and order of the connector variables declared when the sub-graph was created.

\subpara{Specifying FactorTables}

For all discrete factors, Dimple can build FactorTables under the hood. A Factor Table consists of a two dimensional array of integers and a single dimensional array of doubles. Each row of the two dimensional table represents a combination of variables for which the Factor Function returns a non-zero value. Each column represents a variable connected to the factor. The values of the table specify an index into the discrete domain of a variable. Each row of the two dimensional table corresponds to one entry of the array of doubles and that entry contains the return value of the Factor Function.

In some situations, a user might be able to build a FactorTable much more efficiently than Dimple can. Dimple tries all combinations of variable values exhaustively and discards combinations for which the Factor Function returns zeros. If the user can avoid this exhaustive search, they can build the FactorTable more quickly.

Once the user has created the table, they can add a Factor using this table in one of two ways. The first is to provide the indices and values directly to the addFactor call:

\begin{lstlisting}
   %First let's just add the factor as indices and values
   fg = FactorGraph();
   b = Bit(2,1);
   ind = [0 0; 1 1];
   val = [1 1];
   fg.addFactor(ind,val,b);
   b(1).Input = .8;
   fg.solve();
   assertEqual(b(2).Belief,.8);
\end{lstlisting}
   
Notice that the indices and values chosen here will be equivalent to creating an equals factor between the two variables. In the following example we see that it's possible to first create a Table and then add a factor using that table. This has the advantage of using less overall memory if this same table will be used in multiple factors.

\begin{lstlisting}
   %Now let's add the Factor using createTable to test we can share tables
   fg = FactorGraph();
   b = Bit(2,1);   
   t = FactorTable(ind,val,b.Domain,b.Domain);	
   fg.addFactor(t,b);
   b(1).Input = .8;
   fg.solve();
   assertEqual(b(2).Belief,.8);
\end{lstlisting}

Users can also create a FactorTable given a single n-dimensional matrix such that: $ val(x_1,...,x_n) = F(d_1(x_11),...,d_n(x_n)) $

Where val is the N dimensional matrix, F is the desired Factor Function, and di is a function mapping 1 based indices to domain elements of domain i.  This will result in an uncompressed table, so it doesn�t offer a speed improvement, but provides a convenient way to create Factors in some cases.

\begin{lstlisting}
   fg = FactorGraph();
   b = Bit(2,1);
   val = [1 0; 0 1];
   t = FactorTable(val,b.Domain,b.Domain);
   fg.addFactor(t,b);
   b(1).Input = .8;
   fg.solve();
   assertEqual(b(2).Belief,.8);
\end{lstlisting}

or equivalently:

\begin{lstlisting}
   fg = FactorGraph();
   b = Bit(2,1);
   val = [1 0; 0 1];
   fg.addFactor(val,b);
   b(1).Input = .8;
   fg.solve();
   assertEqual(b(2).Belief,.8);
\end{lstlisting}

In yet another way to specify FactorTables, users can create an empty Factor Table and use the �set� function to set values at specified domain combinations.

\begin{lstlisting}
   fg = FactorGraph();
   b = Discrete({'on','off'},2,1);
   t = FactorTable(b.Domain,b.Domain);
   t.set('on','on',1);
   t.set('off','off',1);
   fg.addFactor(t,b);
   b(1).Input = [.8 .2];
   fg.solve();
   assertEqual(b(2).Belief(1),.8);
\end{lstlisting}

Users can retrieve values using the FactorTable �get� function.

\begin{lstlisting}
>> t.get('on','on')

ans =

     1
\end{lstlisting}


\subpara{Built-in Factors}
\label{sec:builtInFactors}

Dimple supports a set of built-in factors that can be specified when adding a factor to a graph.  When calling the addFactor method, in many cases, the user may specify these functions by name (which is case sensitive).  In some cases, there are optional arguments for these built-in factor, in which case an alternative syntax must be used (see below).

As an example,

\begin{lstlisting}
Mean = Real();
InverseVariance = Real();
Values = Real(1,100);
MyGraph.addFactor('Normal', Mean, InverseVariance, Values);
\end{lstlisting}

The following table lists the current set of built-in Dimple factors.  For each, the name is given, followed by the set of variables that would be connected to the factor, followed by any constructor arguments.  Optional variables and constructor arguments are in brackets.  And an arbitrary length list or vector of variables is followed by ellipses.  The allowed set of variable data-types for each variable is given in parentheses (B = Bit, D = Discrete, or R = Real).  If no list is shown, then any data-type may be used.

\begin{longtable} {l p{2.2cm} p{2cm} p{7cm}}
Name & Variables & Constructor & Description \\
\hline
\endhead
%
Abs & out(D,R) \newline in(D,R) & [smoothing] & Deterministic absolute value function, where out = abs(in).  An optional smoothing value may be specified as a constructor argument\footnote{\label{ftn:smoothing}If smoothing is enabled, the factor function becomes $e^{-(\textrm{out} - F(\textrm{in}))^2/\textrm{smoothing}}$ (making it non-deterministic) instead of $\delta(\textrm{out} - F(\textrm{in}))$, where $F$ is the deterministic function associated with this factor.  This is useful for solvers that do not work well with deterministic real-valued factors, such as particle BP, particularly when tempering is used.}.\\
%
ACos & out(D,R) \newline in(D,R) & [smoothing] & Deterministic arc-cosine function, where out = acos(in). An optional smoothing value may be specified as a constructor argument$^{\ref{ftn:smoothing}}$. \\
%
AdditiveNoise & out(R) \newline in(B,D,R) & $\sigma$ & Add Gaussian noise with a known standard deviation, $\sigma$, specified in constructor. \\
%
And & out(B) \newline in...(B) & - & Deterministic logical AND function, where out = AND(in...). \\
%
ASin & out(D,R) \newline in(D,R) & [smoothing] & Deterministic arc-sine function, where out = asin(in). An optional smoothing value may be specified as a constructor argument$^{\ref{ftn:smoothing}}$. \\
%
ATan & out(D,R) \newline in(D,R) & [smoothing] & Deterministic arc-tangent function, where out = atan(in). An optional smoothing value may be specified as a constructor argument$^{\ref{ftn:smoothing}}$. \\
%
Beta & [$\alpha$](R) \newline [$\beta$](R) \newline value...(R) & [$\alpha$] \newline [$\beta$] & Beta distribution. There can be any number of value variables, all associated with the same parameter values.  Parameters $\alpha$ and $\beta$ can be variables, or if both are constant they can be specified in the constructor. \\
%
Categorical & $\alpha$...(R)  \newline x...(D) & N & Categorical distribution, $p(x | \alpha)$, where $\alpha$ is a vector of parameter variables and x is a vector of discrete variables.  The number of elements in $\alpha$ and the domain size of x must equal the value of the constructor argument, N.  There can be any number of x variables, all associated with the same parameter values.  \newline
The $\alpha$ parameters are represented as energy values, that is, $\alpha = -\log(\rho)$, where $\rho$ are unnormalized probabilities. The conjugate prior for this representation is such that each entry of $\alpha$ is independently distributed according to a negative exp-Gamma distribution, all with a common $\beta$ parameter (it is not necessary to use the conjugate prior, but in some cases there may be a benefit).  \newline
In the current implementation, the domain of the x variable must be zero-based contiguous integers, $0...N-1$ (this limitation may be lifted in a future version). \\
%
ConstantPower & out(D,R) \newline base(D,R) & power \newline [smoothing] & Deterministic power function, with a constant power. The power value is specified in the constructor. An optional smoothing value may be specified as a constructor argument$^{\ref{ftn:smoothing}}$. \\
%
ConstantProduct & out(D,R) \newline in(D,R) & constant \newline [smoothing] &  Deterministic product function, multiplying the input times a constant value.  The constant value is specified in the constructor. An optional smoothing value may be specified as a constructor argument$^{\ref{ftn:smoothing}}$. \\
%
Cos & out(D,R) \newline in(D,R) & [smoothing] & Deterministic cosine function, where out = cos(in). An optional smoothing value may be specified as a constructor argument$^{\ref{ftn:smoothing}}$. \\
%
Cosh & out(D,R) \newline in(D,R) & [smoothing] & Deterministic hyperbolic-cosine function, where out = cosh(in). An optional smoothing value may be specified as a constructor argument$^{\ref{ftn:smoothing}}$. \\
%
DiscreteTransition & x(D) \newline y(D) \newline A(R) & $N_{x}, N_{y} | \newline N$ & 
Parameterized discrete transition factor, $p(y | x, A)$, where x and y are discrete variables, and $A$ is a matrix of transition probabilities. The transition matrix is organized such that columns correspond to the output distribution for each input state. That is, the transition matrix multiplies on the left. The number of columns in A and the domain size of x must equal the value of the constructor argument, $N_{x}$ and the number of rows in A and the domain size of y must equal the value of the constructor argument $N_{y}$.  If $N_{x}$ and $N_{y}$ are equal, a single constructor argument, $N$, may be used.  \newline
The elements of the matrix A are represented as energy values, that is, $A_{i,j} = -\log(\rho_{i,j})$, where $\rho$ is an unnormalized transition probability matrix.  The conjugate prior for this representation is such that each entry of A is independently distributed according to a negative exp-Gamma distribution, all with a common $\beta$ parameter (it is not necessary to use the conjugate prior, but in some cases there may be a benefit).  \newline
In the current implementation, the domain of the x variable must be zero-based contiguous integers, $0...N-1$ (this limitation may be lifted in a future version). \\
%
Divide & quotient(D,R) \newline dividend(D,R) \newline divisor(D,R) & [smoothing] & Deterministic divide function, where $\mathrm{quotient} = \frac{\mathrm{dividend}}{\mathrm{divisor}}$. An optional smoothing value may be specified as a constructor argument$^{\ref{ftn:smoothing}}$. \\
%
Equality & value...(B,D,R) & [smoothing] & Deterministic equality constraint.  An optional smoothing value may be specified as a constructor argument$^{\ref{ftn:smoothing}}$. \\
%
Equals & out(B) \newline in...(B,D,R) & - & Deterministic equals function, where out~=~(in(1)~==~in(2)~== ... ). \\
%
Exp & out(D,R) \newline in(D,R) & [smoothing] & Deterministic exponentiation function, where out = exp(in). An optional smoothing value may be specified as a constructor argument$^{\ref{ftn:smoothing}}$. \\
%
Gamma & [$\alpha$](R) \newline [$\beta$](R) \newline value...(R) & [$\alpha$] \newline [$\beta$] & Gamma distribution. There can be any number of value variables, all associated with the same parameter values.  Parameters $\alpha$ and $\beta$ can be variables, or if both are constant they can be specified in the constructor. \\
%
GreaterThan & out(B) \newline in1(B,D,R) \newline in2(B,D,R) & - & Deterministic greater-than function, where out = in1 $>$ in2.  \\
%
InverseGamma & [$\alpha$](R) \newline [$\beta$](R) \newline value...(R) & [$\alpha$] \newline [$\beta$] & Inverse Gamma distribution. There can be any number of value variables, all associated with the same parameter values.  Parameters $\alpha$ and $\beta$ can be variables, or if both are constant they can be specified in the constructor. \\
%
LessThan & out(B) \newline in1(B,D,R) \newline in2(B,D,R) & - & Deterministic greater-than function, where out = in1 $<$ in2.  \\
%
LinearEquation & out(D,R) \newline in(B,D,R) & constants \newline [smoothing] & Deterministic linear equation, multiplying an input vector by a constant vector. The constant vector is specified in the constructor.  The number of \emph{in} variables must equal the length of the constant vector. An optional smoothing value may be specified as a constructor argument\textsuperscript{\ref{ftn:smoothing}}. \\
%
Log & out(D,R) \newline in(D,R) & [smoothing] & Deterministic natural log function, where out = log(in). An optional smoothing value may be specified as a constructor argument$^{\ref{ftn:smoothing}}$. \\
%
LogNormal & [$\mu$](R) \newline [$\tau$](R) \newline value...(R) & [$\mu$] \newline [$\tau$] & Log-normal distribution. There can be any number of value variables, all associated with the same parameter values.  Parameters $\mu$ (mean) and $\tau = \frac{1}{\sigma^{2}}$ (precision) can be variables, or if both are constant then fixed parameters can be specified in the constructor. \\
%
MatrixVectorProduct & y(D,R) \newline M(D,R) \newline x(D,R) & $N_{x}$ \newline $N_{y}$ \newline [smoothing] & Deterministic matrix-vector product function, $y = Mx$, where $x$ and $y$ are vectors and $M$ is a matrix. Constructor arguments, $N_{x}$ and $N_{y}$, specify the input and output vector lengths, respectively. The matrix dimension is $N_{y} \times N_{x}$. An optional smoothing value may be specified as a constructor argument$^{\ref{ftn:smoothing}}$. \\
%
MixedNormal & value(R) \newline control(B) & $\mu_{0} \newline \tau_{0} \newline \mu_{1} \newline \tau_{1}$ & Simple mixture of two fixed-parameter Normal distributions. The choice of distribution parameters (0 vs. 1) is a function of the control bit. \\
%
Negate & out(D,R) \newline in(D,R) & [smoothing] & Deterministic negation function, where out = -in. An optional smoothing value may be specified as a constructor argument$^{\ref{ftn:smoothing}}$. \\
%
NegativeExpGamma & [$\alpha$](R) \newline [$\beta$](R) \newline value...(R) & [$\alpha$] \newline [$\beta$] & Negative exp-Gamma distribution, which is a distribution over a variable whose negative exponential is Gamma distributed. That is, this is the negative log of a Gamma distributed variable. There can be any number of value variables, all associated with the same parameter values.  Parameters $\alpha$ and $\beta$ can be variables, or if both are constant they can be specified in the constructor, and correspond to the parameters of the underlying Gamma distribution. \\
%
Normal & [$\mu$](R) \newline [$\tau$](R) \newline value...(R) & [$\mu$] \newline [$\tau$] & Normal distribution. There can be any number of value variables, all associated with the same parameter values.  Parameters $\mu$ (mean) and $\tau = \frac{1}{\sigma^{2}}$ (precision) can be variables, or if both are constant then fixed parameters can be specified in the constructor. \\
%
Not & out(B) \newline in(B) & - & Deterministic logical NOT of function, where out = ~in. \\
%
NotEquals & out(B) \newline in...(B,D,R) & - & Deterministic not-equals function, where out~=~$\sim$(in(1)~==~in(2)~== ... ). \\
%
Or & out(B) \newline in...(B) & - & Deterministic logical OR function, where out = OR(in...). \\
%
Power & out(D,R) \newline base(D,R) \newline power(D,R) & [smoothing] & Deterministic power function, where out~=~base~$^{\mathrm{power}}$. An optional smoothing value may be specified as a constructor argument$^{\ref{ftn:smoothing}}$. \\
%
Product & out(D,R) \newline in...(B,D,R) & [smoothing] & Deterministic product function, where $\mathrm{out} = \prod \mathrm{in}$. An optional smoothing value may be specified as a constructor argument$^{\ref{ftn:smoothing}}$. \\
%
Rayleigh & [$\sigma$](R) \newline value...(R) & [$\sigma$] & Rayleigh distribution. There can be any number of value variables, all associated with the same parameter value.  Parameter $\sigma$ can be a variable, or if constant, can be specified in the constructor. \\
%
Sin & out(D,R) \newline in(D,R) & [smoothing] & Deterministic sine function, where out = sin(in). An optional smoothing value may be specified as a constructor argument$^{\ref{ftn:smoothing}}$. \\
%
Sinh & out(D,R) \newline in(D,R) & [smoothing] & Deterministic hyperbolic-sine function, where out = sinh(in). An optional smoothing value may be specified as a constructor argument$^{\ref{ftn:smoothing}}$. \\
%
Sqrt & out(D,R) \newline in(D,R) & [smoothing] & Deterministic square root function, where out = sqrt(in). An optional smoothing value may be specified as a constructor argument$^{\ref{ftn:smoothing}}$. \\
%
Square & out(D,R) \newline in(D,R) & [smoothing] & Deterministic square function, where out = in$^{2}$. An optional smoothing value may be specified as a constructor argument$^{\ref{ftn:smoothing}}$. \\
%
Subtract & out(D,R) \newline posIn(B,D,R) \newline negIn...(B,D,R) & [smoothing] & Deterministic summation function, where $\mathrm{out} = \mathrm{posIn} - \sum \mathrm{negIn}$. An optional smoothing value may be specified as a constructor argument$^{\ref{ftn:smoothing}}$. \\
%
Sum & out(D,R) \newline in...(B,D,R) & [smoothing] & Deterministic summation function, where $\mathrm{out} = \sum \mathrm{in}$. An optional smoothing value may be specified as a constructor argument$^{\ref{ftn:smoothing}}$. \\
%
Tan & out(D,R) \newline in(D,R) & [smoothing] & Deterministic tangent function, where out = tan(in). An optional smoothing value may be specified as a constructor argument$^{\ref{ftn:smoothing}}$. \\
%
Tanh & out(D,R) \newline in(D,R) & [smoothing] & Deterministic hyperbolic-tangent function, where out = tanh(in). An optional smoothing value may be specified as a constructor argument$^{\ref{ftn:smoothing}}$. \\
%
VonMises & [$\mu$](R) \newline [$\tau$](R) \newline value...(R) & [$\mu$] \newline [$\tau$] & Von Mises distribution. There can be any number of value variables, all associated with the same parameter values.  Parameters $\mu$ (mean) and $\tau = \frac{1}{\sigma^{2}}$ (precision) can be variables, or if both are constant then fixed parameters can be specified in the constructor.  The distribution is non-zero for value variables in the range $-\pi$ to $\pi$. \\
%
Xor & out(B) \newline in...(B) & - & Deterministic logical XOR function, where out = XOR(in...). \\
%
\end{longtable}

The above factors may be specified by name if no constructor arguments are needed.  If constructor arguments are needed, then there are two ways to specify the factor.  The preferred way is to create a FactorFunction object, which takes the name of the factor followed by the constructor arguments for that factor.  For example:

\begin{lstlisting}
MyGraph.addFactor(FactorFunction('Gamma', 1, 1), X);
\end{lstlisting}

The same FactorFunction can be used more than once, which avoids creating additional copies of the FactorFunction object.  For example:

\begin{lstlisting}
myFactorFunction = FactorFunction('Gamma', 1, 1);
MyGraph.addFactor(myFactorFunction, X1);
MyGraph.addFactor(myFactorFunction, X2);
\end{lstlisting}

A short-hand notation may alternatively be used, in which the name of the factor function and its constructor arguments are contained in a cell array.  For example:

\begin{lstlisting}
MyGraph.addFactor({'Gamma', 1, 1}, X);
\end{lstlisting}


In addition to the built-in factors listed above, there are a set of solver-specific built-in factors, also referred to as ``custom factors.''  These factors may also be specified by name in the addFactor call (using either a quoted string or as a function handle).  These are summarized in the following table:

\begin{longtable} {l p{3cm} p{7cm}}
Name & Solver & Description \\
\hline
\endhead
%
FiniteFieldFactor & SumProduct\footnote{\label{ftn:fff} These may also be used for discrete variables with the Gaussian or particle BP solvers, which use the sum-product solver for discrete-only portions of the graph.}  & See section~\ref{sec:finiteFields} \\ 
FiniteFieldProjection & SumProduct$^{\ref{ftn:fff}}$ & See section~\ref{sec:finiteFields} \\ 
FiniteFieldAdd & SumProduct$^{\ref{ftn:fff}}$ & See section~\ref{sec:finiteFields} \\ 
FiniteFieldConstMult & SumProduct$^{\ref{ftn:fff}}$ & See section~\ref{sec:finiteFields} \\ 
FiniteFieldMult & SumProduct$^{\ref{ftn:fff}}$ & See section~\ref{sec:finiteFields} \\ 
CustomXor & MinSum & Same as the Xor factor described above, but with a significantly faster implementation. \\ 
\end{longtable}


Dimple also includes some built-in helper functions to create structured graphs, combining smaller factors to form an efficient implementation of a larger subgraph.  Specifically, the following functions are provided.

\begin{itemize}
\item getNBitXorDef(n), where n is a positive integer. Returns a nestable graph and an array of n-Bit connector variables. Efficient tree implementation of the XORDelta function.
\item getVXOR(n), where n is a positive integer. Returns a nestable graph and an array of n-Bits connector variables. Constrains exactly one bit to be 1, and all others to be 0.
\end{itemize}




\subpara{Adding Factors Implicitly Using MATLAB Operators or Functions}

Dimple supports a set of built-in factors that can be added implicitly using overloaded MATLAB operators or functions.  For example,

\begin{lstlisting}
fg = FactorGraph();
a = Discrete(1:4);
b = Discrete(1:10);
c = a + b;
\end{lstlisting}

The last line of this example will create a new variable, c, and a 'Sum' factor with variables c, a, and b.  The domain of c will be defined appropriately given the domains of the input variables.  In this example, the domain of c would automatically be set to the range [2:14].

These operations can be compounded in a single line of code, and variables of different data types as well as constants can be intermingled (as long as the type is supported by the specific operator).  In this case, intermediate anonymous variables will be created in the graph associated with intermediate results of the operation.  For example,

\begin{lstlisting}
z = (a + b) * c^d - sqrt(-e);
\end{lstlisting}

Like using the addFactor method, adding factors implicitly can include constants.  Specifically, for binary operators, one of the inputs may be a constant instead of a variable.  For example:

\begin{lstlisting}
x = a^2;
y = (a + b + 2) * 3;
\end{lstlisting}

Since adding a factor implicitly does not specifically refer to the factor graph, the graph to which these factors are added is also implicit.  In particular, these implicitly defined factors are added to the last factor graph that was created.  So, in the first example above, the factor would be added to fg, regardless of whether other factor graphs had previously been created.

The specific set of operators supported is given in section~\ref{sec:overloaded}.



\subpara{Specifying Java Factor Functions}
\label{sec:userJava}

There are some cases in which it is desirable to add a factor function that is defined in Java rather than MATLAB. Specific cases where this is desirable are:

\begin{itemize}
\item The user wants to use a sample-based solver. Dimple's sample based solvers must use Java Factor Functions in order to be able to call the Factor Function at runtime. This is partly because MATLAB does not officially support callbacks from Java to MATLAB and also because calls from Java to Java will be faster than those that cross a language boundary.
\item The user wants their code to run faster. MATLAB function calls are very slow as compared to java function calls.
The following sections provides the steps users must follow to add a Java Factor Function to their FactorGraph.
\end{itemize}


\subparaNoToc{Create a Class That Inherits from Java FactorFunction}
\label{sec:createJavaFactorFunction}

Users must extend the FactorFunction class. They have to provide two methods:

\begin{itemize}
\item A constructor that call's the parent constructor.
\item An eval function that returns a weight for each possible set of inputs.
\end{itemize}

\begin{lstlisting}
import com.analog.lyric.dimple.FactorFunctions.core.FactorFunction;

/*
 * This factor enforces equality between all variables and weights
 * elements of the domain proportional to their value
 */
public class BigEquals extends FactorFunction
{	
    public BigEquals() 
    {
	super("BigEquals");
    }
  
    @Override
    public double eval(Object... input) throws Exception 
    {
	if (input.length == 0)
	    return 0;
	else
	{
	    double first = (Double) input[0];
	    
	    for (int i = 1; i < input.length; i++)
	    {
	        if ((Double)input[i] != first)
		    return 0;
	    }
	    return first;   
	}
    }
}
\end{lstlisting}

\subparaNoToc{Compiling}

The new class must be compiled to class files. Users can optionally create a jar file. If using Eclipse, users can simply create a new project, create the new class, and the .class files will be created automatically.

\subparaNoToc{Adding Binary to MATLAB Path}

In MATLAB, the user must use the javaaddpath call to add the java files to the javaclasspath.

\begin{lstlisting}
javaaddpath('<path to my project>/MyFactorFunctions/bin');
\end{lstlisting}

or

\begin{lstlisting}
javaaddpath('<path to the jar>/myjar.jar');
\end{lstlisting}

\subpara{Using the Factor Function}

Users can now instantiate the FactorFunction and call addFactor with the instance.

\begin{lstlisting}
be = BigEquals();
fg = FactorGraph(); 
v = Variable({0,1,2},2,1);
fg.addFactor(be,v(1),v(2));
fg.solve();
v(1).Belief
\end{lstlisting}

Note that the BigEquals class was added to the default package, so we can just call:

\begin{lstlisting}
be = BigEquals();
\end{lstlisting}

The previous code results in the following answer from MATLAB:

\begin{lstlisting}
ans =

   0.0000    0.3333    0.6667
\end{lstlisting}

\subparaNoToc{Re-using the Factor Function}
We can re-use the BigEquals() FactorFunction multiple times within a single graph or across graphs. This will result in caching of FactorTables when the arguments to the addFactor function have identical domains.

\begin{lstlisting}
be = BigEquals();
fg = FactorGraph(); 
v = Variable({0,1,2},2,1);
fg.addFactor(be,v(1),v(2));
fg.solve();
v(1).Belief

fg2 = FactorGraph();
v2 = Variable({0,1,2},2,1);
fg2.addFactor(be,v2(1),v2(2));
fg2.solve();
v2(1).Belief
\end{lstlisting}


\subpara{Constants}

Users can pass constants to addFactor.  The following passes the constant 1 to the xorDelta factor function as the final argument.

\begin{lstlisting}
fg = FactorGraph();
b = Bit(10,3);
fg.addFactorVectorized(@xorDelta,{b 1},1);
\end{lstlisting}


\para{addFactorNoCache}
\label{sec:addFactorNoCache}

There are several cases where Dimple�s factor caching mechanism is undesirable.  As an example, if the user is calling addFactor with the same factor many times, but they know the arguments are unique, they can save time by skipping the code that searches for a match in the cache. 

\begin{lstlisting}
fg = FactorGraph(); 
func = @(x,y) 1;
b = Bit(2,1);
for i=1:1000
fg.addFactorNoCache(func,b,rand());
end
\end{lstlisting}

This is a somewhat contrived example but, in this case, replacing addFactorNoCache with addFactor will significantly slow down the code.




\para{addFactorVectorized}

To get reasonable speed out of MATLAB, one needs to vectorize their code.  If a user wishes to build a FactorGraph with large numbers of factors in a regular arrangement, they will want to avoid making many calls to addFactor.  The addFactorVectorized method can be used instead to create many factors at once.

\subpara{Vectorizing over All Variables}

In this first example, we create three variable matrices and generate 12 factors connecting a(i,j), b(i,j), and c(i,j) for all i and j.

\begin{lstlisting}
fg = FactorGraph();
a = Bit(3,4);
b = Bit(3,4);
c = Bit(3,4);
fg.addFactorVectorized(@xorDelta,a,b,c);
\end{lstlisting}

\subpara{Specify Vectorize Dimensions}

By default addFactorVectorized will add factors for every dimension.  However, if a user wants to only vectorize over a subset of dimensions, addFactorVectorize
allows them to provide a cell array containing a variable matrix as the first argument and a list of dimensions to vectorize over as the second argument.  The following code will vectorize over the first two dimensions to create 12 xor factors.  Each xor factor is connected to 5 bits.

\begin{lstlisting}
fg = FactorGraph();
b = Bit(3,4,5);
fg.addFactorVectorized(@xorDelta,{b,[1 2]});
\end{lstlisting}

\subpara{Connecting a single variable with addFactorVectorized}

The addFactorVectorized method will ensure that dimensions being vectorized match for all inputs.  However, the call allows single variables to be connected to many as follows:

\begin{lstlisting}
fg = FactorGraph();
a = Bit(10,1);
b = Bit();
fg.addFactorVectorized(@xorDelta,a,b);
\end{lstlisting}

This will create a graph such that b is connected to every element of a via a total of 10 factors.

\subpara{repmat}

Similar things can be done with repmat.  Suppose you have a 10x1 vector of variables that you want connected to a 10x20 matrix of different variables.
The following code will create 200 factors.  Each element of a will be connected to 20 elements of b via 20 factors.

\begin{lstlisting}
fg = FactorGraph();
a = Bit(10,1);
b = Bit(10,20);
fg.addFactorVectorized(@xorDelta,repmat(a,1,10),b);
\end{lstlisting}

\para{Nested Graphs}

Nested Graphs are also supported

\begin{lstlisting}
a = Bit(3,1);
ng = FactorGraph(a);
ng.addFactor(@xorDelta,a);
fg = FactorGraph();
b = Bit(10,3);
fg.addFactorVectorized(ng,{b, 1});
\end{lstlisting}


\para{Adding Vectorized Factors Implicitly Using MATLAB Operators or Functions}

Adding built-in factors implicitly using overloaded MATLAB operators or functions can also be vectorized, with some limitations.  Specifically, if each of the input variables are vectors of the same dimension, then the result will be to create a vector of output variables of the same dimension, along with a vector of factors relating the inputs and outputs.

In some cases, to be consistent with MATLAB notation, there is a distinction made between the vectorized and non-vectorized operator.  Specifically, Dimple uses MATLAB's notation for point wise product and power operators to indicate a vectorized operation (currently using the matrix version of these operators on vector inputs is not supported).  For example, if variables a through e are vectors of variables of identical size, then the following would create a variable vector z, and a series of factors relating these variables.

\begin{lstlisting}
z = (a .* b) + c.^d - sqrt(-e);
\end{lstlisting}

For binary operators, one of the inputs may be a scalar variable or a scalar constant instead of a variable vector.  For a scalar variable, the result is that scalar variable connecting to each instance of the factors that are created.  For a constant, each instance of the factor uses the same constant for that input (vectors of distinct constants are not currently supported).  As an example:

\begin{lstlisting}
a = Real();
b = Discrete(domain, 1, 10);
z = a + b;					
\end{lstlisting}

The specific set of operators supported is given in section~\ref{sec:overloaded}.




\para{addDirectedFactor}

\para{addFactorNoCache}

\para{addRepeatedFactor}

\para{setSolver}
\para{Using a Name}

Dimple has been modified to allow users to register commonly used solvers so that they can be assigned using shorter names. Previously users had to do:

\begin{lstlisting}
setSolver(com.analog.lyric.dimple.solvers.sumproduct.Solver());
fg = FactorGraph();
\end{lstlisting}

or

\begin{lstlisting}
fg = FactorGraph();
fg.Solver = com.analog.lyric.dimple.solvers.sumproduct.Solver();
Users can now do:
setSolver('SumProduct'); 
fg = FactorGraph();
\end{lstlisting}

or

\begin{lstlisting}
fg = FactorGraph();
fg.Solver = 'SumProduct';
\end{lstlisting}

When setting the solver by name, the name is case insensitive.

For solvers that take constructor arguments, users can use the setSolver command:

\begin{lstlisting}
fg = FactorGraph();
fg.setSolver('Gibbs',param1,param2,param3,param4);
\end{lstlisting}

or

\begin{lstlisting}
setSolver('Gibbs',param1,param2,param3,param4);
\end{lstlisting}

For a full list of registered solvers, type:

\begin{lstlisting}
getSolverNames()
\end{lstlisting}

If users wish to register their own solvers they can call:

\begin{lstlisting}
registerSolver(solverName,solverConstructor);
\end{lstlisting}

And if they wish to unregister a solver they can call:

\begin{lstlisting}
unregisterSolver(solverName);
\end{lstlisting}




\para{advance}

\para{hasNext}

\para{solveOneStep}

\para{continueSolve}

\para{baumWelch}

\para{estimateParameters}

\para{join}

\para{split}

\para{removeFactor}

\para{plot}

%\para{getVariableByName}
%
%\para{getVariableByUUID}
%
%\para{getFactorByName}
%
%\para{getFactorByUUID}
%
%\para{getGraphByName}
%
%\para{getGraphByUUID}
%
%\para{isTree}
%
%\para{isTreeFlat}
%
%\para{isTreeTop}
%
%\para{isAncestorOf}
%
%\para{depthFirstSearch}
%
%\para{depthFirstSearchFlat}
%
%\para{depthFirstSearchTop}
%
%\para{getFactorGraphDiffsByName}
%
%\para{getAdjacencyMatrix}
%
%\para{getAdjacencyMatrixFlat}
%
%\para{getAdjacencyMatrixTop}
%
%\para{getAdjacencyString}
%
%\para{getNodeString}
%
%\para{getFullString}
%
%\para{disp}







\subsubsection{List of Overloaded MATLAB Operators and Functions}
\label{sec:overloaded}

The following table lists the set of overloaded MATLAB operators that can be used to implicitly create factors.  The table shows the operator, the corresponding factor, the valid variable data types of the inputs and outputs (B = Bit, D = Discrete, or R = Real), and wether or not vectorized inputs are supported.

\begin{longtable} {l p{3cm} p{1cm} p{1cm} p{1cm} l p{4cm}}
Operator & Factor & Out & In1 & In2 & Vectorized & Description \\
\hline
\endhead
%
$\&$ & And & B & B & B & \checkmark & Logical AND \\
$|$ & Or & B & B & B & \checkmark & Logical OR \\
xor() & Xor & B & B & B & \checkmark & Logical XOR \\
$\sim$ & Not & B & B & - & \checkmark & Logical NOT \\
$+$ & Sum & D,R\footnote{\label{ftn:outReal}If either input is Real, then the output is Real} & D,R & D,R & \checkmark & Plus \\
$-$ & Subtract & D,R\textsuperscript{\ref{ftn:outReal}} & D,R & D,R & \checkmark & Minus \\
$-$ & Negate & D,R\textsuperscript{\ref{ftn:outReal}} & D,R & - & \checkmark & Unary minus \\
$*$ & Product \newline MatrixVectorProduct & D,R\textsuperscript{\ref{ftn:outReal}} & D,R & D,R & \checkmark\footnote{One of the inputs may be a vector as long as the other is a scalar.} & Scalar multiply, or matrix-vector multiply\footnote{If one input is a vector and the other is a matrix of appropriate dimension, then the MatrixVectorProduct factor will be used.  Otherwise the Product factor will be used.} \\
$.*$ & Product & D,R\textsuperscript{\ref{ftn:outReal}} & D,R & D,R & \checkmark & Point-wise multiply \\
$/$ & Divide & D,R\textsuperscript{\ref{ftn:outReal}} & D,R & D,R & \checkmark\footnote{The dividend may be a vector as long as the divisor is a scalar.} & Scalar divide \\
$./$ & Divide & D,R\textsuperscript{\ref{ftn:outReal}} & D,R & D,R & \checkmark & Point-wise divide \\
$\wedge$ & Power & D,R\textsuperscript{\ref{ftn:outReal}} & D,R & D,R & \checkmark\footnote{The base may be a vector as long as the exponent is a scalar.} & Scalar power \\
$.\wedge$ & Power & D,R\textsuperscript{\ref{ftn:outReal}} & D,R & D,R & \checkmark & Point-wise power \\
$<$ & LessThan & B & D,R & D,R & \checkmark & Less than \\
$>$ & GreaterThan & B & D,R & D,R & \checkmark & Greater than \\
$<=$ & GreaterThan\footnote{Uses GreaterThan factor, reversing the order.} & B & D,R & D,R & \checkmark & Less than or equal to \\
$>=$ & LessThan\footnote{Uses LessThan factor, reversing the order.} & B & D,R & D,R & \checkmark & Greater than or equal to \\
Equals() & Equals & B & B,D,R & B,D,R\footnote{\label{ftn:equals}This function is not limited to two inputs, but can take an arbitrary number of inputs} & \checkmark & Equals\footnote{Equivalent to the $==$ operator, but the $==$ operator is not overloaded for this purpose so that it can instead be used to determine whether or not two variables reference the same Dimple variable.} \\
NotEquals() & NotEquals & B & B,D,R & B,D,R$^{\ref{ftn:equals}}$ & \checkmark & Not equals\footnote{Equivalent to the $\sim=$ operator, but the $\sim=$ operator is not overloaded for this purpose so that it can instead be used to determine whether or not two variables reference the same Dimple variable.} \\
mod() & - & D  & D & D & \checkmark & Modulo function\footnote{Currently, the mod() operator supports discrete variables only, and it uses the MATLAB definition of mod on negative numbers.  This may be subject to change in future versions.} \\
abs() & Abs & D,R\textsuperscript{\ref{ftn:outReal}} & D,R & - & \checkmark & Absolute value \\
sqrt() & Sqrt & R & R & - & \checkmark & Square root \\
log() & Log & R & R & - & \checkmark & Natural log \\
exp() & Exp & R & R & - & \checkmark & Exponential function \\
sin() & Sin & R & R & - & \checkmark & Sine \\
cos() & Cos & R & R & - & \checkmark & Cosine \\
tan() & Tan & R & R & - & \checkmark & Tangent \\
asin() & ASin & R & R & - & \checkmark & Arc-sine \\
acos() & ACos & R & R & - & \checkmark & Arc-cosine \\
atan() & ATan & R & R & - & \checkmark & Arc-tangent \\
sinh() & Sinh & R & R & - & \checkmark & Hyperbolic sine \\
cosh() & Cosh & R & R & - & \checkmark & Hyperbolic cosine \\
tanh() & Tanh & R & R & - & \checkmark & Hyperbolic tangent \\
\end{longtable}


Dimple also includes some helper functions to create other built-in factors using a similar syntax to the overloaded MATLAB functions listed above.  As for other overloaded functions, above, Dimple automatically creates the factors as well as the output variable(s).  The built-in factors that support this syntax are:

\begin{itemize}
\item Beta
\item Gamma
\item InverseGamma
\item NegativeExpGamma
\item Normal
\item LogNormal
\item VonMises
\item Rayleigh
\item Categorical
\end{itemize}

For each, the arguments are the parameters of the distribution.  For example:

\begin{lstlisting}
W = Gamma(alpha, beta);
X = Normal(mean, precision);
Y = Categorical(alphaVector);
Z = Rayleigh(sigma);
\end{lstlisting}


The parameters can be variables, constants, or some of each.

By default, calling one of these functions creates a single output variable, and the factor is added to the most-recently created graph. �But, optional arguments allow you to specify the dimensions of the array of output variables, or to specify the factor graph. �These arguments can be in either order after the parameters. �For example:

\begin{lstlisting}
W = Gamma(alpha, beta, altGraph);
X = Normal(mean, precision, [100, 1]);
Y = Categorical(alphaVector, [10, 10, 2], aGraph);
Z = Rayleigh(sigma, myGraph, size(somethingElse));
\end{lstlisting}



\subsubsection{Setting the Solver}



\subsubsection{Setting the Scheduler}

The scheduler determines the schedule to apply when solving the graph, which is the order of updates of variables and factors.  The user may choose among one of several pre-defined schedulers using the Scheduler method.  Each scheduler examines the structure of the graph and applies a rule to determine the schedule that will be used.  Alternatively, the user may supply a custom schedule using the Schedule method (see section~\ref{sec:customSchedules}).

If no scheduler or custom schedule is specified, a default scheduler will be used.  The default scheduler depends on the selected solver.

A scheduler is specified as follows:
\begin{lstlisting}
MyGraph.Scheduler = 'ExampleScheduler';	
\end{lstlisting}

When setting the scheduler for a graph, the name of the scheduler is case sensitive.

For all solvers except for the Gibbs solver (SumProduct, MinSum, Gaussian, ParticleBP), the following schedulers are available:

\begin{longtable}{l p{4in}}
\textbf{Name} & \textbf{Description} \\ \hline \hline
%
\textsf{DefaultScheduler} & Same as the TreeOrFloodingScheduler, which is the default if no scheduler or custom schedule is specified. \\ \hline
%
\textsf{TreeOrFloodingScheduler} & The solver will use either a Tree Schedule or a Flooding Schedule depending on whether the factor-graph contains cycles.  In a nested graph, this choice is applied independently in each subgraph.  If the factor-graph is a tree, the scheduler will automatically detect this and use a Tree Schedule.  In this schedule, each node is updated in an order that will result in the correct beliefs being computed after just one iteration.  If the entire graph is a tree, NumIterations should be set to 1, which is its default value.  If the factor-graph is loopy, the solver will instead use a Flooding Schedule (as described below). \\ \hline
%
\textsf{TreeOrSequentialScheduler} & The solver will use either a Tree Schedule (as described above) or a Sequential Schedule (as described below) depending on whether the factor-graph contains cycles.  In a nested graph, this choice is applied independently in each subgraph.  \\ \hline
%
\textsf{FloodingScheduler} & The solver will apply a Flooding Schedule.  For each iteration, all variable nodes are updated, followed by all factor nodes.  Because the graph is bipartite (factor nodes only connect to variable nodes, and vice versa), the order of update within each node type does not affect the result. \\ \hline
%
\textsf{SequentialScheduler} & The solver will apply a Sequential Schedule.  For each factor node in the graph, first, for each variable connected to that factor, the edge connecting the variable to the factor is updated; then the factor node is updated.  The specific order of factors chosen is arbitrary, and depends on the order that factors were added to the graph. \\ \hline
%
\textsf{RandomWithoutReplacementScheduler} & The solver will apply a Sequential Schedule with the order of factors chosen randomly without replacement.  On each subsequent iteration, a new random order is chosen.  Since the factor order is chosen randomly with replacement, on each iteration, each factor will be updated exactly once. \\ \hline
%
\textsf{RandomWithReplacementScheduler} & The solver will apply a Sequential Schedule with the order of factors chosen randomly with replacement.  On each subsequent iteration, a new random order is chosen.  The number of factors updated per iteration is equal to the total number of factors in the graph. However, since the factors are chosen randomly with replacement, not all factors are necessarily updated in a single iteration, and some may be updated more than once. \\ \hline
%
\end{longtable}


In a nested graph, for most of the schedulers listed above (except for the random schedulers), the schedule is applied hierarchically.  In particular, a subgraph is treated as a factor in the nesting level that it appears.  When that subgraph is updated, the schedule for the corresponding subgraph is run in its entirety, updating all factors and variables contained within according to its specified schedule.

It is possible for subgraphs to be designated to use a schedule different from that of its parent graph.  This can be done by specifying either a scheduler or a custom schedule for the subgraph prior to adding it to the parent graph.  For example:
%
\begin{lstlisting}
SubGraph.Scheduler = 'SequentialScheduler';
ParentGraph.addFactor(SubGraph, boundaryVariables);
ParentGraph.Scheduler = 'FloodingScheduler';
\end{lstlisting}

For the TreeOrFloodingScheduler and the TreeOrSequentialScheduler, the choice of schedule is done independently in the outer graph and in each subgraph.  In case that a subgraph is a tree, the tree scheduler will be applied when updating that subgraph even if the parent graph is loopy.  This structure can improve the performance of belief propagation by ensuring that the effect of variables at the boundary of the subgraph fully propagates to all other variables in the subgraph on each iteration.

For the RandomWithoutReplacementScheduler and RandomWithReplacementScheduler, if these are applied to a graph or subgraph, the hierarchy of any lower nesting layers is ignored.  That is, the subgraphs below are essentially flattened prior to schedule creation, and any schedulers or custom schedules specified in lower layers of the hierarchy are ignored.


Because of the differences in operation between the Gibbs solver and the BP based solvers, the Gibbs solver supports a distinct set of schedulers.  For the Gibbs solver, the following schedulers are available:

\begin{tabular}{l p{4in}}
\textbf{Name} & \textbf{Description} \\ \hline \hline
%
\textsf{GibbsDefaultScheduler} & Same as the GibbsSequentialScanScheduler, which is the default when using the Gibbs solver. \\ \hline
%
\textsf{GibbsSequentialScanScheduler} & The solver will apply a Sequential Scan Schedule.  For each scan, each variable is resampled in a fixed order.  The specific order of variables chosen is arbitrary, and depends on the order that variables were added to the graph. \\ \hline
%
\textsf{GibbsRandomScanScheduler} & The solver will apply a Random Scan Schedule.  Each successive variable to be resampled is chosen randomly with replacement.  The number of variables resampled per scan is equal to the total number of variables in the graph, but not all variables are necessarily resampled in a given scan, and some may be resampled more than once. \\ \hline
%
\end{tabular}

Because of the nature of the Gibbs solver, the nested structure of a graph is ignored in creating the schedule.  That is, the graph hierarchy is essentially flattened prior to schedule creation, and only the scheduler specified on the outermost graph is applied.


\subsubsection{Custom Schedules}
\label{sec:customSchedules}


Dimple supports user defined custom schedules created in MATLAB.  A custom schedule is specified using the Schedule method.  Specifying a custom schedule overrides any scheduler that the graph would otherwise use.

The following code demonstrates this feature:

\begin{lstlisting}
   eq = @(x,y) x == y;
   fg = FactorGraph();
   a = Bit();
   b = Bit();
   c = Bit();
   eq1 = fg.addFactor(eq,a,b);
   eq2 = fg.addFactor(eq,b,c);
   
   %define schedule
   % update b
   % update eq1->a
   % update eq2->c
   % update a->eq1
   % update c->eq2
   % update eq1->b
   % update eq2->b
   schedule = {
       b,
       {eq1,a},
       {eq2,c},
       {a,eq1},
       {c,eq2},
       {eq1,b},
       {eq2,b}
       };
   
   fg.Schedule = schedule;
   
   %Set priors
   a.Input = .6;
   b.Input = .7;
   c.Input = .8;
   
   %Solve
   fg.NumIterations = 1;
   fg.solve();
\end{lstlisting}

Dimple also supports nesting custom schedules and nesting in general. The following example demonstrates specifying nested graphs in a schedule.

\begin{lstlisting}
   eq = @(x,y) x == y;
   b = Bit(2,1);
   nfg = FactorGraph(b);
   nfg.addFactor(eq,b(1),b(2));
   b = Bit(3,1);
   fg = FactorGraph();
   nf1 = fg.addFactor(nfg,b(1),b(2));
   nf2 = fg.addFactor(nfg,b(2),b(3));
   
   
   fg.Schedule = {b(1),nf1,b(2),nf2,b(3)};
   b(1).Input = .7;
   fg.NumIterations = 1;
   fg.solve();
\end{lstlisting}

And finally we look at nesting a custom schedule:


\begin{lstlisting}
   %Now let's try nesting with a custom schedule on the nested graph.
   
   %Create a graph to nest and give it a funny schedule    
   % nfg: eb(1) - f1 - ib - f2 - eb(2)
   eb = Bit(2,1);
   ib = Bit();
   nfg = FactorGraph(eb);
   f1 = nfg.addFactor(eq,eb(1),ib);
   f2 = nfg.addFactor(eq,ib,eb(2));
   %Set an input and solve
   eb(1).Input = .8;
   
   nfg.NumIterations = 1;
   nfg.solve();
   
   %We expect the output to be equal to the input since the tree
   %scheduler passes the info along.
   assertElementsAlmostEqual(eb(2).Belief,eb(1).Input(1));
   
   %Now we create a schedule that will not propagate the info.
   nfg.Schedule = {ib,{f1,eb(1)},{f2,eb(2)},eb(1),eb(2),f1,f2};
   nfg.solve();
   
   assertElementsAlmostEqual(eb(2).Belief,.5);

   %Nest it and see if the schedule is preserved
   b = Bit(2,1);
   fg = FactorGraph();
   g = fg.addFactor(nfg,b);
   
   fg.Schedule = {b(1),b(2),g};
   
   b(1).Input = .8;
   fg.NumIterations = 1;
   fg.solve();
   assertElementsAlmostEqual(b(2).Belief,.5);
\end{lstlisting}





\subsubsection{Modifying an Existing Graph}


\para{Removing a Factor}

It�s possible to remove a Factor from a FactorGraph:

\begin{lstlisting}
fg = FactorGraph();
   b = Bit(3,1);
   fg.addFactor(@xorDelta,b(1:2));
   f = fg.addFactor(@xorDelta,b(2:3));
   b.Input = [.8 .8 .6];
   fg.NumIterations = 2;
   fg.solve();
   assertElementsAlmostEqual([.96 .96 .96]',b.Belief);
   fg.removeFactor(f);
   fg.solve();
   p1 = .8*.8;
   p0 = .2*.2;
   total = p1+p0;
   p1 = p1/total;
   p0 = p0/total;
   assertElementsAlmostEqual([p1 p1 .6]',b.Belief);
\end{lstlisting}

\para{Splitting Variables}

It can be useful to make a copy of a variable and relate it to the old variable with an equals factor. The following code shows how to do this.

\begin{lstlisting}
   a = Bit();
   a.Name = 'a';
   b = Bit();
   b.Name = 'b';
  
   fg = FactorGraph();
  
   f = fg.addFactor(@(x,y) x~=y,a,b);
   f.Name = 'unequal';
   
   b2 = fg.split(b);
   b2.Name = 'b2';
   a2 = fg.split(a,f);
   a2.Name = 'a2';
   
   fg.plot(1);
\end{lstlisting}

We've added code to name all the variables and factors so that the following plot is informative.

\includegraphics{images/SplitGraph.png}
 
Note that the split method takes a list of factors as the second through nth argument. This is the list of factors that will be moved from the original variable to the copied variable. All unspecified factors will remain connected to the initial variable.



\para{Joining Variables}

Users can now join variables. This will create a new joint variable and modify all factors connected to the original variables to reconnect to the new joint variable. This can be useful in eliminating loops in a graph. The following code creates a loopy graph and then uses join to remove the loop.

\begin{lstlisting}
a = Bit(); 
a.Name = 'a'; 
b = Bit();
b.Name  = 'b';
c = Bit();
c.Name = 'c';
d = Bit();
d.Name = 'd';

fg = FactorGraph();
f1 = fg.addFactor(@xorDelta,a,b,c);
f1.Name = 'xor';
f2 = fg.addFactor(@(x,y,z) (x|y)==z ,a,b,d);
f2.Name = 'or';
 
newvar = fg.join(a,b);
newvar.Name = 'a,b';

fg.plot(1);
\end{lstlisting}

The following is the loopy graph:

\includegraphics{images/LoopyGraph.png}
  
And after joining the variables we have:

\includegraphics{images/NonLoopyGraph.png}

\para{Joining Factors}

Users can also join factors. It is possible to remove loops by joining factors as well as by joining variables.

\begin{lstlisting}
   b = Bit(4,1);
   for i = 1:4
       b(i).Name = sprintf('b%d',i);
   end
   fg = FactorGraph();
   f1 = fg.addFactor(@xorDelta,b(1:3));
   f2 = fg.addFactor(@xorDelta,b(2:4));
   
   f3 = fg.join(f1,f2);
   f3.Name = 'twoxors';
   
   b.Input = input;
   fg.solve();
   actualBelief = b.Belief;
   
   fg.plot(1);
\end{lstlisting}

The following plot shows the graph with the loops:  
 
\includegraphics{images/LoopyGraph2.png}

And the following plot shows the graph after the factor is joined: 
 
\includegraphics{images/NonLoopyGraph2.png}

To join factors, Dimple does the following:

\begin{itemize}
\item Find the variables in common between two factors.
\item Take the cartesian product of the tables but discard rows where the common variable indices differ.
\item Consolidate the columns with common variables.
\item Multiply the values for each row.
\end{itemize}



\subsubsection{Explicit Scheduling and Retrieving Message Values}

Dimple supports the ability to retrieve and set messages as well as to explicitly update edges, factors and variables.

\begin{lstlisting}
%OK, first we create a simple Factor Graph with a single xor connecting two  
%variables.
fg = FactorGraph(); 
b = Bit(2,1);
f = fg.addFactor(@xorDelta,b);
%We can go ahead and set some inputs
b(1).Input = .8;
b(2).Input = .7;


%we can examine some edges 
disp(f.Ports{1}.InputMsg);
disp(f.Ports{1}.OutputMsg);

%we can even set some edge messages
f.Ports{1}.InputMsg = [.6 .4];

%we can update a node 
b(1).update();
b(2).update();

%or all the variables in a vector.
b.update();

%or a specific edge
b(1).updateEdge(f);
 
%but updating via portNum is quicker
b(1).updateEdge(1);

%of course, if we don't know the portNum, we can get it
portNum = b(1).getPortNum(f);
b(1).updateEdge(portNum);

%We can do the same kind of stuff with factors
f.updateEdge(b(1));
f.updateEdge(f.getPortNum(b(2)));

%Let's look at some messages again
b(1).Ports{1}.InputMsg
b(2).Ports{1}.InputMsg

%and some beliefs
b.Belief
\end{lstlisting}



\subsubsection{Plotting}


When debugging Factor Graphs, it is sometimes useful to be able to plot a Factor Graph.  The FactorGraph class provides a plot method that can be used to visualize a Factor Graph.

The following code describes how to use the plot function in various ways.

\begin{lstlisting}
%First we build a Factor Graph to use for plotting examples
fg = FactorGraph();
b = Bit(6,1);
for i = 1:6
    b(i).Name = sprintf('b%d',i);
end
 
%We use Label rather than Name for the factors so that we can assign
%them the same Label.  Name must be a unique identifier,
%Label is just used for printing/plotting.
f1 = fg.addFactor(@xorDelta,b(1:4));
f1.Label = 'f';
f2 = fg.addFactor(@xorDelta,b(4:6));
f2.Label = 'f';
 
 
pause_time = 1;
 
%Calling plot with no arguments shows no labels.  It draws variables as
%circles and factors as squares.
fg.plot();
\end{lstlisting}

This will result in the following graph:


\includegraphics{images/plot1.png}

Note that factors are displayed as squares and variables as circles.

\begin{lstlisting}
pause(pause_time);
 
%The following is equivalent to the previous plot command.  We are simply
%explicitly turning off labels.
fg.plot('labels',false);
\end{lstlisting}

Results in the same plot as above.

\begin{lstlisting}
pause(pause_time);
 
%Now we turn on the labels.  Now we see the names we assigned to the
%various nodes and variables of the Factor Graph.
fg.plot('labels',true);
\end{lstlisting}

Results in the following:
 

 
\includegraphics{images/plot2.png}

If the user has specified a Label, those will be displayed, otherwise the object�s Name�s will be displayed.


\begin{lstlisting}
pause(pause_time)
 
%We can specify a subset of nodes to plot
fg.plot('labels',1,'nodes',{b(1:2),f1});
\end{lstlisting}

Results in the following:



\includegraphics{images/plot3.png}

Only the specified nodes and their connectivity were included.

\begin{lstlisting}
pause(pause_time)
 
%By can set a global color for all the nodes in the graph.
fg.plot('labels',1,'color','g');
\end{lstlisting}

Setting a global color:



\includegraphics{images/plot4.png}

\begin{lstlisting}
pause(pause_time)
 
%We can specify a color for one node in the graph.
fg.plot('labels',1,'nodes',{b(1:2),f1},'color',b(1),'g');
\end{lstlisting}

Setting a color for a specific node:
 
\includegraphics{images/plot5.png}
 
\begin{lstlisting}
pause(pause_time)
 
%We can specify colors for multiple nodes in the graph.
fg.plot('labels',1,'nodes',{b(1:2),f1},'color',{b(2),f1},{'r','c'});
\end{lstlisting}

Setting colors for multiple nodes:

\includegraphics{images/plot6.png}

\begin{lstlisting}
pause(pause_time)
 
%We can mix setting a global color, colors for a single node mutliple
%times, and colors for multiple nodes.  The global color is used in all
%cases where a color has not explicitly been set for a node.
fg.plot('labels',1,'color',b(1),'g','color',b(2),'r','color',{b(3),b(4)},{'y','c'},'color','b');
\end{lstlisting}

Mixing and matching the various coloring options:

\includegraphics{images/plot7.png}

\begin{lstlisting}
pause(pause_time)
 
%We can also specify a root node on which we perform a depth first search
%up to a specific depth and then only print nodes up to that depth.
for depth = 0:5
 
    %Furthermore, we color the root node green so we know which is the root
    %node.
    fg.plot('labels',1,'depth',b(1),depth,'color',b(1),'g','color','b');
    
    pause(pause_time);
end
\end{lstlisting}

Specifying a depth will display a specified root node and all nodes that are N steps away.  The following shows the result of calling plot with �depth� of b(1) and 2:
 

\includegraphics{images/plot8.png}

We also colored the root node green to make it clear which was the root node.

\begin{lstlisting}
%The following shows how using the depth feature we might be able to find
%out interesting information.  Here we increase the depth until we visually
%see a loop.
[ldpc,vars] = createLDPC();
v = vars(1);
 
for depth = 0:6
    ldpc.plot('depth',v,depth);
    
    pause(pause_time);
end
\end{lstlisting}

Here, we show how we can use plot to find interesting information about a large graph.  The final plot with a depth of 6 shows a cycle in an LDPC Factor Graph:

\includegraphics{images/plot9.png}

 
\para{Nesting}

By default, the plotting method ignores hierarchy and plots the flattened graph.  If the user specifies the 'nesting' parameter, however, they can specify how deep to descend into the hierarchy before considering NestedGraphs to be Factors and plotting them as such.

When labels are turned off, nested graphs are displayed as triangles

First let's build a graph with three levels of nesting.

\begin{lstlisting}
b = Bit(2,1);
template1 = FactorGraph(b);
iv = Bit();
template1.addFactor(@xorDelta,b(1),iv);
template1.addFactor(@xorDelta,b(2),iv);
 
b = Bit(2,1);
template2 = FactorGraph(b);
iv = Bit();
template2.addFactor(template1,b(1),iv);
template2.addFactor(template1,b(2),iv);
 
template2.plot();
 
b = Bit(2,1);
fg = FactorGraph(b);
iv = Bit();
fg.addFactor(template2,b(1),iv);
fg.addFactor(template2,b(2),iv);
\end{lstlisting}
 
Here we show the graph plotted with various levels of nesting.

fg.plot('nesting',0);

\includegraphics{images/plot10.png}


Notice the Nested Graphs show up as triangles.

\begin{lstlisting}
fg.plot('nesting',1);
\end{lstlisting}

\includegraphics{images/plot11.png}
 
\begin{lstlisting}
fg.plot('nesting',2);
\end{lstlisting}


\includegraphics{images/plot12.png}
 
Note that once we�ve reached the bottom, we�re actually seeing the Factors plotted.
 
We can retrieve an instance of a nested graph and plot that with nesting set.

\begin{lstlisting}
fg.NestedGraphs{1}.plot('nesting',0);
pause(pause_time);
\end{lstlisting}



\includegraphics{images/plot13.png}
 
 
When plotting graphs, boundary variables show up as stars.

Next we mix depth first search with nesting


\begin{lstlisting}
fg.plot('nesting',0,'depth',iv,0);
\end{lstlisting}

\includegraphics{images/plot14.png}
 

 
\begin{lstlisting}
fg.plot('nesting',0,'depth',iv,1);
\end{lstlisting}

\includegraphics{images/plot15.png}
 

 
\begin{lstlisting}
fg.plot('nesting',0,'depth',iv,2);
\end{lstlisting}


\includegraphics{images/plot16.png}




\subsubsection{Energy}

\para{Bethe Free Energy}

\[
BetheFreeEnergy = InternalEnergy - BetheEntropy
\]

\begin{lstlisting}
bfe = fg.BetheFreeEnergy;
\end{lstlisting}

\para{Internal Energy}

\[
 InternalEnergy = \sum_{a \in F }InternalEnergy(a) + \sum_{i \in V}InternalEnergy(i) 
 \]

Where F is the set of all Factors and V is the set of all variables.  If Dimple treated inputs as single node Factors, this method would only sum over factors.

\begin{lstlisting}
ie = fg.InternalEnergy;
\end{lstlisting}

\para{Bethe Entropy}

\[
 BetheEntropy = \sum_{a \in F}BetheEntropy(a) - \sum_{i \in V}BetheEntropy(i)(degree(i)-1) 
 \]

Where F is the set of all Factors and V is the set of all variables.

\begin{lstlisting}
be = fg.BetheEntropy;
\end{lstlisting}

\para{Factor Internal Energy}

\[
InternalEnergy(a) = \sum_{\vec{x} \in \vec{X}}B_a(\vec{x})*(-log(Weight(\vec{x}))) 
\]

Where a is an instance of a Factor, X is the set of variables connected to a, Weight is the FactorTable entry for the specified set of variable values, and B~a~ is the belief of that factor node.

\begin{lstlisting}
ie = f.InternalEnergy;
\end{lstlisting}

\para{Factor Bethe Entropy}

\[
BetheEntropy(a) = - \sum_{\vec{x} \in domain(\vec{X})}B_a(\vec{x})*log(B_a(\vec{x})) 
\]

Where a is an instance of a Factor, X is the set of variables connected to a, and B~a~ is the belief of that factor node.

\begin{lstlisting}
be = f.BetheEntropy;
\end{lstlisting}

\para{Variable Internal Energy}

\[
 InternalEnergy(i) = \sum_{d \in D}B_i(d)*(-log(Input(d))) 
 \]

Where D is variable i's domain and Input is the variable's input.

\begin{lstlisting}
ie = v.InternalEnergy;
\end{lstlisting}

\para{Variable Bethe Entropy}

\[
BentheEntropy(i) = - \sum_{d \in D}B_i(d)*log(B_i(d)) 
\]

Where D is variable i's domain.

\begin{lstlisting}
be = v.BetheEntropy();
\end{lstlisting}

Factor Belief

To support the Bethe Free Energy property, Dimple provides getBelief associated with a Factor.

\[
B_a(\vec{x}) = \frac{Weight(\vec{x})\prod_{i=0}^N \mu_{X_i \rightarrow a}(x_i)}{\sum_{\vec{x} \in \vec{X}}Weight(\vec{x})\prod_{i=0}^N \mu_{X_i \rightarrow a}(x_i\
)} 
\]

Where $ \vec{x} \in domain(\vec{X}) $ and $ \vec{X} $ is the set of variables connected to the factor a.

\begin{lstlisting}
b = f.Belief;
\end{lstlisting}

\para{Score}
\label{sec:score}

Users can retrieve the Energy of a Factor Graph, Variable Vector, or Factor. This feature is currently only supported with the SumProduct algorithm.

The following are examples of ways to retrieve the energy of FactorGraphs, factors, and variables:

\begin{lstlisting}
fg = FactorGraph();
b = Bit(3,1);
f = fg.addFactor(@xorDelta,b);

fg.Score
f.Score
b.Score
b(1).Score
\end{lstlisting}

When computing the score, the most likely variable value is used. It's possible to override that by setting a guess on a variable using:

\begin{lstlisting}
b.Guess = 1; %For other variable types, var.Guess takes a domain item.
\end{lstlisting}

\para{Math behind the score}

The variable energy is normalized by the maximum input probability.

\[
score(v) = -log(inputProb(guess) + log(max(inputProb)) 
\]

The factor energy is normalized by the largest factor value.

\[
score(f) = - log(f(guessxguessyguessz))+log(max(f)) 
\]

And the Factor Graph�s energy is as follows:

\[
score(fg) = \sum_{v \in V} score(v) + \sum_{f \in F} score(f) 
\]


\subsubsection{Introspection}

The FactorGraph class provides several feature for inspecting aspects of the graph.  The ability to nest graphs complicates things a bit.  Nested FactorGraphs can be considered Factors.  All of the introspection features allow the user to view nested graphs as leaf factors or to descend into them and operate on the children of the nested graphs.  Each feature provides several methods:

\begin{itemize}
\item $<$FeatureName$>$(int relativeNestingDepth) -- The relativeNestingDepth specifies how deep to descend into the nested FactorGraphs before treating deeper NestedGraphs as Factors.  Specifying 0 will treat the top level nested Graphs as factors.  Specifying a large enough number will descend all the way to the leaf factors.  Specifying something between 0 and the FactorGraph�s maximum depth will descend as far as this parameter specifies before considering NestedGraphs to be factors.  The parameter contains the word �relative� because users can retrieve nested graphs.  They can call one of the feature�s methods on that nested graph.  
\item $<$FeatureName$>$Flat() -- equivalent of $<$FeatureName$>$(max int)
\item $<$FeatureName$>$Top() -- equivalent of $<$FeatureName$>$(0)
\item $<$FeatureName$>$() -- equivalent of $<$FeatureName$>$Flat().  It was thought that users will most often want to operate on the FactorGraph in its flattened form.
\end{itemize}

Now, on to the specific features.

\para{Retrieving All Factors}

Users can retrieve Factors and/or NestedGraphs associated with a graph using the Factors methods and properties:

\begin{itemize}
\item Fg.Factors
\item Fg.FactorsFlat
\item Fg.FactorsTop
\item Fg.getFactors(relativeNestingDepth)
\end{itemize}

When the user specifies a relativeNestingDepth or calls FactorsTop, the resulting cell array will contain a mix of leaf factors and Nested Graphs.

\para{Retrieving Factors but not Nested Factor Graphs}

The FactorGraph class provides the following:

\begin{itemize}
\item NonFactorGraphFactors
\item NonFactorGraphFactorsFlat
\item NonFactorGraphFactorsTop
\item getNonFactorGraphFactors(relativeNestingDepth)
\end{itemize}

As the name implies, this will behave similar to the Factors properties and methods but will exclude nested graphs.

\para{Retrieving Variables}

The FactorGraph class provides the following:

\begin{itemize}
\item Variables -- calls VariablesFlat
\item VariablesFlat -- Returns a list of all the Variables in the graph, including those contained by nested graphs.
\item VariablesTop -- Returns only those variables contained in the top level of the graph.
\item getVariables(relativeNestingDepth,forceIncludeBoundaryVariables) -- Returns all variables contained in the FactorGraph from which the method is called as Variables that are as deep as the specified relativeNestingDepth.  The second parameter is optional and defaults to false.  When false, boundary variables are only included by the root graph.  When true, boundary variables are included regardless of whether a graph is a root or nested graph.
\end{itemize}

\para{Retrieving all Nodes}

The FactorGraph provides the following:

\begin{itemize}
\item Nodes
\item NodesFlat
\item NodesTop
\item getNodes(relativeNestingDepth,forceIncludeBoundaryVariables)
\end{itemize}

These methods call the Factor and Variable methods and concatenate the results together.

\para{Determining if a FactorGraph is a tree}

The FactorGraph class provides the following:

\begin{itemize}
\item isTree(relativeNestingDepth) 
\item isTreeTop 
\item isTreeFlat 
\end{itemize}


isTree -- Users can call <factor graph name>.isTree() to determine if a FactorGraph is a tree.  If the graph contains cycles, this method will return false. Like the other methods, the relativeNestingDepth determines at what point to consider NestedGraphs to be leaf cells.  

\para{Retrieving an Adjacency Matrix}

All of the following methods return a pair: [A, labels] where A is a square connectivity matrix and labels is a cell array of strings specifying the names of the nodes in A. 

\begin{itemize}
\item getAdjacencyMatrix(relativeNestingDepth,forceIncludeBoundaryVariables) -- relativeNestingDepth behaves the same as in other methods that take this parameter.  So does forceIncludeBoundaryVariables. forceIncludeBoundaryVariables has a default value of false.
\item getAdjacencyMatrix(nodes,forceIncludeBoundaryVariables) -- Users can specify a specific subset of nodes in which they�re interested.  This method will return an adjacency matrix with only those nodes.  Nodes are considered connected only if there is an edge directly connecting them.
\item getAdjacencyMatrixTop() -- equivalent to getAdjacencyMatrix(0,false)
\item getAdjacencyMatrixFlat() -- equivalent to getAdjacencyMatrix(intmax,false)
\end{itemize}

FactorGraph also provides an AdjacencyMatrix Property:

\begin{itemize}
\item AdjacencyMatrix -- equivalent to getAdjacencyMatrixFlat and only returns A (not the labels).  MATLAB properties can only return one object.
\end{itemize}

An example of getAdjacencyMatrix:

\begin{lstlisting}
fg = FactorGraph();
b = Bit(2,1);
b(1).Name = �b1�;
b(2).Name = �b2�;
f = fg.addFactor(@xorDelta,b);
f.Name = �f�;
[A,labels] = fg.getAdjacencyMatrix();
A =

     0     0     1
     0     0     1
     1     1     0


labels = 

    'b1'
    'b2'
    'f'
\end{lstlisting}

\para{Depth first search}

\begin{itemize}
\item depthFirstSearch(node, searchDepth, relativeNestingDepth) -- 
\begin{itemize}
\item node -- Specifies the node from which to initiate the search
\item searchDepth -- specifies how far from �node� the search should go.
\item relativeNestingDepth -- determines how deep to go down the NestedGraphs before considering NestedGraphs to be leaf cells. 
\end{itemize}
\item depthFirstSearchFlat(node, searchDepth) -- equivalent of depthFirstSearch(node,searchDepth,maxint)
\item depthFirstSearchThop(node, searchDepth) -- equivalent of depthFirstSearch(node,searchDepth,0)
\end{itemize}

An example:

\begin{lstlisting}
fg = FactorGraph();
b = Bit(6,1);
for i = 1:6
b(i).Name = sprintf('b%d',i);
end
f1 = fg.addFactor(@xorDelta,b(1:4));
f1.Name = 'f1';
f2 = fg.addFactor(@xorDelta,b(4:6));
f2.Name = 'f2';
 
nodes = fg.depthFirstSearch(b(1),3);
\end{lstlisting}

calling fg.plot(�color�,b(1),�g�,�labels�,true) reveals the following structure of this graph

\includegraphics{images/Introspection.png}
 
As you might guess fg.depthFirstSearch(b(1),3) will return a collection of six nodes: b1, f1, b2, b3, b4, and f2.  It will not include b5 and b6 since those are at a depth of four from b1.

\subsubsection{Estimating Parameters}

Dimple provides support for estimating parameters.  Only the sumproduct solver currently supports this. 
The sum product algorithm computes the gradient of the BetheFreeEnergy and performs a gradient descent on this.  Users can
specify how many times to restart the gradient descent with random table values.

Users provide four arguments:

\begin{itemize}
\item factorsAndFactorTables - A cell array of FactorTables for which the user wants parameters to be estimated.  They can mix Factors and FactorTables
\item numRestarts - The number of times to randomly initialize the FactorTables and restart the gradient descent.
\item numSteps - The number of times to descend down along the gradient
\item stepScaleFactor - The value we scale the gradient by.
\end{itemize}

\begin{lstlisting}
fg.estimateParameters({factorTable1,factorTable2,factor1,etc...},numRestarts,numSteps,stepScaleFactor);
\end{lstlisting}

\subsubsection{Invoking Methods on the Underlying Solver Objects}

Users sometimes need access to the underlying solver objects in order to set and get
solver specific properties.  For instance K-best is implemented by minsum and sumproduct 
solvers.  

Let's use the following code to show examples of accessing underlying solver objects

\begin{lstlisting}
fg = FactorGraph();
d = Discrete(1:10,10,1);
ft = FactorTable(rand(10),d.Domain,d.Domain);
ft.normalize(1);
f = fg.addFactorVectorized(ft,d(1:(end-1)),d(2:end));
f.DirectedTo = d(2:end);
\end{lstlisting}

This code constructs a Markov Model of 10 variables with domain 1 through 10 and a randomized transition matrix.  Suppose we want to use K-best for all the factors.  We could access the underlying solver object of each factor and call setK on that object:

\begin{lstlisting}
for i = 1:length(f)
   f(i).Solver.setK(5);
end
\end{lstlisting}

However, in the case of very large graphs, this can be very inefficient.  MATLAB has a fairly large penalty for invoking function calls in a large for loop.  Instead, variables, factors, and factor graphs all have an invokeSolverMethod that can be used to call a method on all objects in the vector.

\begin{lstlisting}
f.invokeSolverMethod('setK',uint32(5));
\end{lstlisting}

After this is invoked, all factors will have K set to 5.  There are two issues with the current implementation that should be fixed at some point.  The first is that it currently only supports providing the same values to all solver objects.  If you wanted to set a different K for all factors, you could not use this mechanism.  The second issue is that, users must be careful to correctly cast their arguments to the appropriate types.  This could and should be fixed.

It's also possible to call methods with return values as follows:

\begin{lstlisting}
d.invokeSolverMethod('setDamping',uint32(0),2.0);
damping = d.invokeSolverMethodWithReturnValue('getDamping',uint32(0));

damping

>> damping

damping = 

    [2]
    [2]
    [2]
    [2]
    [2]
    [2]
    [2]
    [2]
    [2]
    [2]

\end{lstlisting}

Here we set the damping value for the first edge of every variable.





\subsection{Variables and Related Classes}

\subsubsection{Variable Types}

The following variable types are defined in Dimple.  Some variable types are supported by only a subset of solvers.  The following table lists the Dimple variable types and the solvers that support them.

\begin{longtable} {l | p{5cm}}
Variable Type & Supported Solvers \\
\hline
\endhead
Discrete & all \\
Bit & all \\
Real & Gibbs, Gaussian, ParticleBP \\
RealJoint & Gaussian \\ 
\ifmatlab ComplexVar & Gaussian \\ \fi
FiniteFieldVariable & SumProduct \\
\end{longtable} 

\subsubsection{Common Properties and Methods}

The following properties and methods are common to variables of all types.

\para{Properties}

\subpara{Name}

Read-write.  When read, retrieves the current name of the variable or array of variables.  When set, modifies the name of the variable to the corresponding value.  The value set must be a string.

\ifmatlab
\begin{lstlisting}
var.Name = 'string';
\end{lstlisting}

When setting the Name, only one variable in an array may be set at a time.  To set the names of an entire array of variables to distinct values, the setNames method may be used (see section~\ref{sec:Variable.setNames}).

\fi

\ifjava
\begin{lstlisting}
var.setName('string');
\end{lstlisting}

\fi

\subpara{Label}

Read-write. All variables and factors in a Factor Graph must have unique names. However, sometimes it�s desirable to have variables or factors share similar strings when being plotted or printed. Users can set the Label property to set the name for display. If the Label is not set, the Name will be used for display. Once the label is set, the label will be used for display.

\ifmatlab
\begin{lstlisting}
var.Label = 'string';
\end{lstlisting}
\fi

\ifjava
\begin{lstlisting}
var.setLabel('string');
\end{lstlisting}
\fi


\subpara{Solver}

Read-only.  Returns the solver-object associated with the variable, to which solver-specific methods can be called.  See section~\ref{sec:SolversAPI}, which describes the solvers, including the solver-specific methods for each solver.

\subpara{Guess}
\label{sec:Variable.Guess}

Read-write.  Specifies a value from the variable to be used when computing the Score of the factor graph (or of the variable or neighboring factors).  The Guess must be a valid value from the domain of the variable.

If the Guess had not yet been set, its value defaults to the most likely belief (which corresponds to the Value property of the variable)\footnote{For some solvers, beliefs are not supported for all variable types; in such cases there is no default value, so a Guess must be specified.}.

\subpara{Score}
\label{sec:Variable.Score}

Read-only.  When read, computes and returns the score (energy) of the Input to this variable, which is treated as a single-edge factors, given a specified value for the variable.  The score value is relative, and may be arbitrarily normalized by an additive constant.

The value of the variable used when computing the Score is the Guess value for this variable (see section~\ref{sec:Variable.Guess}).  If no Guess had yet been specified, the value with the most likely belief (which corresponds to the Value property of the variable) is used\footnote{For some solvers, beliefs are not supported for all variable types; in such cases there is no default value, so a Guess must be specified.}.


\subpara{Ports}

Read-only.  Retrieves a cell array containing a list of Ports connecting the variable to its neighboring factors.

\ifmatlab
\para{Methods}
\fi

\ifmatlab
\subpara{setNames}
\label{sec:Variable.setNames}

For an array of variables, the setNames method sets the name of each variable in the array to a distinct value derived from the supplied string argument.  When called with:

\begin{lstlisting}
varArray.setName('baseName');
\end{lstlisting}

the resulting variable names are of the form: \texttt{baseName\textunderscore vv0}, \texttt{baseName\textunderscore vv1}, \texttt{baseName\textunderscore vv2}, etc., where each variable's name is the concatenation of the base name with the suffix \texttt{\textunderscore vv} followed by a unique number for each variable in the array.
\fi

\ifmatlab
\subpara{invokeSolverSpecificMethod}

\begin{lstlisting}
variableArray.invokeSolverSpecificMethod('methodName', arguments);
\end{lstlisting}

For an array of variables, the invokeSolverSpecificMethod calls the specified solver-specific method on each of the variables in the array.  Here, 'methodName' is a text string with the name of the solver-specific method, and arguments is an optional comma-separated list of arguments to that method.  This method does not result in any return values.  For solver specific methods that return results, use the invokeSolverSpecificMethodWithReturnValue method instead.

\fi

\ifmatlab
\subpara{invokeSolverSpecificMethodWithReturnValue}

\begin{lstlisting}
returnArray = variableArray.invokeSolverSpecificMethodWithReturnValue('methodName', arguments);
\end{lstlisting}

For an array of variables, the invokeSolverSpecificMethodWithReturnValue calls the specified solver-specific method on each of the variables in the array, returning a return value for each variable.  Here, 'methodName' is a text string with the name of the solver-specific method, and arguments is an optional comma-separated list of arguments to that method.  The returnArray is a cell-array of the return values of the method, with dimensions equal to the dimensions of the variable array.

\fi

\ifmatlab
\para{Operators}

\subpara{Operators for Implicit Factor Creation}

Dimple supports a set of overloaded MATLAB operators and functions that operate on variables to implicitly add factors to a factor graph.

The list of supported operators and functions is given in section~\ref{sec:overloaded}.  A description of how to use these operators and functions is given in section~\ref{sec:ImplicitFactorCreation}.

Using one of the defined operators or functions with one or more variables will result in creation of a new variable, which can be assigned to a new variable name, with the appropriate domain.  It will also result in the creation of a factor which is added to the most recently created factor graph.  This factor will connect to the input variables and the newly created result variable.  For example:

\begin{lstlisting}
c = a + b;
\end{lstlisting}

This results in the creation of a Sum factor with connected variables, c, a, and b (in that order).

These operators and functions can be compounded into a single line of code, resulting in creation of intermediate anonymous variables.  For example:

\begin{lstlisting}
z = (a + b) * c^d - sqrt(-e);
\end{lstlisting}

Like using the addFactor method, some of the inputs to these operators and functions may be constants.  Specifically, for binary operators, one of the inputs may be a constant instead of a variable.  For example:

\begin{lstlisting}
x = a^2;
y = (a + b + 2) * 3;
\end{lstlisting}

These operators and functions can be applied to arrays of variables, with some limitations.  Specifically, if each of the input variables are vectors of the same dimension, then the result will be to create a vector of output variables of the same dimension, along with a vector of factors relating the inputs and outputs.

In some cases, to be consistent with MATLAB notation, there is a distinction made between the vectorized and non-vectorized operator.  Specifically, Dimple uses MATLAB's notation for pointwise product and power operators to indicate a vectorized operation.  For example, if variables a through e are vectors of variables of identical size, then the following would create a variable vector z, and a series of factors relating these variables.

\begin{lstlisting}
z = (a .* b) + c.^d - sqrt(-e);
\end{lstlisting}

For binary operators, one of the inputs may be a scalar variable or a scalar constant instead of a variable vector.  For a scalar variable, the result is that scalar variable connecting to each instance of the factors that are created.  For a constant, each instance of the factor uses the same constant for that input (vectors of distinct constants are not currently supported).

\subpara{repmat}

Dimple overloads the MATLAB repmat function when called with a Dimple variable or variable array as its first argument.  The result of this function is an array of variables in the requested dimensions.  Dimple does not actually make multiple copies of the variables, but instead creates a variable array that provides repeated references to the same variables in the existing array.  

For example:
\begin{lstlisting}
var = Bit(10, 1);
varRef = repmat(var, 1, 10);
\end{lstlisting}

In this case varRef is a 10x10 array of variables.  Each element of varRef, however, is not a distinct Dimple variable, but a reference to an element in var, where for each row of varRef, there are 10 repeated copies of the corresponding variable in var.  Use of repmat for variables can be useful when using addFactorVectorized (see sections~\ref{sec:vectorizedFactorCreation} and~\ref{sec:FactorGraph.addFactorVectorized}).

\fi

\subsubsection{Discrete}
\label{sec:Discrete}

A Discrete variable represents a variable that can take on a finite set of distinct states.  The Discrete class corresponds to either a single Discrete variable or a multidimensional array of Discrete variables.  All properties/methods can either be called for all elements in the collection or for individual elements of the collection.

\para{Constructor}

The Discrete constructor can be used to create an N-dimensional collection of Dimple Discrete variables.  The constructor is called with the following arguments (arguments in brackets are optional).

\ifmatlab
\begin{lstlisting}
Discrete(domain, [dimensions])
\end{lstlisting}
\fi

\ifjava
\begin{lstlisting}
Discrete(domain)
\end{lstlisting}
\fi

\ifmatlab
\begin{itemize}
\item domain is a required argument indicating the domain of the variable.  The domain may either be a numeric array of domain elements, a cell array of domain elements, or a DiscreteDomain object (see section~\ref{Discrete.Domain}).
\item dimensions is an optional variable-length comma-separated list of matrix dimensions (an empty list indicates a single Discrete variable).
\end{itemize}
\fi

\ifjava
\begin{itemize}
\item domain is a required argument indicating the domain of the variable.  The domain may either be an object array of domain elements, a comma separated list, or a DiscreteDomain object (see section~\ref{Discrete.Domain}).
\end{itemize}
\fi

For example:

\ifmatlab
\begin{lstlisting}
domain = [0 1 2];
w = Discrete(domain);
x = Discrete(domain, 4);
y = Discrete(domain, 2, 3);
z = Discrete(domain, 2, 3, 4);
\end{lstlisting}
\fi

\ifjava
\begin{lstlisting}
Object [] domain = new Object [] {0,1,2};
Discrete w = new Discrete(domain);
Discrete x = new Discrete(0,1,2);
DiscreteDomain dd = new DiscreteDomain(0,1,2);
Discrete y = new Discrete(dd);
\end{lstlisting}
\fi

We examine each of these arguments in more detail in the following sections.

\subpara{Domain}
\label{Discrete.Domain}

Every Discrete random variable has a domain associated with it.  A domain is a set.  Elements of the set may be any object type.  For example, the following are Discrete variables with valid domains:

\ifmatlab
\begin{lstlisting}
a = Discrete({1, 2, 3});
b = Discrete({1+i, i, 2*i});
c = Discrete({[1 0; 0 1], [i 1, 2*i 1]});
d = Discrete({[1 0; 0 1], 2, i+1});
e = Discrete({1.2, 3, pi/2});
f = Discrete({'red', 'green', 'blue'});
\end{lstlisting}
\fi

\ifjava
\begin{lstlisting}
Discrete a = new Discrete(1, 2, 3);
Discrete b = new Discrete(new double [] {1,0,0,1},1); 
Discrete c = new Discrete("red","blue",2);
\end{lstlisting}
\fi

\ifmatlab
(a) creates a variable whose domain consists of three values: 1, 2, and 3.  (b) creates a variable whose domain consists of three complex numbers.  (c) creates a variable whose domain consists of two elements, each of which is a 2x2 complex matrix.  (d) creates a variable whose domain consists of three elements: a matrix, real scalar, and complex scalar.  (e) creates a variable whose domain consists of both floating-point and integer values.  (f) creates a variable whose domain is a set of strings.

In the previous example we used cell arrays to specify the elements of a domain.  When the domain consists only of numeric values (integer or floating-point), domains can instead be specified as a numeric array.  In this case, each element of the array (regardless of the array's dimensions) is considered a distinct entry in the domain.

\begin{lstlisting}
a = Discrete(0:2);
b = Discrete([1 2 3; 4 5 6]);
c = Discrete([0:2]�);
\end{lstlisting}

(a) creates a variable with a domain of 0, 1, and 2.  (b) creates a variable with a domain of 1, 2, 3, 4, 5, 6.  (c) creates a variable with domain of 0, 1, and 2.

\fi

The domain may also be specified using a DiscreteDomain object.  In that case, the domain of the variable consists of the elements of this object.  For example:

\ifmatlab
\begin{lstlisting}
myDomain = DiscreteDomain(0:10);
a = Discrete(myDomain);
\end{lstlisting}
\fi

\ifjava
\begin{lstlisting}
DiscreteDomain myDomain = new DiscreteDomain(0,1,2);
Discrete a = new Discrete(myDomain);
\end{lstlisting}
\fi

See section~\ref{sec:DiscreteDomain} for more information about the DiscreteDomain class.

\ifmatlab
\subpara{List of Matrix Dimensions}
\label{sec:VariableMatrixDimensions}

If the variable constructor is called without any dimensions, a single variable will be created.

If one dimension n is specified, a square array of dimensions n x n variables will be created\footnote{This follows a common MATLAB convention.}.

With k dimensions specified, n1, n2, �, nk, a multidimensional variable array of dimensions n1 x n2 x ... x nk will be created.

\fi

\para{Properties}

\subpara{Domain}
\label{sec:Discrete.Domain}

Read-only.  The Domain property returns the domain of that variable in the form of a DiscreteDomain object.

\subpara{Belief}
\label{sec:Discrete.Belief}

Read-only.  For any single variable, the Belief method returns a vector whose length is the total number of elements of the domain of the variable.  When called after running a solver to perform inference on the graph, each element of the vector contains the estimated marginal probability of the corresponding element of the domain of the variable.  The results are undefined if called prior to running a solver.

For an array of variables, the Belief method will return an array of vectors (that is, an array one dimension larger than the variable array) containing the beliefs of each variable in the array.

\subpara{Value}
\label{sec:Discrete.Value}

Read-only.  In some cases, one may wish to retrieve the single most likely element of a variable's domain.  The Value property does just that.

For any single variable, the Value method returns a single value chosen from the domain of the variable.  When called after running a solver to perform inference on the graph, the value returned corresponds to the element in the variable's domain that has the largest estimated marginal probability\footnote{If more than one domain element has identical marginal probabilities that are larger than for any other value, a single value from the domain is returned, chosen arbitrarily among these.}.  The results are undefined if called prior to running a solver.

For an array of variables, the Value method will return an array of values, each from the domain of the corresponding variable representing the largest estimated marginal probability for that variable.

\subpara{Input}
\label{sec:Discrete.Input}

Read-write.  For any variable, the Input method can be used to set and return the current input of that variable. An input behaves much like a single edge factor connected to the variable, and is typically used the represent the likelihood function associated with a measured value (see section~\ref{sec:LikelihoodInput}).

When read, for a single variable returns an array of values with each value representing the current input setting for the corresponding element of the variable's domain.  The length of this array is equal to the total number of elements of the domain.  When read, for an array of variables, the result is an array with dimension one larger than the dimension of the variable array.  The additional dimension represents the current set of input values for the corresponding variable in the array.

When written, for a single variable, the value must be an array of length equal to the domain of the variable.  The values in the array must all be non-negative, and non-infinite, but are otherwise arbitrary.  When written, for an array of variables, the values must be a multidimensional array where the first set of dimensions exactly match the dimensions of the array (or the portion of the array) being set, and length of the last dimension is the number of elements in the variable's domain.


\subpara{FixedValue}
\label{sec:Discrete.FixedValue}

Read-write.  For any variable, the FixedValue method can be used to set the variable to a specific fixed value, and to retrieve the fixed-value if one has been set.  This would generally be used for conditioning a graph on known data without modifying the graph (see section~\ref{sec:FixingAVariableValue}).

Reading this property results in an error if no fixed value has been set.  To determine if a fixed value has been set, use the hasFixedValue method (see section~\ref{sec:Discrete.hasFixedValue}).

When setting this property on a single variable, the value must be a value included in the domain of the variable.
The fixed value must be a value chosen from the domain of the variable.  For example:

\ifmatlab
\begin{lstlisting}
a = Discrete(1:10);
a.FixedValue = 3;
\end{lstlisting}
\fi

\ifjava
\begin{lstlisting}
Discrete a = new Discrete(1,2,3);
a.setFixedValue(3);
\end{lstlisting}
\fi

When setting this property on a variable array, the value must be an array of the same dimensions as the variable array, and each entry in the array must be an element of the domain.

Because the Input and FixedValue properties serve similar purposes, setting one of these overrides any previous use of the other.  Setting the Input property removes any fixed value and setting the FixedValue property replaces the input with a delta function---the value 0 except in the position corresponding to the fixed value that had been set.


\para{Methods}

\subpara{hasFixedValue}
\label{sec:Discrete.hasFixedValue}

This method takes no arguments.  When called for a single variable, it returns a boolean indicating whether or not a fixed-value is currently set for this variable.  When called for a variable array, it returns a boolean array of dimensions equal to the size of the variable array, where each entry indicates whether a fixed value is set for the corresponding variable.

\subsubsection{Bit}

A Bit is a special kind of Discrete with domain [0 1].

\para{Constructor}


\ifmatlab
The Bit constructor can be used to create an N-dimensional collection of Dimple Discrete variables.  Its constructor does not require a domain, since the domain is predetermined.  The constructor takes only a variable-length list of matrix dimensions, where an empty list indicates a single Bit variable.

\begin{lstlisting}
Bit([dimensions])
\end{lstlisting}

The behavior of the list of dimensions is identical to that for Discrete variables as described in section~\ref{sec:VariableMatrixDimensions}.

\fi

\ifjava
\begin{lstlisting}
Bit()
\end{lstlisting}
\fi

\para{Properties}

\subpara{Domain}

See section~\ref{sec:Discrete.Domain}.

\subpara{Belief}

Read-only.  For a single Bit variable, the Belief property is a single number that represents the estimated marginal probability of the value one.

For an array of Bit variables, the Belief property is an array of numbers with size equal to the size of the variable array, with each value representing the estimated marginal probability of one for the corresponding variable.

\subpara{Value}

See section~\ref{sec:Discrete.Value}.

\subpara{Input}

Read-write.  For setting the Input property on a single Bit variable, the value must be a single number in the range 0 to 1, which represents the normalized likelihood of the value one (see section~\ref{sec:LikelihoodInput}).  If $L(x)$ is the likelihood of the variable, the Input should be set to $\frac{L(x=1)}{L(x=0) + L(x=1)}$.

For setting the Input property on an array of Bit variables, the value must be an array of normalized likelihood values, where the array dimensions must match the dimensions of the array (or the portion of the array) being set.


\subpara{FixedValue}

See section~\ref{sec:Discrete.FixedValue}.

\para{Methods}

\subpara{hasFixedValue}

See section~\ref{sec:Discrete.hasFixedValue}.


\subsubsection{Real}

A Real variable represents a variable that takes values on the real line, or on a contiguous subset of the real line.  \ifmatlab The Real class corresponds to either a single Real variable or a multidimensional array of Real variables.  All properties/methods can either be called for all elements in the collection or for individual elements of the collection.  \fi

\para{Constructor}

\ifmatlab
\begin{lstlisting}
Real([domain], [dimensions])
\end{lstlisting}
\fi

\ifjava
\begin{lstlisting}
Real(RealDomain)
Real()
Real(lowerBound,upperBound)
\end{lstlisting}
\fi

\ifmatlab
All arguments are optional and can be used in any combination.

\begin{itemize}
\item domain is specifies a bound on the domain of the variable. It can either be specified as a two-element array or a RealDomain object (see section~\ref{sec:RealDomain}).  If specified as an array, the first element is the lower bound and the second element is the upper bound. -Inf and Inf are allowed values for the lower or upper bound, respectively.  If no domain is specified, then a domain from $-\infty$ to $\infty$ is assumed.
\item dimensions specify the array dimensions.  The behavior of the list of dimensions is identical to that for Discrete variables as described in section~\ref{sec:VariableMatrixDimensions}.
\end{itemize}
\fi

\ifjava
\begin{itemize}
\item domain is specifies a bound on the domain of the variable. It can either be specified as two elements or a RealDomain object (see section~\ref{sec:RealDomain}).  If specified as two values, the first element is the lower bound and the second element is the upper bound. -Inf and Inf are allowed values for the lower or upper bound, respectively.  If no domain is specified, then a domain from $-\infty$ to $\infty$ is assumed.
\end{itemize}
\fi

Examples:

\ifmatlab
\begin{itemize}
\item Real() specifies a scalar real variable with an unbounded domain.
\item Real(4,1) specifies a 4x1 vector of real variables with unbounded domain.
\item Real([-1 1]) specifies a scalar real variable with domain from -1 to 1.
\item Real([-Inf 0], 4, 10, 2) specifies a 4x10x2 array of real variables, each with the domain from negative infinity to zero.
\item Real(RealDomain(-pi, pi)) specifies a scalar real variable with domain from $-\pi$ to $\pi$.
\end{itemize}
\fi

\ifjava
\begin{itemize}
\item Real() specifies a scalar real variable with an unbounded domain.
\item Real(-1, 1) specifies a scalar real variable with domain from -1 to 1.
\item Real(Double.NEGATIVE\_INFINITY,0) specifies a variable with the domain from negative infinity to zero.
\item Real(RealDomain(-Math.PI, Math.PI)) specifies a scalar real variable with domain from $-\pi$ to $\pi$.
\end{itemize}
\fi


\para{Properties}

\subpara{Domain}

Read-only.  The Domain property returns the domain of that variable in the form of a RealDomain object (see section~\ref{sec:RealDomain}).

\subpara{Belief}

Read-only.  The behavior of this property for Real variables is solver specific.  Some solvers do not support this property at all and will return an error when read.  See section~\ref{sec:SolversAPI} for more detail on each of the supported solvers.


\subpara{Value}

Read-only.  The behavior of this property for Real variables is solver specific.  Some solvers do not support this property at all and will return an error when read.  See section~\ref{sec:SolversAPI} for more detail on each of the supported solvers.


\subpara{Input}

Read-write.  For a Real variable, the form of the Input property depends on the particular solver.  For the Gibbs and ParticleBP solvers, the Input property must be a built-in factor function, as described in section~\ref{sec:usingBuiltInFactors}.  Typically, it would be one of the standard distributions included in the list of available built-in factor functions.  In this case, it must be one in which all the parameters can be fixed to pre-defined constants.  For example:

\ifmatlab
\begin{lstlisting}
r = Real();
r.Input = FactorFunction('Normal', measuredMean, measurementPrecision);
\end{lstlisting}
\fi

\ifjava
\begin{lstlisting}
Real r = new Real();
r.setInput(new Normal(measuredMean, measurementPrecision));
\end{lstlisting}
\fi

For the Gaussian solver, the Input property is a two-element array, where the first element is the mean, and the second element is the standard deviation.  For example:

\ifmatlab
\begin{lstlisting}
r = Real();
r.Input = [measuredMean, measurementStandardDeviation];
\end{lstlisting}
\fi

\ifjava
\begin{lstlisting}
Real r = new Real();
r.setInput( new double [] {measuredMean, measurementStandardDeviation});
\end{lstlisting}
\fi

In the current version of Dimple, Inputs on Real variable arrays must be set one at a time, or all set to a single common value\footnote{This restriction may be removed in a future version.}.

\subpara{FixedValue}

Read-write.  The behavior of the FixedValue property for a Real variable is nearly identical to that of Discrete variables (see section~\ref{sec:Discrete.FixedValue}).  When setting the FixedValue of a Real variable, the value must be within the domain of the variable, that is greater than or equal to the lower bound and less than or equal to the upper bound.  For example:

\ifmatlab
\begin{lstlisting}
a = Real([-pi pi]);
a.FixedValue = 1.7;
\end{lstlisting}
\fi

\ifjava
\begin{lstlisting}
Real a = new Real(-Math.PI,Math.PI);
a.setFixedValue(1.7);
\end{lstlisting}
\fi

Because the Input and FixedValue properties serve similar purposes, setting one of these overrides any previous use of the other.  Setting the Input property removes any fixed value and setting the FixedValue property removes the input.


\para{Methods}

\subpara{hasFixedValue}

See section~\ref{sec:Discrete.hasFixedValue}.



\subsubsection{RealJoint}

A RealJoint variable is a tightly coupled set of real variables that are treated by a solver as a single joint variable rather than a separate collection of variables.  For example, in the Gaussian solver, the messages associated with RealJoint variables involve joint mean and covariance matrix rather than an individual mean and variance for each variable.

Like other variables, the RealJoint class can represent either a single RealJoint variable (representing a collection of real values) or an array of RealJoint variables.

\para{Constructor}

\begin{lstlisting}
RealJoint(numElements, [dimensions])
\end{lstlisting}

The arguments are defined as follows:

\begin{itemize}
\item numElements specifies the number of joint real-valued elements.
\item dimensions specify the array dimensions (the array of individual RealJoint variables).  The behavior of the list of dimensions is identical to that for Discrete variables as described in section~\ref{sec:VariableMatrixDimensions}.
\end{itemize}

Note that in the current version of Dimple, RealJoint variables do not support bounded domains, thus the domain of each of the individual real values is assumed to range from $-\infty$ to $\infty$\footnote{This limitation may be removed in a future version of Dimple.}.

\para{Properties}

\subpara{Belief}

Read-only.  After running the Gaussian solver, this property returns the estimated marginal distribution of the variable in the form of a MultivariateMsg (see section~\ref{sec:MultivariateMsg}), which includes a mean vector and covariance matrix.  For an array of RealJoint variables, this property returns a cell array of MultivariateMsg objects, each corresponding to the estimated marginal distribution of the corresponding variable.  The results are undefined if called prior to running a solver.

\subpara{Input}

Read-write.  For the Gaussian solver, for a single RealJoint variable, the value of this property must be a MultivariateMsg (see section~\ref{sec:MultivariateMsg}), which includes a mean vector and covariance matrix.  For an array of RealJoint variables, the value of this property must be a cell array of MultivariateMsg objects, with dimensions equal to the dimension of the array (or the portion of the array) being set.


\ifmatlab
\subsubsection{ComplexVar}

ComplexVar is a special kind of RealJoint variable with exactly two joint elements.

The behavior of all properties and methods is identical to that of RealJoint variables.

\para{Constructor}

\begin{lstlisting}
ComplexVar([dimensions])
\end{lstlisting}

The behavior of the list of dimensions is identical to that for Discrete variables as described in section~\ref{sec:VariableMatrixDimensions}.

\fi

\subsubsection{FiniteFieldVariable}

Dimple supports a special variable type called a FiniteFieldVariable, which represent finite fields with $N=2^{n}$ elements. These fields find frequent use in error correcting codes.  These variables are used along with certain custom factors that are implemented more efficiently for sum-product belief propagation than the alternative using discrete variables and factors implemented directly.  See section~\ref{sec:finiteFields} for more information on how these variables are used.

The behavior of all properties and methods is identical to that of Discrete variables.

\para{Constructor}

\ifmatlab
\begin{lstlisting}
FiniteFieldVariable(primitivePolynomial, [dimensions])
\end{lstlisting}
\fi

\ifjava
\begin{lstlisting}
FiniteFieldVariable(primitivePolynomial)
\end{lstlisting}
\fi

The arguments are defined as follows:

\ifmatlab
\begin{itemize}
\item primitivePolynomial the primitive polynomial of the finite field.  The format of the primitive polynomial follows the same definition used by MATLAB in the \texttt{gf} function.  See the MATLAB help on the \texttt{gf} function for more detail.
\item dimensions specify the array dimensions (the array of individual FiniteFieldVariable variables).  The behavior of the list of dimensions is identical to that for Discrete variables as described in section~\ref{sec:VariableMatrixDimensions}.
\end{itemize}
\fi

\ifjava
\begin{itemize}
\item primitivePolynomial the primitive polynomial of the finite field.  The format of the primitive polynomial follows the same definition used by MATLAB in the \texttt{gf} function.  See the MATLAB help on the \texttt{gf} function for more detail.
\end{itemize}
\fi

\subsubsection{DiscreteDomain}
\label{sec:DiscreteDomain}

The DiscreteDomain class is used to refer to the domain of Discrete variables.

\para{Constructor}

\begin{lstlisting}
DiscreteDomain(elementList)
\end{lstlisting}

The elementList argument is either a cell array or array of domain elements.  Every entry of the array or cell array is considered an element of the domain, regardless of the number of dimensions it has.  For a cell array, each object in the cell array is considered an element of the domain regardless of the object type.  For a numeric array, every entry in the array must be numeric.

\para{Properties}

\subpara{Elements}

Read-only.  This property returns the set of elements in the discrete domain in the form of a one-dimensional cell array.




\subsubsection{RealDomain}
\label{sec:RealDomain}

The RealDomain class is used to refer to the domain of Real variables.

\para{Constructor}

\ifmatlab
\begin{lstlisting}
RealDomain([lowerBound, [upperBound] ])
\end{lstlisting}

\begin{itemize}
\item lowerBound indicates the lower bound of the domain.  The value must be a scalar numeric value.  It may be set to -Inf to indicate that there is no lower bound.  The default value is -Inf.
\item upperBound indicates the upper bound of the domain.  The value must be a scalar numeric value.  It may be set to Inf to indicate that there is no upper bound.  The default value is Inf.
\end{itemize}

\fi

\ifjava
\begin{lstlisting}
RealDomain(lowerBound,upperBound)
RealDomain()
RealDomain(double [] bounds)
\end{lstlisting}

\begin{itemize}
\item lowerBound indicates the lower bound of the domain.  The value must be a scalar numeric value.  It may be set to -Inf to indicate that there is no lower bound.  The default value is -Inf.
\item upperBound indicates the upper bound of the domain.  The value must be a scalar numeric value.  It may be set to Inf to indicate that there is no upper bound.  The default value is Inf.
\end{itemize}

\fi

\para{Properties}

\ifmatlab
\subpara{LB}
\fi
\ifjava
\subpara{LowerBound}
\fi

Read-only.  This property returns the value of the lower bound.  The default value is -Inf.

\ifmatlab
\subpara{UB}
\fi
\ifjava
\subpara{UpperBound}
\fi

Read-only.  This property returns the value of the upper bound.  The default value is Inf.


\subsubsection{MultivariateMsg}
\label{sec:MultivariateMsg}

The MultivariateMsg class is used to specify the parameters of a multivariate Gaussian distribution, as used in the Gaussian solver.

\para{Constructor}

\begin{lstlisting}
MultivariateMsg(meanVector, covarianceMatrix)
\end{lstlisting}

\begin{itemize}
\item meanVector indicates the mean value of each element in a joint set of variables.  The value must be a one-dimensional numeric array.
\item covarianceMatrix indicates the covariance matrix associated with the elements of a joint set of variables.  The value must be a two-dimensional numeric array with each dimension identical to the length of the meanVector.
\end{itemize}


\para{Properties}

\subpara{Means}

Read-only.  Returns a vector of values, where each value indicates the mean value of each element in a joint set of variables.

\subpara{Covariance}

Read-only.  Returns a two-dimensional array of values, representing the covariance matrix of a joint set of variables.








\subsection{Factors}

\subsubsection{Factor}

The Factor class can represent either a single factor or a multidimensional array of factors.  The Factor class is never created directly, but is the result of using the addFactor or addFactorVectorized (or related) methods on a FactorGraph.

\para{Properties}

\subpara{Name}

Read-write.  When read, retrieves the current name of the factor or array of factors.  When set, modifies the name of the factor to the corresponding value.  The value set must be a string.

\begin{lstlisting}
factor.Name = 'string';
\end{lstlisting}

When setting the Name, only one factor in an array may be set at a time.  To set the names of an entire array of factors to distinct values, the setNames method may be used (see section~\ref{sec:Factor.setNames}).


\subpara{DirectedTo}
\label{sec:Factor.DirectedTo}

Read-write.  The DirectedTo property indicates a set of variables to which the factor is directed.  The value may be either a single variable or a cell array of variables.  The DirectedTo property is used by some solvers, and in some cases is required for proper operation of certain features.  Such cases are identified elsewhere in this manual.

For example, if a factor F corresponds to a function $F(a, b, c, d)$, where $a$, $b$, $c$, and $d$ are variables, then the factor is directed toward $c$ and $d$ if $\sum_{c, d} F(a, b, c, d)$ is constant for all values of $a$ and $b$.  In this case, we may set:

\begin{lstlisting}
F.DirectedTo = {c, d};
\end{lstlisting}

In some cases, the set of DirectedTo variables can be automatically determined when a factor is created.  In this case it need not be set manually by the user.  This includes many built-in factors supported by Dimple.  If this property is set by the user, then in the case of factors connected only to discrete variables, Dimple will check that the factor is in fact directed in the specified direction.

If the set of variables the factor are directed toward are part of a variable array, then these may be specified together in a single cell array.  For example, if varArray is an array of variables, and a factor F is directed toward all of the variables in varArray, then we can set:

\begin{lstlisting}
F.DirectedTo = varArray;
\end{lstlisting}

In the case of a vector of factors, we can identify the variables to which each factor is directed in a vectorized way.  For example:

\begin{lstlisting}
s = Discrete(domain,N);
fg.addFactorVectorized(factorFunction, s(1:(end-1)), s(2:end)).DirectedTo = s(2:end);
\end{lstlisting}

This example also shows that the DirectedTo property can be set directly on the result of the factor creation without assigning the factor to a named variable.

As a more complicated vectorized example, the following creates 12 factors, each of which contains 10 variables (5 from a and 5 from b).  The first 2 of the 5 from a and the first from b are what the factor is directed to.

\begin{lstlisting}
a = Bit(3,4,5);
b = Bit(3,4,5);
fg.addFactorVectorized(factorFunction, {a, [1 2]}, {b, [1 2]}).DirectedTo = {a(:,:,1:2), b(:,:,1)};
\end{lstlisting}

\subpara{Belief}
\label{sec:Factor.Belief}

Read-only.  The belief of a factor is the joint belief over all joint states of the variables connected to that factor.  There are two properties that represent the belief in different ways: Belief and FullBelief.  Reading the Belief property after the solver has been run\footnote{In the current version of Dimple, the Belief property is only supported for factors connected exclusively to discrete variables, and is supported only by the SumProduct solver.  These restrictions may be removed in a future version.} returns the belief in a compact one-dimensional vector that includes only values that correspond to non-zero entries in the factor table.  This form is used because in some situation, the full representation over all possible variable values (as returned by the FullBelief property) would result in a data structure too large to be practical.

\begin{lstlisting}
fb = myFactor.Belief;
\end{lstlisting}

The result is a vector of belief values, where the order of the vector corresponds to the order of the factor table entries.  The order of factor table entries can be determined from the factor using:

\begin{lstlisting}
ind = f.FactorTable.Indices
\end{lstlisting}

This returns a two-dimensional array, where each row corresponds to one entry in the factor table, and where each column-entry in a row indicates the index into the domain of the corresponding variable (where the order of the variable is the order used when the factor was created).


\subpara{FullBelief}
\label{sec:Factor.FullBelief}

Read-only.  Reading the FullBelief property after the solver has been run\footnote{In the current version of Dimple, the Belief property is only supported for factors connected exclusively to discrete variables, and is supported only by the SumProduct solver.  These restrictions may be removed in a future version.} returns the belief in a multi-dimensional array, where each dimension of the multi-dimensional array ranges over the domain of the corresponding variable (the order of the dimensions corresponds to the variable order used when the factor was created).

\begin{lstlisting}
fb = myFactor.FullBelief;
\end{lstlisting}

\subpara{Ports}

Read-only.  Retrieves a cell array containing a list of Ports connecting the factor to its neighboring variables.


\para{Methods}

\subpara{setNames}
\label{sec:Factor.setNames}

For an array of factors, the setNames method sets the name of each factor in the array to a distinct value derived from the supplied string argument.  When called with:

\begin{lstlisting}
factorArray.setName('baseName');
\end{lstlisting}

the resulting factor names are of the form: \texttt{baseName\textunderscore vv0}, \texttt{baseName\textunderscore vv1}, \texttt{baseName\textunderscore vv2}, etc., where each factor's name is the concatenation of the base name with the suffix \texttt{\textunderscore vv} followed by a unique number for each factor in the array.


\subpara{invokeSolverSpecificMethod}

\begin{lstlisting}
factorArray.invokeSolverSpecificMethod('methodName', arguments);
\end{lstlisting}

For an array of factors, the invokeSolverSpecificMethod calls the specified solver-specific method on each of the factors in the array.  Here, 'methodName' is a text string with the name of the solver-specific method, and arguments is an optional comma-separated list of arguments to that method.  This method does not result in any return values.  For solver specific methods that return results, use the invokeSolverSpecificMethodWithReturnValue method instead.


\subpara{invokeSolverSpecificMethodWithReturnValue}

\begin{lstlisting}
returnArray = factorArray.invokeSolverSpecificMethodWithReturnValue('methodName', arguments);
\end{lstlisting}

For an array of factors, the invokeSolverSpecificMethodWithReturnValue calls the specified solver-specific method on each of the factors in the array, returning a return value for each variable.  Here, 'methodName' is a text string with the name of the solver-specific method, and arguments is an optional comma-separated list of arguments to that method.  The returnArray is a cell-array of the return values of the method, with dimensions equal to the dimensions of the factor array.




\subsubsection{FactorFunction}
\label{sec:FactorFunction}

The FactorFunction class is used specify a Dimple built-in factor function in a way that can be reused for creating multiple factors, and that allows specification of constructor arguments.

\para{Constructor}

\begin{lstlisting}
FactorFunction(factorFunctionName, [constructorArguments])
\end{lstlisting}

\begin{itemize}
\item factorFunctionName is a string that indicates the name of the built-in factor function.
\item constructorArguments is a variable-length comma-separated list of constructor arguments, whose interpretation is specific to the particular built-in factor function.  If no arguments are needed, this list would be empty.
\end{itemize}

There are no available properties or methods in this class.



\subsubsection{FactorTable}

The FactorTable class is used to explicitly specify a factor table in lieu of Dimple creating one automatically from a factor function.  This is sometimes useful in cases where the factor table is highly structured, but automatic creation would be time consuming due to a large number of possible states of the connected variables.

\para{Constructor}

\begin{lstlisting}
FactorTable([ ( indexList, weightList ) | weightMatrix ], domains)
\end{lstlisting}

A FactorTable is constructed by specifying the table values in one of two forms, or by creating an all-zeros FactorTable to be filled in later using the set method.  The first form, specifying an indexList and weightList, is useful for sparse factor tables in which many entires are zero and need not be included in the table.  The second form, specifying a weightMatrix, is useful for dense factor tables in which most or all of the entries are non-zero.

\begin{itemize}
\item indexList is an array where each row represents a set of zero-based indices into the list of domain elements for each successive domain in the given set of domains.
\item weightList is a one-dimensional array of real-valued entries in the factor table, where each entry corresponds to the indices given by the corresponding row of the indexList.
\item weightArray is an N dimensional array of real-valued entries in the factor table.  The number of dimensions, N, must correspond to the number of discrete domain elements given in the subsequent arguments, and the number of elements in each dimension must equal the number of elements in the corresponding domain element.
\item domains is a comma-separated list of one or more discrete variable domains.  Each domain either in the form of a DiscreteDomain object or a cell-array of the elements of the domain.  From an existing variable, the domain can be obtained using the Domain property of that variable.
\end{itemize}

An example using the first method is:
\begin{lstlisting}
a = Bit;
b = Bit;
c = Bit;
ft = FactorTable([0 0 0; 0 1 1; 1 0 1; 1 1 0], [1 .9 .5 .3], a.Domain, b.Domain, c.Domain);
\end{lstlisting}

An example using the second method is:
\begin{lstlisting}
a = Discrete({'red','green','blue'});
b = Bit;
ft = FactorTable([.5 .4 .3; .1 0 .4], a.Domain, b.Domain);
\end{lstlisting}


\para{Properties}

\subpara{Indices}

Read-write.  When read, returns an array of indices of the factor table corresponding to entries in the factor table.  Each row represents a set of zero-based indices into the list of domain elements for each successive domain in the given set of domains.

When written, replaces the previous array of indices with a new array.  When writing using this property, the number of rows in the table must not change since this must equal the number of entires in the Weights.  To change both Indices and Weights simultaneously, use the change method.

\subpara{Weights}

Read-write.  When read, returns a one-dimensional array of real-valued entries in the factor table, where each entry corresponds to the indices given by the corresponding row of the indexList.

When written, replaces the previous array of weights.  When writing using this property, the number of entries must not change since this must equal the number of rows in the Indices.  To change both Indices and Weights simultaneously, use the change method.

\subpara{Domains}

Read-only.  Returns a cell-array of DiscreteDomain objects, each of which represents the corresponding domain specified in the constructor, in the same order as specified in the constructor.


\para{Methods}

\subpara{set}
\label{sec:FactorTable.set}

This method allows setting individual entries in the factor table.  For each entry to be set, the domain values (not indices) are specified followed by the weight value.  To set a single entry, the domain values are specified in a comma-separated list, followed by the weight value.  To set multiple entries in a single call, a cell array of such comma-separated lists are specified.

An example of setting a single entry:
\begin{lstlisting}
ft.set('red', 1, 0.45);
\end{lstlisting}

An example of setting a single entry:
\begin{lstlisting}
ft.set({'red', 1, 0.45}, {'blue', 0, 0.75});
\end{lstlisting}


\subpara{get}

This method retrieves the weight associated with a particular entry in the factor table.  The entry is specified by a comma-separated list of domain values (not indices).  For example:

\begin{lstlisting}
w = ft.get('red', 1);
\end{lstlisting}


\subpara{change}

This method allows simultaneously replacing both the array of Indices and Weights in the factor table.

\begin{lstlisting}
ft.change(indexList, weightList);
\end{lstlisting}

The arguments indexList and weightList are exactly as described in the FactorTable constructor.




%\subsubsection{Factor Belief}
%
%
%Users can retrieve beliefs from Factors. As demonstrated in this unit test:
%
%\begin{lstlisting}
%function output = funkyFactor(x,y,z)
%   if x == y && y == z
%       if x == 0
%           output = 1;
%       else
%           output = 2;
%       end
%   else
%       output = 0;
%   end
%end
%    %Create a Factor Graph
%       b = Bit(3,1);
%       fg = FactorGraph();
%       f = fg.addFactor(@funkyFactor,b(1),b(2),b(3));
%       
%       %Set inputs
%       input = [.8 .8 .6];
%       b.Input = input;
%       %We have to solver right now
%       fg.Solver.setNumIterations(1);
%       fg.solve();
%       %funkyFactor only allows these two combos.
%       expectedDomain = {0,0,0;1,1,1};
%       assertEqual(expectedDomain,f.Domain);
%       %Bit stores values as 1,0.
%       expectedIndices = int32([2,2,2;1,1,1]);
%       assertEqual(expectedIndices,f.Indices);
%       %Let's calculate the belief by hand and compare to Dimple.
%       val0 = prod(1-input)*funkyFactor(0,0,0);
%       val1 = prod(input)*funkyFactor(1,1,1);
%       total = val0 + val1;
%       val0 = val0 / total;
%       val1 = val1 / total;
%       expectedBelief = [val0; val1];
%       assertElementsAlmostEqual(expectedBelief,f.Belief);
%       %Now we create a full belief and compare to our expected.
%      expectedFullBelief = zeros(2,2,2);
%       expectedFullBelief(1,1,1) = val1;
%       expectedFullBelief(2,2,2) = val0;
%       assertElementsAlmostEqual(expectedFullBelief,f.FullBelief);
%\end{lstlisting}
%
%%%%%%%%%%%%%%%%%%%%%

\subsection{Schedulers}
\label{sec:Schedulers}

A scheduler defines a rule that determines the update schedule of a factor graph when performing inference.  This section describes all of the built-in schedulers available in Dimple.

Each scheduler is applicable only to a certain subset of solvers.  For all of the BP solvers (SumProduct, MinSum, Gaussian, ParticleBP), the following schedulers are available:

\begin{longtable}{l p{4in}}
\textbf{Name} & \textbf{Description} \\ \hline \hline
%
\textsf{DefaultScheduler} & Same as the TreeOrFloodingScheduler, which is the default if no scheduler or custom schedule is specified. \\ \hline
%
\textsf{TreeOrFloodingScheduler} & The solver will use either a Tree Schedule or a Flooding Schedule depending on whether the factor-graph contains cycles.  In a nested graph, this choice is applied independently in each subgraph.  If the factor-graph is a tree, the scheduler will automatically detect this and use a Tree Schedule.  In this schedule, each node is updated in an order that will result in the correct beliefs being computed after just one iteration.  If the entire graph is a tree, NumIterations should be set to 1, which is its default value.  If the factor-graph is loopy, the solver will instead use a Flooding Schedule (as described below). \\ \hline
%
\textsf{TreeOrSequentialScheduler} & The solver will use either a Tree Schedule (as described above) or a Sequential Schedule (as described below) depending on whether the factor-graph contains cycles.  In a nested graph, this choice is applied independently in each subgraph.  \\ \hline
%
\textsf{FloodingScheduler} & The solver will apply a Flooding Schedule.  For each iteration, all variable nodes are updated, followed by all factor nodes.  Because the graph is bipartite (factor nodes only connect to variable nodes, and vice versa), the order of update within each node type does not affect the result. \\ \hline
%
\textsf{SequentialScheduler} & The solver will apply a Sequential Schedule.  For each factor node in the graph, first, for each variable connected to that factor, the edge connecting the variable to the factor is updated; then the factor node is updated.  The specific order of factors chosen is arbitrary, and depends on the order that factors were added to the graph. \\ \hline
%
\textsf{RandomWithoutReplacementScheduler} & The solver will apply a Sequential Schedule with the order of factors chosen randomly without replacement.  On each subsequent iteration, a new random order is chosen.  Since the factor order is chosen randomly with replacement, on each iteration, each factor will be updated exactly once. \\ \hline
%
\textsf{RandomWithReplacementScheduler} & The solver will apply a Sequential Schedule with the order of factors chosen randomly with replacement.  On each subsequent iteration, a new random order is chosen.  The number of factors updated per iteration is equal to the total number of factors in the graph. However, since the factors are chosen randomly with replacement, not all factors are necessarily updated in a single iteration, and some may be updated more than once. \\ \hline
%
\end{longtable}


In a nested graph, for most of the schedulers listed above (except for the random schedulers), the schedule is applied hierarchically.  In particular, a subgraph is treated as a factor in the nesting level that it appears.  When that subgraph is updated, the schedule for the corresponding subgraph is run in its entirety, updating all factors and variables contained within according to its specified schedule.

It is possible for subgraphs to be designated to use a schedule different from that of its parent graph.  This can be done by specifying either a scheduler or a custom schedule for the subgraph prior to adding it to the parent graph.  For example:
%
\begin{lstlisting}
SubGraph.Scheduler = 'SequentialScheduler';
ParentGraph.addFactor(SubGraph, boundaryVariables);
ParentGraph.Scheduler = 'FloodingScheduler';
\end{lstlisting}

For the TreeOrFloodingScheduler and the TreeOrSequentialScheduler, the choice of schedule is done independently in the outer graph and in each subgraph.  In case that a subgraph is a tree, the tree scheduler will be applied when updating that subgraph even if the parent graph is loopy.  This structure can improve the performance of belief propagation by ensuring that the effect of variables at the boundary of the subgraph fully propagates to all other variables in the subgraph on each iteration.

For the RandomWithoutReplacementScheduler and RandomWithReplacementScheduler, if these are applied to a graph or subgraph, the hierarchy of any lower nesting layers is ignored.  That is, the subgraphs below are essentially flattened prior to schedule creation, and any schedulers or custom schedules specified in lower layers of the hierarchy are ignored.


Because of the differences in operation between the Gibbs solver and the BP based solvers, the Gibbs solver supports a distinct set of schedulers.  For the Gibbs solver, the following schedulers are available:

\begin{tabular}{l p{4in}}
\textbf{Name} & \textbf{Description} \\ \hline \hline
%
\textsf{GibbsDefaultScheduler} & Same as the GibbsSequentialScanScheduler, which is the default when using the Gibbs solver. \\ \hline
%
\textsf{GibbsSequentialScanScheduler} & The solver will apply a Sequential Scan Schedule.  For each scan, each variable is resampled in a fixed order.  The specific order of variables chosen is arbitrary, and depends on the order that variables were added to the graph. \\ \hline
%
\textsf{GibbsRandomScanScheduler} & The solver will apply a Random Scan Schedule.  Each successive variable to be resampled is chosen randomly with replacement.  The number of variables resampled per scan is equal to the total number of variables in the graph, but not all variables are necessarily resampled in a given scan, and some may be resampled more than once. \\ \hline
%
\end{tabular}

Because of the nature of the Gibbs solver, the nested structure of a graph is ignored in creating the schedule.  That is, the graph hierarchy is essentially flattened prior to schedule creation, and only the scheduler specified on the outermost graph is applied.


\subsection{Solvers}
\label{sec:SolversAPI}

\subsubsection{Solver-Specific Methods}

Each solver supports a series of solver-specific methods, which are described in the following sections.  Solver-specific methods may be available for various objects: a factor-graph, variable, or factor.  In each case, to call a solver-specific method, the method is applied to the solver object, returned by the Solver property.  For example:

\ifmatlab
\begin{lstlisting}
factorGraph.Solver.solverSpecificMethod(arguments);
\end{lstlisting}

\begin{lstlisting}
variable.Solver.solverSpecificMethod(arguments);
\end{lstlisting}

\begin{lstlisting}
factor.Solver.solverSpecificMethod(arguments);
\end{lstlisting}
\fi

\ifjava
\begin{lstlisting}
factorGraph.getSolver().solverSpecificMethod(arguments);
\end{lstlisting}

\begin{lstlisting}
variable.getSolver().solverSpecificMethod(arguments);
\end{lstlisting}

\begin{lstlisting}
factor.getSolver().solverSpecificMethod(arguments);
\end{lstlisting}
\fi


Some solver-specific methods return results, while others do not.  Some solver-specific methods require arguments, while others do not.  \ifmatlab If no arguments are needed, the parentheses are optional. \fi

\ifmatlab

In some cases it is convenient to call solver-specific methods on each element in an array of objects.  Utility methods are provided for this purpose.  Specifically, to call a solver-specific method that has no return value:
\begin{lstlisting}
objectArray.invokeSolverSpecificMethod('methodName', arguments);
\end{lstlisting}

In this case, 'methodName' is a text string with the name of the solver-specific method, and arguments is an optional comma-separated list of arguments to that method.

To call a solver specific method that has a return value:
\begin{lstlisting}
returnArray = objectArray.invokeSolverSpecificMethodWithReturnValue('methodName', arguments);
\end{lstlisting}

In this case, returnArray is a cell-array of the return values of the method, with dimensions equal to the dimensions of the object array.

\fi

\subsubsection{Common Methods}

\para{useMultithreading}

Users can turn on and off multithreading with the useMultithreading method:

\ifmatlab
\begin{itemize}
\item fg.Solver.useMultithreading(true);
\item m = fg.Solver.useMultithreading();
\end{itemize}
\fi

\ifjava
\begin{itemize}
\item fg.getSolver().useMultithreading(true);
\item m = fg.getSolver().useMultithreading();
\end{itemize}
\fi

\para{getMultithreadingManager}

Dimple users can retrieve a MultithreadingManager on which to perform additional actions.

\ifmatlab
\begin{lstlisting}
fg.Solver.getMultithreadingManager()
\end{lstlisting}
\fi

\ifjava
\begin{lstlisting}
fg.getSolver().getMultithreadingManager()
\end{lstlisting}
\fi

Users can configure both the multithreading mode and the number of workers using the MultithreadingManager.

\subparagraph{Multithreading Modes}

Dimple provides various multithreading algorithms that have different speed advantages depending on the size of the user's graph and FactorTables.  In the future Dimple should be modified to automatically detect the best threading algorithm.  Currently, however, it defaults to the "Phase" multithreading mode and requires the user manually set the mode to change this.  For a given graph, users can try both modes and see which is faster.

The currently supported multithreading modes are:

\begin{itemize}
\item Phase - Divides the schedule into "phases" where each phase contains schedule entries that are entirely independent of one another.  These phases are then easy to parallelize.  
\item SingleQueue - Uses a single queue and a dependency graph to pull off work for each thread on the fly.  
\end{itemize}

The following methods can be used for getting and setting modes:


\ifmatlab

\begin{itemize}
\item fg.Solver.getMultithreadingManager().getModes() - Returns a Java array of enums specifying the valid modes.
\item fg.Solver.getMultithreadingManager().setMode(ModeName) - Allows users to set the mode by string.  Currently "Phase" or "SingleQueue" will work.
\item fg.Solver.getMultithreadingManager().setMode(enum) - Allows users to set the mode by the enums returned by the getModes method.
\end{itemize}

\fi

\ifjava
\begin{itemize}
\item fg.getSolver().getMultithreadingManager().getModes() - Returns an array of enums specifying the valid modes.
\item fg.getSolver().getMultithreadingManager().setMode(ModeName) - Allows users to set the mode by string.  Currently "Phase" or "SingleQueue" will work.
\item fg.getSolver().getMultithreadingManager().setMode(enum) - Allows users to set the mode by the enums returned by the getModes method or with MultithreadingMode.<PhaseName>.
\end{itemize}
\fi

\subparagraph{Setting Number of Threads and Workers}

Dimple provides a ThreadingPool as a singleton for multithreading.  It sets the number of threads in this pool to the number of virtual cores in the user's machine by default.  Users can override this default value.  In addition, Dimple allows users to specify the number of "workers" for a given FactorGraph.  This "NumWorkers" is also set to the number of virtual cores on the user's machine by default.  Whereas NumThreads specifies how many threads are in the threadPool, NumWorkers specifies how work is divided up across the graph.  These workers are run by the thread pool.  Best performance is achieved when NumWorkers and NumThreads are the same.  However, NumThreads is global and shared by all graphs where NumWorkers is specific to a given FactorGraph.

The following methods can be used to change number of workers:

\ifjava
\begin{itemize}
\item fg.getSolver().getMultithreadingManager().getNumWorkers()
\item fg.getSolver().getMultithreadingManager().setNumWorkers(num)
\item fg.getSolver().getMultithreadingManager().setNumWorkersToDefault()
\end{itemize}
\fi

\ifmatlab
\begin{itemize}
\item fg.Solver.getMultithreadingManager().getNumWorkers()
\item fg.Solver.getMultithreadingManager().setNumWorkers(num)
\item fg.Solver.getMultithreadingManager().setNumWorkersToDefault()
\end{itemize}
\fi

The following global methods can be used to set the number of threads in the ThreadPool

\ifmatlab
\begin{itemize}
\item getDimpleNumThreads()
\item setDimpleNumThreads(numThreads)
\item setDimpleNumThreadsToDefault()
\end{itemize}
\fi

\ifjava
\begin{itemize}
\item ThreadPool.getNumThreads()
\item ThreadPool.setNumThreads(numThreads)
\item ThreadPool.setNumThreadsToDefault()
\end{itemize}
\fi


\subsubsection{Sum-Product Solver}
\label{sec:SumProductSolver}

Use of the sum-product solver is specified by calling:

\ifmatlab
\begin{lstlisting}
fg.Solver = 'SumProduct';
\end{lstlisting}
\fi

\ifjava
\begin{lstlisting}
fg.setSolverFactory(new com.analog.lyric.dimple.solvers.sumproduct.Solver());
\end{lstlisting}
\fi

If no solver is specified, the SumProduct solver is used by default.

The SumProduct solver supports both discrete and continuous variables.  The SumProduct solver uses the sum-product form of belief propagation to perform inference on a graph.  For discrete variables, each message to or from a factor is in the form of a vector of length equal to the domain size of the variable.  For continuous variables, messages are represented using a Gaussian parameterization.  In some cases, this is an approximation to the exact message.  For Real variables, a message is in the form of a pair of values representing the mean and variance of the corresponding Normal distribution.  For Complex and RealJoint variables, a message is in the form of a vector and matrix, representing the mean and covariance of the corresponding multivariate Normal distribution.

While the Gaussian representation of messages for continuous variables is sometimes and approximation, there are some specific built-in factors for which exact Gaussian messages are computed.  This can be done when a factor preserves the Gaussian form of the distribution on each edge.  The following table is a list of such lists built-in factors.  See section~\ref{sec:builtInFactors} for more information on built-in factors.

\begin{longtable} {p{4.0cm} p{3.0cm} p{7.0cm}}
Built-in Factor & Variable Types &  Notes \\
\hline
\endhead
Sum & Real & All connected variables must be Real.  \\
Subtract & Real & All connected variables must be Real. \\
Negate & Real & All connected variables must be Real. \\
ComplexSum & Complex & All connected variables must be Complex. \\
ComplexSubtract & Complex & All connected variables must be Complex. \\
ComplexNegate & Complex & All connected variables must be Complex. \\
RealJointSum & RealJoint & All connected variables must be RealJoint of the same dimension. \\
RealJointSubtract & RealJoint & All connected variables must be RealJoint of the same dimension. \\
RealJointNegate & RealJoint & All connected variables must be RealJoint of the same dimension. \\
Product & Real, Constant & Applies only if the product is one Real variable times one scalar constant to produce a Real variable. \\
MatrixRealJoint\newline VectorProduct & RealJoint, \newline Constant & Applies only if the product is one RealJoint variable times a constant matrix to produce a RealJoint variable. \\
linear & Real, Constant & Linear equation.  All connected variables must be Real. \\
\end{longtable}

For factors that are neither discrete-only or listed in the above table, an approximate computation is used in computing messages from such a factor.  This includes any factor that connects to both discrete and continuous variables as well as factors that connect only to continuous variables but do not appear in the list above.  The approximate method is sample based and uses Gibbs sampling to sample from the factor, allowing approximate messages to be computed from the sample statistics.  Several methods described below allow control over the behavior of these sampled factors.

The following sections list the solver-specific methods for the SumProduct solver.

\para{Graph Methods}

\subpara{setDamping}

The SumProduct solver supports damping, in which messages are damped by replacing each message by a weighted sum of the computed message value and the previous value of that message (when the corresponding edge was most-recently updated).

In the current version of Dimple, damping is supported only in discrete variables and factors that connect only to discrete variables\footnote{Support for damping for continuous variables may be added in a future release.}.

The damping parameter sets the weighting value:

\[
\mathrm{message} = \mathrm{computedMessage} \cdot (1 - D) + \mathrm{previousMessage} \cdot D
\]

where $D$ is the damping value.  The set Damping method can be used to set the damping value.

\ifmatlab
\begin{lstlisting}
fg.Solver.setDamping(dampingValue);
\end{lstlisting}
\fi

\ifjava
\begin{lstlisting}
SFactorGraph solver = ((SFactorGraph)fg.getSolver());
solver.setDamping(dampingValue);
\end{lstlisting}
\fi

The dampingValue argument must be in the range from 0 through 1.  If this method is never called, the default damming value of 0 is used.

\subpara{setSampledFactorSamplesPerUpdate}

Factors connected to continuous variables that do not support exact message computation, instead use a sampled approximation (see section~\ref{sec:SumProductSolver}).  For all such factors in a graph, this method sets the number of samples to generate to form the estimate of each output message.  Note that this sample generation is done separately for each edge of the factor.  The default number of samples used if this method is never called is 1000.

\ifmatlab
\begin{lstlisting}
fg.Solver.setSampledFactorSamplesPerUpdate(samplesPerUpdate);
\end{lstlisting}
\fi

\ifjava
\begin{lstlisting}
SFactorGraph solver = ((SFactorGraph)fg.getSolver());
solver.setSampledFactorSamplesPerUpdate(samplesPerUpdate);
\end{lstlisting}
\fi


\subpara{getSampledFactorSamplesPerUpdate}

This method retrieves the number of samples per update used for factors connected to continuous variables that use approximate sampled message computation.

\ifmatlab
\begin{lstlisting}
samplesPerUpdate = fg.Solver.getSampledFactorSamplesPerUpdate();
\end{lstlisting}
\fi

\ifjava
\begin{lstlisting}
SFactorGraph solver = ((SFactorGraph)fg.getSolver());
int samplesPerUpdate = solver.getSampledFactorSamplesPerUpdate();
\end{lstlisting}
\fi


\subpara{setSampledFactorBurnInScansPerUpdate}

For all factors connected to continuous variables that use approximate sampled message computation, this method sets the number of burn-in scans to perform prior to collecting samples used to compute the output messages.  The default number of burn-in samples used if this method is never called is 10.

\ifmatlab
\begin{lstlisting}
fg.Solver.setSampledFactorBurnInScansPerUpdate(burnInScans);
\end{lstlisting}
\fi

\ifjava
\begin{lstlisting}
SFactorGraph solver = ((SFactorGraph)fg.getSolver());
solver.setSampledFactorBurnInScansPerUpdate(burnInScans);
\end{lstlisting}
\fi


\subpara{getSampledFactorBurnInScansPerUpdate}

This method retrieves the number of burn-in scans per update used for factors connected to continuous variables that use approximate sampled message computation.

\ifmatlab
\begin{lstlisting}
burnInScans = fg.Solver.getSampledFactorBurnInScansPerUpdate();
\end{lstlisting}
\fi

\ifjava
\begin{lstlisting}
SFactorGraph solver = ((SFactorGraph)fg.getSolver());
int burnInScans = solver.getSampledFactorBurnInScansPerUpdate();
\end{lstlisting}
\fi


\subpara{setSampledFactorScansPerSample}

For all factors connected to continuous variables that use approximate sampled message computation, this method sets the number of scans to perform for each sample used to compute the output messages.  The default number of scans per sample used if this method is never called is 1.  Setting this to a value larger than one allows thinning of the sampled output.

\ifmatlab
\begin{lstlisting}
fg.Solver.setSampledFactorScansPerSample(scansPerSample);
\end{lstlisting}
\fi

\ifjava
\begin{lstlisting}
SFactorGraph solver = ((SFactorGraph)fg.getSolver());
solver.setSampledFactorScansPerSample(scansPerSample);
\end{lstlisting}
\fi

\subpara{getSampledFactorScansPerSample}

This method retrieves the scans per sample used for factors connected to continuous variables that use approximate sampled message computation.

\ifmatlab
\begin{lstlisting}
scansPerSample = fg.Solver.getSampledFactorScansPerSample();
\end{lstlisting}
\fi

\ifjava
\begin{lstlisting}
SFactorGraph solver = ((SFactorGraph)fg.getSolver());
int scansPerSample = solver.getSampledFactorScansPerSample();
\end{lstlisting}
\fi


\para{Factor Methods}

\subpara{setK}

The SumProduct solver supports the k-best variant of the sum-product algorithm for discrete variables and factors connected only to discrete variables.  Users can set a K value on each factor:

\ifmatlab
\begin{lstlisting}
factor.Solver.setK(K);
\end{lstlisting}
\fi

\ifjava
\begin{lstlisting}
STableFactor sfactor = (STableFactor)f.getSolver();
sfactor.setK(K);
\end{lstlisting}
\fi


The solver will only use the K best values for each variable when calculating output messages on a factor node.

IMPORTANT: k-best and damping are not compatible with each other\footnote{This restriction may be removed in a future version of Dimple.}.





\subsubsection{Min-Sum Solver}

Use of the MinSum solver is specified by calling:

\ifmatlab
\begin{lstlisting}
fg.Solver = 'MinSum';
\end{lstlisting}
\fi

\ifjava
\begin{lstlisting}
fg.setSolver(new com.analog.lyric.dimple.solvers.minsum.Solver());
\end{lstlisting}
\fi

The MinSum solver supports only discrete variables.

The following sections list the solver-specific methods for the MinSum solver.

\para{Graph Methods}

\subpara{setDamping}

The MinSum solver supports damping, in which messages are damped by replacing each message by a weighted sum of the computed message value and the previous value of that message (when the corresponding edge was most-recently updated).  The damping parameter sets the weighting value:

\[
\mathrm{message} = \mathrm{computedMessage} \cdot (1 - D) + \mathrm{previousMessage} \cdot D
\]

where $D$ is the damping value.  The set Damping method can be used to set the damping value.

\ifmatlab
\begin{lstlisting}
fg.Solver.setDamping(dampingValue);
\end{lstlisting}
\fi

\ifjava
\begin{lstlisting}
SFactorGraph solver = ((SFactorGraph)fg.getSolver());
solver.setDamping(dampingValue);
\end{lstlisting}
\fi


The dampingValue argument must be in the range from 0 through 1.  If this method is never called, the default damming value of 0 is used.

Note that damping in the MinSum solver is done on messages in the log-domain, which has slightly different behavior than damping in the probability domain, as is done in the SumProduct solver.


\para{Factor Methods}

\subpara{setK}

The SumProduct solver supports the k-best variant of the sum-product algorithm.  Users can set a K value on each factor:

\ifmatlab
\begin{lstlisting}
factor.Solver.setK(K);
\end{lstlisting}
\fi

\ifjava
\begin{lstlisting}
STableFactor sfactor = (STableFactor)f.getSolver();
sfactor.setK(K);
\end{lstlisting}
\fi

The solver will only use the K best values for each variable when calculating output messages on a factor node.

IMPORTANT: k-best and damping are not compatible with each other\footnote{This restriction may be removed in a future version of Dimple.}.


\subsubsection{Gibbs Solver}
\label{sec:GibbsSolverAPI}

Use of the Gibbs solver is specified by calling:

\ifmatlab
\begin{lstlisting}
fg.Solver = 'Gibbs';
\end{lstlisting}
\fi

\ifjava
\begin{lstlisting}
fg.setSolver(new com.analog.lyric.dimple.solvers.gibbs.Solver());
\end{lstlisting}
\fi

The Gibbs solver supports both discrete and continuous variables.

This solver performs Gibbs sampling on a factor graph.  It supports a variety of output information on the sampled graph, including the best joint sample (lowest energy), marginals of each variable (discrete variables only), and a full set of samples for a user-selected set of variables.  The solver supports both sequential and randomized scan, and it supports tempering with an exponential tempering schedule.

The Gibbs solver supports several user selectable generic samplers (those that don't require specific conjugacy relationships).  The following table lists the available generic samplers, and the variable types supported by each.

\begin{longtable} {l l p{8.0cm}}
Sampler & Variable Type &  Description \\
\hline
\endhead
CDFSampler & Discrete\footnote{In this table, Discrete support implies any of the discrete variable types, including Discrete and Bit.} & Samples from the full conditional distribution of the variable.  This is the default sampler for discrete variables. \\
SliceSampler & Real\footnote{In this table, Real support implies any of the continuous variable data types, including Real, Complex, and RealJoint.} & Slice sampler using the doubling procedure.  See Neal, Slice Sampling (2000).  This is the default sampler for real variables. \\
MHSampler & Discrete \& Real & Metropolis-Hastings sampler.  For discrete variables, the default proposal kernel is uniform over values other than the current value.  For real variables, the default proposal kernel is Normal with standard deviation 1 (the standard deviation is user settable).  Alternate proposal kernels are also available (see below). \\
SuwaTodoSampler & Discrete & Suwa-Todo sampler.  See Suwa, Todo, Markov Chain Monte Carlo Method without Detailed Balance (2010). \\
\end{longtable}


In cases where the factors of the graph support a conjugate distribution, the solver will automatically determine this and use the appropriate conjugate sampler.  The following table lists the supported conjugate samplers and the corresponding factors that support them.  The corresponding sampler will be used for a given variable if all of the edges connected to that variable support the same sampler\footnote{Additionally, for the conjugate sampler to be used, the domain of the variable must not be bounded to a range smaller than the natural range of the corresponding distribution.}.

\begin{longtable} {l p{7.1cm} p{2.5cm}}
Sampler & Built-in Factor & Edge \\
\hline
\endhead
BetaSampler & Beta & value \\
 & Binomial & $\rho$ \\
DirichletSampler & Dirichlet & value \\
 & Categorical & $\alpha$ \\
 & DiscreteTransition & $\alpha$ \\
GammaSampler & Gamma & value, $\beta$ \\
 & NegativeExpGamma & $\beta$ \\
 & Normal & $\tau$ \\
 & LogNormal & $\tau$ \\
 & CategoricalUnnormalizedParameters & $\alpha$ \\
 & DiscreteTransitionUnnormalizedParameters & $\alpha$ \\
NegativeExpGammaSampler & NegativeExpGamma & value \\
 & CategoricalEnergyParameters & $\alpha$ \\ 
 & DiscreteTransitionEnergyParameters & $\alpha$ \\ 
NormalSampler & Normal & value, $\mu$ \\
 & LogNormal & $\mu$ \\
\end{longtable}

Additionally, conjugate sampling is supported across a subset of applicable deterministic functions.  In the current version of Dimple, this includes the following factors:

\begin{longtable} {l p{2cm} p{9cm}}
Factor Function & Edges & Condition \\
\hline
\endhead
Multiplexer & in, out & If any of the \emph{in} variables would support the same conjugate sampler as the \emph{out} variable, then those \emph{in} variables will use conjugate sampling. \\
\end{longtable}


The Gibbs solver automatically performs block-Gibbs updates for variables that are deterministically related.  The Gibbs solver automatically detects deterministic relationships associated with built-in deterministic factor functions (see section~\ref{sec:builtInFactors} for a list of these functions).  

\ifmatlab
For user-defined factors specified by MATLAB factor functions or factor tables, the Gibbs solver will detect if they are deterministic functions as along as the factor is marked as the directed outputs are indicated using the DirectedTo property, as described in section~\ref{sec:Factor.DirectedTo}.
\fi

\ifjava
For user-defined factors specified by factor functions or factor tables, the Gibbs solver will detect if they are deterministic functions as along as the factor is marked as the directed outputs are indicated using the DirectedTo property, as described in section~\ref{sec:Factor.DirectedTo}.
\fi
 

Most Dimple methods work more-or-less as normal when using the Gibbs solver, but in some cases the interpretation is slightly different. For example, the .Belief method for a discrete variable returns an estimate of the belief based on averaging over the sample values.

NOTE: The setNumIterations() method is not supported by the Gibbs solver as the term ``iteration'' is ambiguous in this case. Instead, the method setNumSamples() should be used to set the length of the run. The Solver.iterate() method performs a single-variable update in the case of the Gibbs solver, rather than an entire scan of all variables.

The following sections list the solver-specific methods for the Gibbs solver.

\para{Graph Methods}

The following methods are available on a graph set to use the Gibbs solver:

\ifmatlab
\begin{lstlisting}
graph.Solver.setNumSamples(numSamples);
graph.Solver.getNumSamples();
\end{lstlisting}
\fi

\ifjava
\begin{lstlisting}
graph.getSolver().setNumSamples(numSamples);
graph.getSolver().getNumSamples();
\end{lstlisting}
\fi

Set/get the number of samples to be run when solving the graph (post burn-in).

\ifmatlab
\begin{lstlisting}
graph.Solver.setScansPerSample(scansPerSample);
graph.Solver.getScansPerSample();
\end{lstlisting}
\fi

\ifjava
\begin{lstlisting}
graph.getSolver().setScansPerSample(scansPerSample);
graph.getSolver().getScansPerSample();
\end{lstlisting}
\fi

Set the number of scans between samples as an alternative means of specifying the sample rate. A scan is an update of the number of variables equal to the total number of variables in the graph.

\ifmatlab
\begin{lstlisting}
graph.Solver.setUpdatesPerSample(updatesPerSample);
graph.Solver.getUpdatesPerSample();
\end{lstlisting}
\fi

\ifjava
\begin{lstlisting}
graph.getSolver().setUpdatesPerSample(updatesPerSample);
graph.getSolver().getUpdatesPerSample();
\end{lstlisting}
\fi


Set/get the number of single-variable updates between samples.

\ifmatlab
\begin{lstlisting}
graph.Solver.setBurnInScans(burnInScans);
graph.Solver.getBurnInScans();
\end{lstlisting}
\fi

\ifjava
\begin{lstlisting}
graph.getSolver().setBurnInScans(burnInScans);
graph.getSolver().getBurnInScans();
\end{lstlisting}
\fi

Set the number of scans for burn-in as an alternative means of specifying the burn-in period.

\ifmatlab
\begin{lstlisting}
graph.Solver.setBurnInUpdates(burnInUpdates);
graph.Solver.getBurnInUpdates();
\end{lstlisting}
\fi

\ifjava
\begin{lstlisting}
graph.getSolver().setBurnInUpdates(burnInUpdates);
graph.getSolver().getBurnInUpdates();
\end{lstlisting}
\fi

Set/get the number of single-variable updates for the burn-in period prior to collecting samples.

\ifmatlab
\begin{lstlisting}
graph.Solver.setNumRestarts(numRestarts)
graph.Solver.getNumRestarts() 
\end{lstlisting}
\fi

\ifjava
\begin{lstlisting}
graph.getSolver().setNumRestarts(numRestarts)
graph.getSolver().getNumRestarts() 
\end{lstlisting}
\fi

Set/get the number of random restarts (zero by default, which means run once and don't restart).  For a value greater than zero, the after running the specified number of samples, the solver is restarted with the variable values randomized, and re-run (including burn-in).  The sample values (the best sample value, or all samples, if requested) are extracted across all runs.

Variable initialization (both on the first run and subsequent restarts) is randomized whenever possible.  For a discrete variable, the value is sampled from the Input (uniform if an input is not specified).  For a real variable, if an Input is specified and the Input supports one of the conjugate samplers listed above, that sampler is used to initialize the variable.  If bounds are also specified for the variable domain, the values is truncated to fall within the bounds.  If only bounds are specified (which are finite above and below), then the value is uniformly sampled from within the bounds.  If no finite bounds are specified and there is no input, the variable is initialized to zero (or the value specified by setInitialSampleValue) on the initial run, and left at the final value of the previous run on restart.


\ifmatlab
\begin{lstlisting}
graph.Solver.setDefaultDiscreteSampler(samplerName);
graph.Solver.getDefaultDiscreteSampler();
\end{lstlisting}
\fi

\ifjava
\begin{lstlisting}
graph.getSolver().setDefaultDiscreteSampler(samplerName);
graph.getSolver().getDefaultDiscreteSampler();
\end{lstlisting}
\fi

Set/get the default sampler for Discrete and Bit.  The default sampler can be set only to one of the generic samplers listed in the table of generic samplers, above.  The CDFSampler is the default sampler for discrete variables unless overridden by this method.  Note that for a particular variable, the default sampler is used only if the user has not overridden the default for that variable using the variable-specific setSampler method.

\ifmatlab
\begin{lstlisting}
graph.Solver.setDefaultRealSampler(samplerName);
graph.Solver.getDefaultRealSampler();
\end{lstlisting}
\fi

\ifjava
\begin{lstlisting}
graph.getSolver().setDefaultRealSampler(samplerName);
graph.getSolver().getDefaultRealSampler();
\end{lstlisting}
\fi

Set/get the default sampler for Real, Complex, and RealJoint variables.  The default sampler can be set only to one of the generic samplers listed in the table of generic samplers, above.  The SliceSampler is the default sampler for real variables unless overridden by this method.  Note that for a particular variable, the default sampler is used only if there is no available conjugate sampler and the user has not overridden the default for that variable using the variable-specific setSampler method.


\ifmatlab
\begin{lstlisting}
graph.Solver.setInitialTemperature(initialTemperature);
graph.Solver.getInitialTemperature();
\end{lstlisting}
\fi

\ifjava
\begin{lstlisting}
graph.getSolver().setInitialTemperature(initialTemperature);
graph.getSolver().getInitialTemperature();
\end{lstlisting}
\fi

Set/get the initial temperature when using tempering. Note that setting the initial temperature automatically enables the use of tempering if it had not been enabled already.

\ifmatlab
\begin{lstlisting}
graph.Solver.setTemperingHalfLifeInSamples();
graph.Solver.getTemperingHalfLifeInSamples();
\end{lstlisting}
\fi

\ifjava
\begin{lstlisting}
graph.getSolver().setTemperingHalfLifeInSamples();
graph.getSolver().getTemperingHalfLifeInSamples();
\end{lstlisting}
\fi


Set/get the tempering half-life.  the number of samples for the temperature to decrease by half. Note that setting the tempering half-life automatically enables the use of tempering if it had not been enabled already.

\ifmatlab
\begin{lstlisting}
graph.Solver.enableTempering();
graph.Solver.disableTempering();
graph.Solver.isTemperingEnabled();
\end{lstlisting}
\fi

\ifjava
\begin{lstlisting}
graph.getSolver().enableTempering();
graph.getSolver().disableTempering();
graph.getSolver().isTemperingEnabled();
\end{lstlisting}
\fi

Enable or disable the use of tempering, or determine if tempering is in use.

\ifmatlab
\begin{lstlisting}
graph.Solver.setTemperature(T);
graph.Solver.getTemperature();
\end{lstlisting}
\fi

\ifjava
\begin{lstlisting}
graph.getSolver().setTemperature(T);
graph.getSolver().getTemperature();
\end{lstlisting}
\fi

Set/get the current temperature. Setting the current temperature overrides the current annealing temperature.

\ifmatlab
\begin{lstlisting}
graph.Solver.saveAllSamples();
\end{lstlisting}
\fi

\ifjava
\begin{lstlisting}
graph.getSolver().saveAllSamples();
\end{lstlisting}
\fi

Prior to solving the graph, this method instructs the solver to save all sample values for all variables. Note that this is practical only for relatively small graphs with relatively small number of samples.

\ifmatlab
\begin{lstlisting}
graph.Solver.disableSavingAllSamples();
\end{lstlisting}
\fi

\ifjava
\begin{lstlisting}
graph.getSolver().disableSavingAllSamples();
\end{lstlisting}
\fi

In case saveAllSamples had previously been called, calling this method disables saving all samples on future runs.

\ifmatlab
\begin{lstlisting}
graph.Solver.saveAllScores();
\end{lstlisting}
\fi

\ifjava
\begin{lstlisting}
graph.getSolver().saveAllScores();
\end{lstlisting}
\fi

Prior to solving the graph, this method instructs the solver to save the score value (see section~\ref{sec:FactorGraph.Score}) of the graph for each sample.

\ifmatlab
\begin{lstlisting}
graph.Solver.disableSavingAllScores();
\end{lstlisting}
\fi

\ifjava
\begin{lstlisting}
graph.getSolver().disableSavingAllScores();
\end{lstlisting}
\fi

In case saveAllScores had previously been called, calling this method disables saving all samples on future runs.

\ifmatlab
\begin{lstlisting}
graph.Solver.getAllScores();
\end{lstlisting}
\fi

\ifjava
\begin{lstlisting}
graph.getSolver().getAllScores();
\end{lstlisting}
\fi

Returns an array including the score value for each sample. This method only returns a non-empty value if .saveAllScores() method had previously been called on the graph.

\ifmatlab
\begin{lstlisting}
graph.Solver.setSeed(seed);
\end{lstlisting}
\fi

\ifjava
\begin{lstlisting}
graph.getSolver().setSeed(seed);
\end{lstlisting}
\fi

Set the random seed used for sampling (and used for random scan, if that schedule is used). Setting the seed allows repeatable execution of the Gibbs solver.

\ifmatlab
\begin{lstlisting}
graph.Solver.getTotalPotential();
\end{lstlisting}
\fi

\ifjava
\begin{lstlisting}
graph.getSolver().getTotalPotential();
\end{lstlisting}
\fi

After running the solver, returns the total potential (score) over all factors of the graph (including input priors on variables) given the most recent sample values.

\ifmatlab
\begin{lstlisting}
graph.Solver.sample(numSamples)
\end{lstlisting}
\fi

\ifjava
\begin{lstlisting}
graph.getSolver().sample(numSamples)
\end{lstlisting}
\fi

This method runs a specified number of samples without re-initializing, burn-in, or random-restarts (this is distinct from iterate(), which runs a specified number of single-variable updates).  Before running this method for the first time, the graph must be initialized using the initialize() method.

\ifmatlab
\begin{lstlisting}
graph.Solver.burnIn()
\end{lstlisting}
\fi

\ifjava
\begin{lstlisting}
graph.getSolver().burnIn()
\end{lstlisting}
\fi

Run the burn-in samples independently of using solve (which automatically runs the burn-in samples).  This may be run before using sample() or iterate().

\para{Variable Methods}

\ifmatlab
\begin{lstlisting}
variable.Solver.getCurrentSample();
\end{lstlisting}
\fi

\ifjava
\begin{lstlisting}
variable.getSolver().getCurrentSample();
\end{lstlisting}
\fi

Returns the current sample value for a variable.

\ifmatlab
\begin{lstlisting}
variable.Solver.getAllSamples();
\end{lstlisting}
\fi

\ifjava
\begin{lstlisting}
variable.getSolver().getAllSamples();
\end{lstlisting}
\fi

Returns an array including all sample values seen so far for a variable. Over multiple variables, samples with the same index correspond to the same joint sample value. This method only returns a non-empty value if .saveAllSamples() method had previously been called on the graph or for the variable.

\ifmatlab
\begin{lstlisting}
variable.Solver.getBestSample();
\end{lstlisting}
\fi

\ifjava
\begin{lstlisting}
variable.getSolver().getBestSample();
\end{lstlisting}
\fi


Returns the value of the best sample value seen so far, where best is defined as the sample with the minimum total potential over the graph (sum of -log of the factor values and input priors).  When getting the best sample from multiple variables, they all correspond to the same sample in time, thus should be a valid sample from the joint distribution.

\ifmatlab
\begin{lstlisting}
variable.Solver.saveAllSamples();
\end{lstlisting}
\fi

\ifjava
\begin{lstlisting}
variable.getSolver().saveAllSamples();
\end{lstlisting}
\fi

Prior to solving the graph, this method instructs the solver to save all sample values for this variable.

\ifmatlab
\begin{lstlisting}
variable.Solver.disableSavingAllSamples();
\end{lstlisting}
\fi

\ifjava
\begin{lstlisting}
variable.getSolver().disableSavingAllSamples();
\end{lstlisting}
\fi

In case saveAllSamples had previously been called, calling this method disables saving all samples for this variable on future runs.

\ifmatlab
\begin{lstlisting}
variable.Solver.setInitialSampleValue(initialSampleValue)
variable.Solver.getInitialSampleValue()
\end{lstlisting}
\fi

\ifjava
\begin{lstlisting}
variable.getSolver().setInitialSampleValue(initialSampleValue)
variable.getSolver().getInitialSampleValue()
\end{lstlisting}
\fi

Set/get the initial sample value to be used as the starting value for this variable.  This value is used only on the first run (not subsequent restarts).  Setting this value overrides any randomization of the starting value on the first run.


\ifmatlab
\begin{lstlisting}
variable.Solver.setSampler(samplerName);
variable.Solver.getSampler();
variable.Solver.getSamplerName();
\end{lstlisting}
\fi

\ifjava
\begin{lstlisting}
variable.getSolver().setSampler(samplerName);
variable.getSolver().getSampler();
variable.getSolver().getSamplerName();
\end{lstlisting}
\fi

Set/get the sampler to be used for this variable.  Setting the sampler for a given variable overrides the default sampler for the given variable type, and also overrides any conjugate sampler that might otherwise be used.  Using this method the sampler may be set only to one of the generic samplers appropriate for the given variable type.

The getSampler method returns the sampler object, while the getSamplerName method returns a string indicating the name of the sampler being used for this variable.  Automatic assignment of a conjugate sampler is done at graph initialization time, so in order to determine what sampler will actually be used, these methods must be called either after a call to the graph initialize method, or after running the solver.


\para{Discrete-Variable-Specific Methods}

The following methods apply to Discrete or Bit variables when using the Gibbs solver.

\ifmatlab
\begin{lstlisting}
variable.Solver.getSampleIndex;
\end{lstlisting}
\fi

\ifjava
\begin{lstlisting}
variable.getSolver().getSampleIndex();
\end{lstlisting}
\fi

Returns the index of the current sample for a variable, where the index refers to the index into the domain of the variable.

\ifmatlab
\begin{lstlisting}
variable.Solver.getAllSampleIndices;
\end{lstlisting}
\fi

\ifjava
\begin{lstlisting}
variable.getSolver().getAllSampleIndices();
\end{lstlisting}
\fi

Returns an array including the indices of all samples seen so far for a variable.

\ifmatlab
\begin{lstlisting}
variable.Solver.getBestSampleIndex;
\end{lstlisting}
\fi

\ifjava
\begin{lstlisting}
variable.getSolver().getBestSampleIndex();
\end{lstlisting}
\fi

Returns the index of the best sample seen so far.


\ifmatlab
\begin{lstlisting}
variable.Solver.setInitialSampleIndex(initialSampleIndex)
variable.Solver.getInitialSampleIndex()
\end{lstlisting}
\fi

\ifjava
\begin{lstlisting}
variable.getSolver().setInitialSampleIndex(initialSampleIndex)
variable.getSolver().getInitialSampleIndex()
\end{lstlisting}
\fi

Set/get the initial sample index associated with the starting value for this variable.  The value associated with this index is used only on the first run (not subsequent restarts).  Setting this index overrides any randomization of the starting value on the first run.


\para{Sampler-Specific Methods}

Some of the generic samplers supported by the Gibbs solver include user configurable options, described in below.

The SliceSampler supports setting/getting the initial slice width and the maximum number of doublings:

\ifmatlab
\begin{lstlisting}
variable.Solver.getSampler().setInitialSliceWidth(width);
variable.Solver.getSampler().getInitialSliceWidth();
\end{lstlisting}
\fi

\ifjava
\begin{lstlisting}
variable.getSolver().getSampler().setInitialSliceWidth(width);
variable.getSolver().getSampler().getInitialSliceWidth();
\end{lstlisting}
\fi

The above methods set/get the initial slice width to be used by the slice sampler.  The default initial width is 1.  For variables with a natural range that is much smaller or much larger than one, it may be beneficial to modify this value.

\ifmatlab
\begin{lstlisting}
variable.Solver.getSampler().setMaximumDoublings(numDoublings);
variable.Solver.getSampler().getMaximumDoublings();
\end{lstlisting}
\fi

\ifjava
\begin{lstlisting}
variable.getSolver().getSampler().setMaximumDoublings(numDoublings);
variable.getSolver().getSampler().getMaximumDoublings();
\end{lstlisting}
\fi

The above methods set/get the maximum number of doublings used by the slice sampler.  The default value is 10.

The MHSampler supports setting/getting the proposal kernel:

\ifmatlab
\begin{lstlisting}
variable.Solver.getSampler().setProposalKernel(proposalKernelName);
variable.Solver.getSampler().setProposalKernel(proposalKernelObject);
variable.Solver.getSampler().getProposalKernel();
variable.Solver.getSampler().getProposalKernelName();
\end{lstlisting}
\fi

\ifjava
\begin{lstlisting}
variable.getSolver().getSampler().setProposalKernel(proposalKernelName);
variable.getSolver().getSampler().setProposalKernel(proposalKernelObject);
variable.getSolver().getSampler().getProposalKernel();
variable.getSolver().getSampler().getProposalKernelName();
\end{lstlisting}
\fi

The setProposalKernel method allows setting the proposal kernel either by name or by specifying an existing proposal kernel object.  For discrete variables, Dimple currently supports only one proposal kernel: UniformDiscreteProposalKernel, though a user may create new proposal kernel Java classes.  For real variables, Dimple currently supports: NormalProposalKernel and CircularNormalProposalKernel.  The default kernel for real variables is NormalProposalKernel.

The method getProposalKernel returns the proposal kernel object, while getProposalKernelName returns a string indicating the name of the current proposal kernel.

The NormalProposalKernel by default makes proposals from a Normal distribution with standard deviation 1.  The standard deviation is settable using:

\ifmatlab
\begin{lstlisting}
variable.Solver.getSampler().getProposalKernel().setStandardDeviation(std);
variable.Solver.getSampler().getProposalKernel().getStandardDeviation();
\end{lstlisting}
\fi

\ifjava
\begin{lstlisting}
variable.getSolver().getSampler().getProposalKernel().setStandardDeviation(std);
variable.getSolver().getSampler().getProposalKernel().getStandardDeviation();
\end{lstlisting}
\fi

The above methods set/get the standard deviation for a NormalProposalKernel.  The default standard deviation is 1.

The CircularNormalProposalKernel makes proposals from a Normal distribution on a circularly wrapping range of the real line.  For example, setting the bounds of the range to $-\pi$ and $\pi$ would create proposals representing angles on a circle.  The CircularNormalProposalKernel has settable parameters that include the standard deviation as well as the lower and upper bounds of the circular range.

\ifmatlab
\begin{lstlisting}
variable.Solver.getSampler().getProposalKernel().setStandardDeviation(std);
variable.Solver.getSampler().getProposalKernel().getStandardDeviation();
\end{lstlisting}
\fi

\ifjava
\begin{lstlisting}
variable.getSolver().getSampler().getProposalKernel().setStandardDeviation(std);
variable.getSolver().getSampler().getProposalKernel().getStandardDeviation();
\end{lstlisting}
\fi

The above methods set/get the standard deviation for a CircularNormalProposalKernel.  The default standard deviation is 1.

\ifmatlab
\begin{lstlisting}
variable.Solver.getSampler().getProposalKernel().setCircularBounds(lower, upper);
variable.Solver.getSampler().getProposalKernel().getLowerBound();
variable.Solver.getSampler().getProposalKernel().getUpperBound();
\end{lstlisting}
\fi

\ifjava
\begin{lstlisting}
variable.getSolver().getSampler().getProposalKernel().setCircularBounds(lower, upper);
variable.getSolver().getSampler().getProposalKernel().getLowerBound();
variable.getSolver().getSampler().getProposalKernel().getUpperBound();
\end{lstlisting}
\fi

The above methods set/get the lower and upper bounds of the circular range of values.  The default lower and upper bounds are $-\pi$ and $\pi$, respectively.


\para{Factor Methods}

\ifmatlab
\begin{lstlisting}
factor.Solver.getPotential();
\end{lstlisting}
\fi

\ifjava
\begin{lstlisting}
factor.getSolver().getPotential();
\end{lstlisting}
\fi

Returns the potential value of a factor given the current values of its connected variables.

\ifmatlab
\begin{lstlisting}
factor.Solver.getPotential(values);
\end{lstlisting}
\fi

\ifjava
\begin{lstlisting}
factor.getSolver().getPotential(values);
\end{lstlisting}
\fi

Get the potential value of a factor given the variable values specified by the argument vector. The argument must be a vector with length equal to the number of connected variables. For a table-factor (connected exclusively to discrete variables), each value corresponds the index into the domain list for that variable (not the value of the variable itself). For a real-factor (connected to one or more real variables), each value corresponds to the value of the variable.

\para{Schedulers}

The built-in schedulers designed for belief propagation are not appropriate for the Gibbs solver.  Instead, there are two built-in schedulers specifically for the Gibbs solver:

\begin{itemize}
\item GibbsSequentialScanScheduler
\item GibbsRandomScanScheduler
\end{itemize}

The GibbsSequentialScanScheduler chooses the next variable for updating in a fixed order. It updates all variables in the graph, completing an entire scan, before repeating the same fixed order. (In Gibbs literature this seems to be known as a sequential-scan, systematic-scan, or fixed-scan schedule.)

The GibbsRandomScanScheduler randomly selects a variable for each update (with replacement).

The default scheduler when using the Gibbs solver is the GibbsSequentialScanScheduler, which is used if no scheduler is explicitly specified.

The user may specify a custom schedule when using the Gibbs solver.  In this case, the schedule should include only Variable node updates (not specific edges), and no Factor updates (any Factor updates specified will be ignored).

To explicitly specify a scheduler, use the Scheduler or Schedule property of the FactorGraph (see sections~\ref{sec:FactorGraph.Scheduler} and~\ref{sec:FactorGraph.Schedule}).


\subsubsection{Particle BP Solver}

Use of the particle BP solver is specified by calling:

\ifmatlab
\begin{lstlisting}
fg.Solver = 'ParticleBP';
\end{lstlisting}
\fi

\ifjava
\begin{lstlisting}
fg.setSolver(new com.analog.lyric.dimple.solvers.particleBP.Solver());
\end{lstlisting}
\fi

The following sections list the solver-specific methods for the ParticleBP solver.

\para{Graph Methods}

\ifmatlab
\begin{lstlisting}
graph.Solver.setNumParticles(numParticles);
\end{lstlisting}
\fi

\ifjava
\begin{lstlisting}
graph.getSolver().setNumParticles(numParticles);
\end{lstlisting}
\fi

For each variable in the graph, sets the number of particles per variable. This is set globally for all variables in the graph as an alternative to setting this for all variables separately.

\ifmatlab
\begin{lstlisting}
graph.Solver.setResamplingUpdatesPerParticle(updatesPerParticle);
\end{lstlisting}
\fi

\ifjava
\begin{lstlisting}
graph.getSolver().setResamplingUpdatesPerParticle(updatesPerParticle);
\end{lstlisting}
\fi

For each variable in the graph, sets the number of updates per particle to perform each time the particle is resampled. This is set globally for all variables in the graph as an alternative to setting this for all variables separately.

\ifmatlab
\begin{lstlisting}
graph.Solver.setNumIterationsBetweenResampling(numIterationsBetweenResampling);
graph.Solver.getNumIterationsBetweenResampling();
\end{lstlisting}
\fi

\ifjava
\begin{lstlisting}
graph.setSolver().setNumIterationsBetweenResampling(numIterationsBetweenResampling);
graph.setSolver().getNumIterationsBetweenResampling();
\end{lstlisting}
\fi

Set/get the number of iterations between re-sampling all of the variables in the graph (default is 1, meaning resample between every iteration).

\ifmatlab
\begin{lstlisting}
graph.Solver.setInitialTemperature(initialTemperature);
graph.Solver.getInitialTemperature();
\end{lstlisting}
\fi

\ifjava
\begin{lstlisting}
graph.getSolver().setInitialTemperature(initialTemperature);
graph.getSolver().getInitialTemperature();
\end{lstlisting}
\fi


Set/get the initial temperature when using tempering. Note that setting the initial temperature automatically enables the use of tempering if it had not been enabled already.

\ifmatlab
\begin{lstlisting}
graph.Solver.setTemperingHalfLifeInIterations();
graph.Solver.getTemperingHalfLifeInIterations();
\end{lstlisting}
\fi

\ifjava
\begin{lstlisting}
graph.getSolver().setTemperingHalfLifeInIterations();
graph.getSolver().getTemperingHalfLifeInIterations();
\end{lstlisting}
\fi

Set/get the tempering half-life�the number of iterations for the temperature to decrease by half. Note that setting the tempering half-life automatically enables the use of tempering if it had not been enabled already.

\ifmatlab
\begin{lstlisting}
graph.Solver.enableTempering();
graph.Solver.disableTempering();
graph.Solver.isTemperingEnabled();
\end{lstlisting}
\fi

\ifjava
\begin{lstlisting}
graph.getSolver().enableTempering();
graph.getSolver().disableTempering();
graph.getSolver().isTemperingEnabled();
\end{lstlisting}
\fi

Enable or disable the use of tempering, or determine if tempering is in use.

\ifmatlab
\begin{lstlisting}
graph.Solver.setTemperature(T);
graph.Solver.getTemperature();
\end{lstlisting}
\fi

\ifjava
\begin{lstlisting}
graph.getSolver().setTemperature(T);
graph.getSolver().getTemperature();
\end{lstlisting}
\fi

Set/get the current temperature. Setting the current temperature overrides the current annealing temperature.

\ifmatlab
\begin{lstlisting}
graph.Solver.setSeed(seed);
\end{lstlisting}
\fi

\ifjava
\begin{lstlisting}
graph.getSolver().setSeed(seed);
\end{lstlisting}
\fi

Set the random seed used for re-sampling. Setting the seed allows repeatable execution of the solver.

\para{Variable Methods}

The Particle BP solver supports both discrete and real variables. For discrete variables, the solver uses sum-product BP as normal, and all of the corresponding methods for the sum-product solver may be used for discrete variables. For real variables, several solver-specific methods are defined, as follows.

\para{Real-Variable-Specific Methods}
\label{sec:ParticleBPRealVariableSpecificMethods}

\ifmatlab
\begin{lstlisting}
variable.Solver.setNumParticles(numParticles);
variable.Solver.getNumParticles();
\end{lstlisting}
\fi

\ifjava
\begin{lstlisting}
variable.getSolver().setNumParticles(numParticles);
variable.getSolver().getNumParticles();
\end{lstlisting}
\fi

Set/get the number of particles to represent this variable.

\ifmatlab
\begin{lstlisting}
variable.Solver.setResamplingUpdatesPerParticle(updatesPerParticle);
variable.Solver.getResamplingUpdatesPerParticle();
\end{lstlisting}
\fi

\ifjava
\begin{lstlisting}
variable.getSolver().setResamplingUpdatesPerParticle(updatesPerParticle);
variable.getSolver().getResamplingUpdatesPerParticle();
\end{lstlisting}
\fi


Set/get the number of updates per particle to perform each time the particle is resampled.

\ifmatlab
\begin{lstlisting}
variable.Solver.setProposalStandardDeviation(stdDev);
variable.Solver.getProposalStandardDeviation();
\end{lstlisting}
\fi

\ifjava
\begin{lstlisting}
variable.getSolver().setProposalStandardDeviation(stdDev);
variable.getSolver().getProposalStandardDeviation();
\end{lstlisting}
\fi


Set/get the standard deviation for a Gaussian proposal distribution (the default is 1).

\ifmatlab
\begin{lstlisting}
variable.Solver.setInitialParticleRange(min, max);
\end{lstlisting}
\fi

\ifjava
\begin{lstlisting}
variable.getSolver().setInitialParticleRange(min, max);
\end{lstlisting}
\fi

Set the range over which the initial particle values will be defined. The initial particle values are uniformly spaced between the min and max values specified. If the range is specified using this method, it overrides any other initial value. Otherwise, if a finite domain has been specified, the initial particle values are uniformly spaced between the lower and upper bound of the domain. Otherwise, all particles are initially set to zero.

\ifmatlab
\begin{lstlisting}
variable.Solver.getParticleValues();
\end{lstlisting}
\fi

\ifjava
\begin{lstlisting}
variable.getSolver().getParticleValues();
\end{lstlisting}
\fi

Returns the current set of particle values associated with the variable.

\ifmatlab
\begin{lstlisting}
variable.Solver.getBelief(valueSet);
\end{lstlisting}
\fi

\ifjava
\begin{lstlisting}
variable.getSolver().getBelief(valueSet);
\end{lstlisting}
\fi

Given a set of values in the domain of the variable, returns the belief evaluated at these points. The result is normalized relative to the set of points requested so that the sum over the set of returned beliefs is 1.

NOTE: the generic variable method Belief (or getBeliefs() with no arguments) operates similarly to the discrete-variable case, but the belief values returned are those at the current set of particle values. Note that this representation does not represent a set of weighted particles. That is, the particle positions are distributed approximately by the belief and the belief values represent the belief. It remains to be see if this should be the representation of belief that is used, or if an alternative representation would be better. The alternative solver-specific getBelief(valueSet) method allows getting the beliefs on a user-specified set of values, which may be uniform, and would not have this unusual interpretation.



\subsubsection{LP Solver}
\label{sec:LPSolver}

\ifmatlab
Use of the linear programming (LP) solver is specified by calling:

\begin{lstlisting}
fg.Solver = 'LP';
\end{lstlisting}

The LP solver transforms a factor graph MAP estimation problem into an equivalent linear program, which is solved using a linear programming software package. The solver can either be a linear programming solver (in which case the MAP is estimated using an LP relaxation, with no guarantees of correctness), or by an integer linear programming (ILP) solver, in which case the solution is guaranteed to be the MAP. 

The choice of the LP solver package used to solve the problem is specified by calling:
\begin{lstlisting}
fg.Solver.setMatlabLPSolver(solvername);
\end{lstlisting}
where solvername is a string containing the solver name. The choices for the string solvername are `matlab', `glpk', `glpkIP', `gurobi', and `gurobiIP'. The `matlab', `glpk', and `gurobi' solvers are linear programming solvers, while `glpkIP' and `gurobiIP' are ILP solvers.

Using the matlab LP solver requires the the MATLAB Optimization Toolbox; using `glpk' or `glpkIP' requires glpkmex to be in the matlab path, and `gurobi' and `gurobiIP' require the gurobi matlab interface to be in the matlab path.\footnote{GLPK and Gurobi are not included in dimple}

The LP solver supports only discrete variables.

In the current version of Dimple (version 0.3), there is no support for rolled-up graphs when using the LP solver.

There are currently no solver-specific methods for the LP solver.

\fi

\ifjava
The LP Solver is not currently supported from java.
\fi

\subsection{Streaming Data}

\subsubsection{Variable Stream Common Properties and Methods}

Dimple supports several types of variable streams:

\begin{itemize}
\item DiscreteStream
\item BitStream
\item RealStream
\item RealJoinStream
\end{itemize}

Each of these share common properties and method listed in the following sections.

\para{Properties}

\subpara{DataSource}

Read-write.  When written, connects the variable stream to a data source (see section~\ref{sec:DataSource}).  The data sink must be of a type appropriate for the particular variable stream type.

\subpara{DataSink}

Read-write.  When written, connects the variable stream to a data sink (see section~\ref{sec:DataSink}).  The data sink must be of a type appropriate for the particular variable stream type.

\subpara{Dimensions}

Read-only.  Indicates the dimensions of the variable stream.  The dimensions correspond to the size of the variable array at each position in the stream.

\subpara{Size}

Read-only.  Indicates the number of elements in the variable stream that are actually instantiated.  Each element corresponds to one copy of the variable or variable array at a specific point in the stream.  Dimple instantiates the minimum number of contiguous elements to cover the slices of the stream that are actually used in factors (see section~\ref{sec:VariableStream.getSlice}), plus the number of additional elements to cover the indicated BufferSize (see section~\ref{sec:FactorGraphStream.BufferSize}).

\subpara{Variables}

Read-only.  Returns a variable array containing all of the currently instantiated variables in the stream.


\para{Methods}

\subpara{getSlice}
\label{sec:VariableStream.getSlice}

\begin{lstlisting}
varStream.getSlice(startIndex);
\end{lstlisting}

The getSlice method is used to extract a \emph{slice} of the stream, which means a version of the stream that may be offset from the original stream itself.  This is generally used for specifying streams to connect to a factor when calling addFactor.

Takes a single numeric argument, startIndex, which indicates the starting position in the stream.  The resulting stream slice is essentially a reference to the stream offset by startIndex-1.  For example, a startIndex of 2 returns a slice offset by 1, such that the first location in the slice corresponds to the second location in the original stream.  A startIndex of 1 returns a slice identical to the original stream.


\subsubsection{DiscreteStream}

\para{Constructor}
\label{DiscreteStream.Constructor}

The DiscreteStream constructor is used to create a stream of Discrete variables or arrays of Discrete variables.

\begin{lstlisting}
DiscreteStream(domain, [dimensions]);
\end{lstlisting}

\begin{itemize}
\item domain is a required argument indicating the domain of the variable.  The domain may either be a numeric array of domain elements, a cell array of domain elements, or a DiscreteDomain object (see section~\ref{Discrete.Domain}).
\item dimensions is an optional variable-length comma-separated list of matrix dimensions (an empty list indicates a single Discrete variable).
\end{itemize}

\subsubsection{BitStream}

\para{Constructor}

The BitStream constructor is used to create a stream of Bit variables or arrays of Bit variables.

\begin{lstlisting}
BitStream([dimensions]);
\end{lstlisting}

\begin{itemize}
\item dimensions is an optional variable-length comma-separated list of matrix dimensions (an empty list indicates a single Bit variable stream).
\end{itemize}

\subsubsection{RealStream}

\para{Constructor}

The RealStream constructor is used to create a stream of Real variables or arrays of Real variables.

\begin{lstlisting}
RealStream([dimensions]);
\end{lstlisting}

\begin{itemize}
\item dimensions is an optional variable-length comma-separated list of matrix dimensions (an empty list indicates a single Real variable stream).
\end{itemize}

\subsubsection{RealJoinStream}

\para{Constructor}

The RealJoinStream constructor is used to create a stream of RealJoint variables or arrays of RealJoint variables.

\begin{lstlisting}
RealJoinStream(numJointVariables, dimensions);
\end{lstlisting}

\begin{itemize}
\item dimensions is an optional variable-length comma-separated list of matrix dimensions (an empty list indicates a single RealJoint variable stream).
\end{itemize}


\subsubsection{FactorGraphStream}

A FactorGraphStream is constructed automatically and returned as the result of adding a factor to a graph using the addFactor method where one or more of the arguments are variable streams.

\para{Properties}

\subpara{BufferSize}
\label{sec:FactorGraphStream.BufferSize}

Read-write. When written, modifies the number of instantiated elements in the FactorGraphStream to include the specified number of copies of the corresponding factor and connected variables.  By default, the BufferSize is 1.  When running the solver on one step of the overall factor graph, the solver uses the entire buffer.  Making the buffer size larger means using more of the history in performing inference for each step.  The results of inference run on previous steps that is beyond the size of the buffer is essentially frozen, and is no longer updated on subsequent steps of the solver.


\subsubsection{Data Source Common Properties and Methods}
\label{sec:DataSource}

Dimple supports several types of streaming data sources.  A data source is a source of Input values to the variables within a variable stream\footnote{In the current version of Dimple, data sources are limited to providing Inputs to variables.  A future version of Dimple may expand this capability to allow sourcing FixedValues or other types of input data.}.  When performing inference, as each step that the graph is advanced, the next source value is read from the data source, and the earlier values are shifted back to earlier time-steps in the graph.

Each source type corresponds to a particular format of input data.  Each type is appropriate only to a specific type of variable stream and solver.  The following table summarizes these requirements.

\begin{longtable} {l | l | l}
Data Source & Variable Stream Type & Supported Solvers \\
\hline
\endhead
DoubleArrayDataSource & DiscreteStream, BitStream & all \\
 & RealStream & Gaussian \\
MultivariateDataSource & RealJointStream & Gaussian \\
FactorFunctionDataSource & RealStream & Gibbs, ParticleBP \\
\end{longtable} 

Each of these share common properties and method listed in the following sections.

\para{Properties}

\subpara{Dimensions}

Read-only.  Indicates the dimensions of the data source.  The dimensions correspond to the size of the variable array at each position in the stream that the data source will feed.


\subsubsection{DoubleArrayDataSource}

\para{Constructor}

\begin{lstlisting}
DoubleArrayDataSource([dimensions], [initialData]);
\end{lstlisting}

\begin{itemize}
\item dimensions is an optional row vector indicating the matrix dimensions of the data source (an empty argument indicates a single data source to connect to a single variable stream).
\item initialData is an optional array of data the stream will contain (more data can be added later).  This is a multidimensional array where the first dimensions correspond to the dimension of the variable stream this will feed, the next dimensions corresponds to the length of the Input vector for each variable (the domain size for discrete variable streams, and 2 for Gaussian variable streams), and the final dimension is the number of time-steps of data to provide.  For single variable streams, the first dimensions are omitted.
\end{itemize}

\para{Methods}

\subpara{add}

\begin{lstlisting}
dataSource.add(data);
\end{lstlisting}

This method appends the data source with the specified data.  The data argument is a multidimensional array, where the first dimensions correspond to the dimension of the variable stream this will feed, the next dimensions corresponds to the length of the Input vector for each variable (the domain size for discrete variable streams, and 2 for Gaussian variable streams), and the final dimension is the number of time-steps of data to provide.  For single variable streams, the first dimensions are omitted.

\subsubsection{MultivariateDataSource}

\para{Constructor}

\begin{lstlisting}
MultivariateDataSource([dimensions]);
\end{lstlisting}

\begin{itemize}
\item dimensions is an optional row vector indicating the matrix dimensions of the data source (an empty argument indicates a single data source to connect to a single variable stream).
\end{itemize}

\para{Methods}

\subpara{add}

\begin{lstlisting}
dataSource.add(means, covariances);
\end{lstlisting}

This method appends the data source with the a single time-step of data (multiple time-steps must be added using successive calls to this method).  Means should have dimensions [VariableDimensions NumberJointVariables] and Covariances should be of the form [VariableDimensions NumberJointVariables NumberJointVariables].

\subsubsection{FactorFunctionDataSource}

\para{Constructor}

\begin{lstlisting}
FactorFunctionDataSource([dimensions]);
\end{lstlisting}

\begin{itemize}
\item dimensions is an optional row vector indicating the matrix dimensions of the data source (an empty argument indicates a single data source to connect to a single variable stream).
\end{itemize}

\para{Methods}

\subpara{add}

\begin{lstlisting}
dataSource.add(data);
\end{lstlisting}

This method appends the data source with the a single time-step of data (multiple time-steps must be added using successive calls to this method).The data argument is a multidimensional cell array, with dimension equal to the corresponding dimensions of the variable stream this will feed.  Each element is a FactorFunctions (see section~\ref{sec:FactorFunction}), which is to represent the Input of the corresponding variable.


\subsubsection{Data Sink Common Properties and Methods}
\label{sec:DataSink}

Dimple supports several types of streaming data sinks:

Dimple supports several types of streaming data sinks.  A data sink is a data structure used to store successive results of inference from the variables with a variable stream.  Specifically, it stores the Belief values of these variables\footnote{In the current version of Dimple, data sinks are limited to the Beliefs to variables.  A future version of Dimple may expand this capability to allow sinking other types of result data.}.  When performing inference, as each step that the graph is advanced, the Belief value for the earliest element of the variable stream is stored in the data sink.

Each sink type corresponds to a particular format of output data.  Each type is appropriate only to a specific type of variable stream and solver.  The following table summarizes these requirements.

\begin{longtable} {l | l | l}
Data Sink & Variable Stream Type & Supported Solvers \\
\hline
\endhead
DoubleArrayDataSink & DiscreteStream, BitStream & all \\
 & RealStream & Gaussian \\
MultivariateDataSink & RealJointStream & Gaussian \\
\end{longtable} 

Each of these share common properties and method listed in the following sections.

\para{Properties}

\subpara{Dimensions}

Read-only.  Indicates the dimensions of the data sink.  The dimensions correspond to the size of the variable array at each position in the stream that the data sink will be fed from.

\para{Methods}

\subpara{hasNext}

\begin{lstlisting}
hasNext = dataSink.hasNext();
\end{lstlisting}

Used in connection with the getNext method (described in the sections below), this method takes no arguments and returns a boolean indicating whether or not there are any more time steps in the dataSink that have not yet been extracted.



\subsubsection{DoubleArrayDataSink}
\para{Constructor}

\begin{lstlisting}
DoubleArrayDataSink([dimensions]);
\end{lstlisting}

\begin{itemize}
\item dimensions is an optional array indicating the matrix dimensions of the data sink (an empty argument indicates a single data sink to connect to a single variable stream).
\end{itemize}

\para{Properties}

\subpara{Array}

Read-only.  Extracts the entire contents of the data sink as an array.  The first dimensions of the array correspond to the Dimensions of the data sink, the next dimension corresponds to the size of the belief array for each variable, and the final dimension corresponds to the number of time steps that had been gathered.  For discrete variables, the dimension of the belief array corresponds to the domain sizes, while for Gaussian variables the dimension is 2, where the elements correspond to the mean and standard deviation, respectively.

\para{Methods}

\subpara{getNext}

\begin{lstlisting}
b = dataSink.getNext();
\end{lstlisting}

This method takes no arguments, and returns the set of belief values from the next time-step.  The returned value is a multidimensional array, where the first dimensions correspond to the Dimension of the data sink, and the next dimension corresponds to the size of the belief array for each variable.  For discrete variables, the dimension of the belief array corresponds to the domain sizes, while for Gaussian variables the dimension is 2, where the elements correspond to the mean and standard deviation, respectively.


\subsubsection{MultivariateDataSink}

\para{Constructor}

\begin{lstlisting}
MultivariateDataSink([dimensions]);
\end{lstlisting}

\begin{itemize}
\item dimensions is an optional array indicating the matrix dimensions of the data sink (an empty argument indicates a single data sink to connect to a single variable stream).
\end{itemize}

\para{Methods}
\subpara{getNext}

\begin{lstlisting}
b = dataSink.getNext();
\end{lstlisting}

This method takes no arguments, and returns the set of belief values from the next time-step.  The returned value is a cell array of MultivariateMsg objects (see section~\ref{sec:MultivariateMsg}), each of which contains and mean vector and covariance matrix.  The dimensions of this cell array correspond to the Dimension of the data sink.



\subsection{List of Built-in Factors}
\label{sec:builtInFactors}

The following table lists the current set of built-in Dimple factors.  For each, the name is given, followed by the set of variables that would be connected to the factor, followed by any constructor arguments.  Optional variables and constructor arguments are in brackets.  And an arbitrary length list or array of variables is followed by ellipses.  The allowed set of variable data-types for each variable is given in parentheses (B~=~Bit, D~=~Discrete, R~=~Real, C~=~Complex, RJ~=~RealJoint).

\begin{longtable} {p{3.5cm} p{2.2cm} p{2cm} p{7cm}}
Name & Variables & Constructor & Description \\
\hline
\endhead
%
Abs & out(D,R) \newline in(D,R) & [smoothing] & Deterministic absolute value function, where out = abs(in).  An optional smoothing value may be specified as a constructor argument\footnote{\label{ftn:smoothing}If smoothing is enabled, the factor function becomes $e^{-(\textrm{out} - F(\textrm{in}))^2/\textrm{smoothing}}$ (making it non-deterministic) instead of $\delta(\textrm{out} - F(\textrm{in}))$, where $F$ is the deterministic function associated with this factor.  This is useful for solvers that do not work well with deterministic real-valued factors, such as particle BP, particularly when tempering is used.}.\\
%
ACos & out(D,R) \newline in(D,R) & [smoothing] & Deterministic arc-cosine function, where out = acos(in). An optional smoothing value may be specified as a constructor argument$^{\ref{ftn:smoothing}}$. \\
%
AdditiveNoise & out(R) \newline in(B,D,R) & $\sigma$ & Add Gaussian noise with a known standard deviation, $\sigma$, specified in constructor. \\
%
And & out(B) \newline in...(B) & - & Deterministic logical AND function, where out = AND(in...). \\
%
ASin & out(D,R) \newline in(D,R) & [smoothing] & Deterministic arc-sine function, where out = asin(in). An optional smoothing value may be specified as a constructor argument$^{\ref{ftn:smoothing}}$. \\
%
ATan & out(D,R) \newline in(D,R) & [smoothing] & Deterministic arc-tangent function, where out = atan(in). An optional smoothing value may be specified as a constructor argument$^{\ref{ftn:smoothing}}$. \\
%
Bernoulli & $\rho$(R) \newline x...(B) & - & Bernoulli distribution, $p(x|\rho)$, where $\rho$ is a parameter indicating the probability of one, and x is an array of Bit variables.  There can be any number of x variables, all associated with the same parameter value. The conjugate prior for the parameter, $\rho$, is a Beta distribution\footnote{\label{ftn:conugatePrior}It is not necessary to use the conjugate prior, but in some cases there may be a benefit.}. \\
%
Beta & [$\alpha$](R) \newline [$\beta$](R) \newline value...(R) & [$\alpha$] \newline [$\beta$] & Beta distribution. There can be any number of value variables, all associated with the same parameter values.  Parameters $\alpha$ and $\beta$ can be variables, or if both are constant they can be specified in the constructor. \\
%
Categorical & $\alpha$...(RJ)  \newline x...(D) & - & Categorical distribution, $p(x | \alpha)$, where $\alpha$ is a vector of parameter variables and x is an array of discrete variables.  The number of elements in $\alpha$ must equal the domain size of x.  There can be any number of x variables, all associated with the same parameter values.  \newline
The $\alpha$ parameters are represented as a (not necessarily normalized) probability vector. The conjugate prior for this representation is a Dirichlet distribution\textsuperscript{\ref{ftn:conugatePrior}}.  \newline
In the current implementation, the domain of the x variable must be zero-based contiguous integers, $0...N-1$\footnote{\label{ftn:zeroBaseRestriction}This limitation may be lifted in a future version.}. \\
%
Categorical\newline EnergyParameters & $\alpha$...(R)  \newline x...(D) & N & Categorical distribution, $p(x | \alpha)$, where $\alpha$ is a vector of parameter variables and x is an array of discrete variables.  The number of elements in $\alpha$ and the domain size of x must equal the value of the constructor argument, N.  There can be any number of x variables, all associated with the same parameter values.  \newline
In this alternative version of the Categorical distribution, the $\alpha$ parameters are represented as energy values, that is, $\alpha = -\log(\rho)$, where $\rho$ are unnormalized probabilities. The conjugate prior for this representation is such that each entry of $\alpha$ is independently distributed according to a negative exp-Gamma distribution, all with a common $\beta$ parameter\textsuperscript{\ref{ftn:conugatePrior}}.  \newline
In the current implementation, the domain of the x variable must be zero-based contiguous integers, $0...N-1$\textsuperscript{\ref{ftn:zeroBaseRestriction}}. \\
%
Categorical\newline Unnormalized\newline Parameters & $\alpha$...(R)  \newline x...(D) & N & Categorical distribution, $p(x | \alpha)$, where $\alpha$ is a vector of parameter variables and x is an array of discrete variables.  The number of elements in $\alpha$ and the domain size of x must equal the value of the constructor argument, N.  There can be any number of x variables, all associated with the same parameter values.  \newline
In this alternative version of the Categorical distribution, the $\alpha$ parameters are represented as a vector of unnormalized probability values. The conjugate prior for this representation is such that each entry of $\alpha$ is independently distributed according to a Gamma distribution, all with a common $\beta$ parameter\textsuperscript{\ref{ftn:conugatePrior}}.  \newline
In the current implementation, the domain of the x variable must be zero-based contiguous integers, $0...N-1$\textsuperscript{\ref{ftn:zeroBaseRestriction}}. \\
%
ComplexConjugate & out(C) \newline in(C,R) & [smoothing] & Deterministic complex conjugate function, where out = in$^{*}$. An optional smoothing value may be specified as a constructor argument$^{\ref{ftn:smoothing}}$. \\
%
ComplexDivide & quotient(C) \newline dividend(C,R) \newline divisor(C,R) & [smoothing] & Deterministic complex divide function, where $\mathrm{quotient} = \frac{\mathrm{dividend}}{\mathrm{divisor}}$. An optional smoothing value may be specified as a constructor argument$^{\ref{ftn:smoothing}}$. \\
%
ComplexExp & out(C) \newline in(C,R) & [smoothing] & Deterministic complex exponentiation function, where out = exp(in). An optional smoothing value may be specified as a constructor argument$^{\ref{ftn:smoothing}}$. \\
%
ComplexNegate & out(C) \newline in(C,R) & [smoothing] & Deterministic complex negation function, where out = -in. An optional smoothing value may be specified as a constructor argument$^{\ref{ftn:smoothing}}$. \\
%
ComplexProduct & out(C) \newline in...(C,R) & [smoothing] & Deterministic complex product function, where $\mathrm{out} = \prod \mathrm{in}$. An optional smoothing value may be specified as a constructor argument$^{\ref{ftn:smoothing}}$. \\
%
ComplexSubtract & out(C) \newline posIn(C,R) \newline negIn...(C,R) & [smoothing] & Deterministic complex subtraction function, where $\mathrm{out} = \mathrm{posIn} - \sum \mathrm{negIn}$. An optional smoothing value may be specified as a constructor argument$^{\ref{ftn:smoothing}}$. \\
%
ComplexSum & out(C) \newline in...(C,R) & [smoothing] & Deterministic complex summation function, where $\mathrm{out} = \sum \mathrm{in}$. An optional smoothing value may be specified as a constructor argument$^{\ref{ftn:smoothing}}$. \\
%%
%% Primarily for testing, need not be listed here.
%ConstantPower & out(D,R) \newline base(D,R) & power \newline [smoothing] & Deterministic power function, with a constant power. The power value is specified in the constructor. An optional smoothing value may be specified as a constructor argument$^{\ref{ftn:smoothing}}$. \\
%%
%% Primarily for testing, need not be listed here.
%ConstantProduct & out(D,R) \newline in(D,R) & constant \newline [smoothing] &  Deterministic product function, multiplying the input times a constant value.  The constant value is specified in the constructor. An optional smoothing value may be specified as a constructor argument$^{\ref{ftn:smoothing}}$. \\
%
Cos & out(D,R) \newline in(D,R) & [smoothing] & Deterministic cosine function, where out = cos(in). An optional smoothing value may be specified as a constructor argument$^{\ref{ftn:smoothing}}$. \\
%
Cosh & out(D,R) \newline in(D,R) & [smoothing] & Deterministic hyperbolic-cosine function, where out = cosh(in). An optional smoothing value may be specified as a constructor argument$^{\ref{ftn:smoothing}}$. \\
%
Dirichlet & [$\alpha$](RJ) \newline value...(RJ) & [$\alpha$] & Dirichlet distribution.  There can be any number of value variables, all associated with the same parameter values.  Parameter vector $\alpha$ can be a RealJoint variable or a constant specified in the constructor.  The dimension of $\alpha$ and each of the value variables must be identical. \\
%
DiscreteTransition & y(D) \newline x(D) \newline A...(RJ) & - & 
Parameterized discrete transition factor, $p(y | x, A)$, where x and y are discrete variables, and $A$ is a matrix of transition probabilities. The transition matrix is organized such that columns correspond to the output distribution for each input state. That is, the transition matrix multiplies on the left. Each column of A corresponds to a RealJoint variable. The number of columns in A must equal the domain size of x, and the dimension of each element of A  must equal the domain size of y. \newline
Each element of A corresponds to a (not necessarily normalized) probability vector.  The conjugate prior for this representation is such that each element of A is distributed according to a Dirichlet distribution\textsuperscript{\ref{ftn:conugatePrior}}.  \newline
In the current implementation, the domain of the x variable must be zero-based contiguous integers, $0...N-1$\textsuperscript{\ref{ftn:zeroBaseRestriction}}. \\
%
DiscreteTransition\newline EnergyParameters & y(D) \newline x(D) \newline A...(R) & $N_{y}, N_{x} | \newline N$ & 
Parameterized discrete transition factor, $p(y | x, A)$, where x and y are discrete variables, and $A$ is a matrix of transition probabilities. The transition matrix is organized such that columns correspond to the output distribution for each input state. That is, the transition matrix multiplies on the left. The number of columns in A and the domain size of x must equal the value of the constructor argument, $N_{x}$ and the number of rows in A and the domain size of y must equal the value of the constructor argument $N_{y}$.  If $N_{x}$ and $N_{y}$ are equal, a single constructor argument, $N$, may be used.  \newline
The elements of the matrix A are represented as energy values, that is, $A_{i,j} = -\log(\rho_{i,j})$, where $\rho$ is an unnormalized transition probability matrix.  The conjugate prior for this representation is such that each entry of A is independently distributed according to a negative exp-Gamma distribution, all with a common $\beta$ parameter\textsuperscript{\ref{ftn:conugatePrior}}.  \newline
In the current implementation, the domain of the x variable must be zero-based contiguous integers, $0...N-1$\textsuperscript{\ref{ftn:zeroBaseRestriction}}. \\
%
DiscreteTransition\newline Unnormalized\newline Parameters & y(D) \newline x(D) \newline A...(R) & $N_{y}, N_{x} | \newline N$ & 
Parameterized discrete transition factor, $p(y | x, A)$, where x and y are discrete variables, and $A$ is a matrix of transition probabilities. The transition matrix is organized such that columns correspond to the output distribution for each input state. That is, the transition matrix multiplies on the left. The number of columns in A and the domain size of x must equal the value of the constructor argument, $N_{x}$ and the number of rows in A and the domain size of y must equal the value of the constructor argument $N_{y}$.  If $N_{x}$ and $N_{y}$ are equal, a single constructor argument, $N$, may be used.  \newline
The elements of the matrix A are represented as unnormalized probability values.  The conjugate prior for this representation is such that each entry of A is independently distributed according to a Gamma distribution, all with a common $\beta$ parameter\textsuperscript{\ref{ftn:conugatePrior}}.  \newline
In the current implementation, the domain of the x variable must be zero-based contiguous integers, $0...N-1$\textsuperscript{\ref{ftn:zeroBaseRestriction}}. \\
%
Divide & quotient(D,R) \newline dividend(D,R) \newline divisor(D,R) & [smoothing] & Deterministic divide function, where $\mathrm{quotient} = \frac{\mathrm{dividend}}{\mathrm{divisor}}$. An optional smoothing value may be specified as a constructor argument$^{\ref{ftn:smoothing}}$. \\
%
Equality & value...(B,D,R) & [smoothing] & Deterministic equality constraint.  An optional smoothing value may be specified as a constructor argument$^{\ref{ftn:smoothing}}$. \\
%
Equals & out(B) \newline in...(B,D,R) & - & Deterministic equals function, where out~=~(in(1)~==~in(2)~== ... ). \\
%
Exp & out(D,R) \newline in(D,R) & [smoothing] & Deterministic exponentiation function, where out = exp(in). An optional smoothing value may be specified as a constructor argument$^{\ref{ftn:smoothing}}$. \\
%
Gamma & [$\alpha$](R) \newline [$\beta$](R) \newline value...(R) & [$\alpha$] \newline [$\beta$] & Gamma distribution. There can be any number of value variables, all associated with the same parameter values.  Parameters $\alpha$ and $\beta$ can be variables, or if both are constant they can be specified in the constructor. \\
%
GreaterThan & out(B) \newline in1(B,D,R) \newline in2(B,D,R) & - & Deterministic greater-than function, where out = in1 $>$ in2.  \\
%
InverseGamma & [$\alpha$](R) \newline [$\beta$](R) \newline value...(R) & [$\alpha$] \newline [$\beta$] & Inverse Gamma distribution. There can be any number of value variables, all associated with the same parameter values.  Parameters $\alpha$ and $\beta$ can be variables, or if both are constant they can be specified in the constructor. \\
%
LessThan & out(B) \newline in1(B,D,R) \newline in2(B,D,R) & - & Deterministic greater-than function, where out = in1 $<$ in2.  \\
%
LinearEquation & out(D,R) \newline in(B,D,R) & constants \newline [smoothing] & Deterministic linear equation, multiplying an input vector by a constant vector. The constant vector is specified in the constructor.  The number of \emph{in} variables must equal the length of the constant vector. An optional smoothing value may be specified as a constructor argument\textsuperscript{\ref{ftn:smoothing}}. \\
%
Log & out(D,R) \newline in(D,R) & [smoothing] & Deterministic natural log function, where out = log(in). An optional smoothing value may be specified as a constructor argument$^{\ref{ftn:smoothing}}$. \\
%
LogNormal & [$\mu$](R) \newline [$\tau$](R) \newline value...(R) & [$\mu$] \newline [$\tau$] & Log-normal distribution. There can be any number of value variables, all associated with the same parameter values.  Parameters $\mu$ (mean) and $\tau = \frac{1}{\sigma^{2}}$ (precision) can be variables, or if both are constant then fixed parameters can be specified in the constructor. \\
%
MatrixVectorProduct & y(D,R) \newline M(D,R) \newline x(D,R) & $N_{x}$ \newline $N_{y}$ \newline [smoothing] & Deterministic matrix-vector product function, $y = Mx$, where $x$ and $y$ are vectors and $M$ is a matrix. Constructor arguments, $N_{x}$ and $N_{y}$, specify the input and output vector lengths, respectively. The matrix dimension is $N_{y} \times N_{x}$. An optional smoothing value may be specified as a constructor argument$^{\ref{ftn:smoothing}}$. \\
%%
%% Primarily for testing, need not be listed here.
%MixedNormal & value(R) \newline control(B) & $\mu_{0} \newline \tau_{0} \newline \mu_{1} \newline \tau_{1}$ & Simple mixture of two fixed-parameter Normal distributions. The choice of distribution parameters (0 vs. 1) is a function of the control bit. \\
%
Negate & out(D,R) \newline in(D,R) & [smoothing] & Deterministic negation function, where out = -in. An optional smoothing value may be specified as a constructor argument$^{\ref{ftn:smoothing}}$. \\
%
NegativeExpGamma & [$\alpha$](R) \newline [$\beta$](R) \newline value...(R) & [$\alpha$] \newline [$\beta$] & Negative exp-Gamma distribution, which is a distribution over a variable whose negative exponential is Gamma distributed. That is, this is the negative log of a Gamma distributed variable. There can be any number of value variables, all associated with the same parameter values.  Parameters $\alpha$ and $\beta$ can be variables, or if both are constant they can be specified in the constructor, and correspond to the parameters of the underlying Gamma distribution. \\
%
Normal & [$\mu$](R) \newline [$\tau$](R) \newline value...(R) & [$\mu$] \newline [$\tau$] & Normal distribution. There can be any number of value variables, all associated with the same parameter values.  Parameters $\mu$ (mean) and $\tau = \frac{1}{\sigma^{2}}$ (precision) can be variables, or if both are constant then fixed parameters can be specified in the constructor. \\
%
Not & out(B) \newline in(B) & - & Deterministic logical NOT of function, where out = ~in. \\
%
NotEquals & out(B) \newline in...(B,D,R) & - & Deterministic not-equals function, where out~=~$\sim$(in(1)~==~in(2)~== ... ). \\
%
Or & out(B) \newline in...(B) & - & Deterministic logical OR function, where out = OR(in...). \\
%
Power & out(D,R) \newline base(D,R) \newline power(D,R) & [smoothing] & Deterministic power function, where out~=~base~$^{\mathrm{power}}$. An optional smoothing value may be specified as a constructor argument$^{\ref{ftn:smoothing}}$. \\
%
Product & out(D,R) \newline in...(B,D,R) & [smoothing] & Deterministic product function, where $\mathrm{out} = \prod \mathrm{in}$. An optional smoothing value may be specified as a constructor argument$^{\ref{ftn:smoothing}}$. \\
%
Rayleigh & [$\sigma$](R) \newline value...(R) & [$\sigma$] & Rayleigh distribution. There can be any number of value variables, all associated with the same parameter value.  Parameter $\sigma$ can be a variable, or if constant, can be specified in the constructor. \\
%
Sin & out(D,R) \newline in(D,R) & [smoothing] & Deterministic sine function, where out = sin(in). An optional smoothing value may be specified as a constructor argument$^{\ref{ftn:smoothing}}$. \\
%
Sinh & out(D,R) \newline in(D,R) & [smoothing] & Deterministic hyperbolic-sine function, where out = sinh(in). An optional smoothing value may be specified as a constructor argument$^{\ref{ftn:smoothing}}$. \\
%
Sqrt & out(D,R) \newline in(D,R) & [smoothing] & Deterministic square root function, where out = sqrt(in). An optional smoothing value may be specified as a constructor argument$^{\ref{ftn:smoothing}}$. \\
%
Square & out(D,R) \newline in(D,R) & [smoothing] & Deterministic square function, where out = in$^{2}$. An optional smoothing value may be specified as a constructor argument$^{\ref{ftn:smoothing}}$. \\
%
Subtract & out(D,R) \newline posIn(B,D,R) \newline negIn...(B,D,R) & [smoothing] & Deterministic subtraction function, where $\mathrm{out} = \mathrm{posIn} - \sum \mathrm{negIn}$. An optional smoothing value may be specified as a constructor argument$^{\ref{ftn:smoothing}}$. \\
%
Sum & out(D,R) \newline in...(B,D,R) & [smoothing] & Deterministic summation function, where $\mathrm{out} = \sum \mathrm{in}$. An optional smoothing value may be specified as a constructor argument$^{\ref{ftn:smoothing}}$. \\
%
Tan & out(D,R) \newline in(D,R) & [smoothing] & Deterministic tangent function, where out = tan(in). An optional smoothing value may be specified as a constructor argument$^{\ref{ftn:smoothing}}$. \\
%
Tanh & out(D,R) \newline in(D,R) & [smoothing] & Deterministic hyperbolic-tangent function, where out = tanh(in). An optional smoothing value may be specified as a constructor argument$^{\ref{ftn:smoothing}}$. \\
%
VonMises & [$\mu$](R) \newline [$\tau$](R) \newline value...(R) & [$\mu$] \newline [$\tau$] & Von Mises distribution. There can be any number of value variables, all associated with the same parameter values.  Parameters $\mu$ (mean) and $\tau = \frac{1}{\sigma^{2}}$ (precision) can be variables, or if both are constant then fixed parameters can be specified in the constructor.  The distribution is non-zero for value variables in the range $-\pi$ to $\pi$. \\
%
Xor & out(B) \newline in...(B) & - & Deterministic logical XOR function, where out = XOR(in...). \\
%
\end{longtable}

\subsubsection{Solver-Specific Built-in Factors}
\label{sec:solverSpecificBuiltInFactors}

In addition to the built-in factors listed above, there are a set of solver-specific built-in factors, also referred to as ``custom factors.''  These factors may also be specified by name in the addFactor call (using either a quoted string or as a function handle).  These are summarized in the following table:

\begin{longtable} {l p{3cm} p{7cm}}
Name & Solver & Description \\
\hline
\endhead
%
FiniteFieldFactor & SumProduct\footnote{\label{ftn:fff} These may also be used for discrete variables with the Gaussian or particle BP solvers, which use the sum-product solver for discrete-only portions of the graph.}  & See section~\ref{sec:finiteFields} \\ 
FiniteFieldProjection & SumProduct$^{\ref{ftn:fff}}$ & See section~\ref{sec:finiteFields} \\ 
FiniteFieldAdd & SumProduct$^{\ref{ftn:fff}}$ & See section~\ref{sec:finiteFields} \\ 
FiniteFieldConstMult & SumProduct$^{\ref{ftn:fff}}$ & See section~\ref{sec:finiteFields} \\ 
FiniteFieldMult & SumProduct$^{\ref{ftn:fff}}$ & See section~\ref{sec:finiteFields} \\ 
multiplexerCPD & SumProduct$^{\ref{ftn:fff}}$ & See section~\ref{sec:multiplexerCPD} \\ 
CustomXor & MinSum & Same as the Xor factor described above, but with a significantly faster implementation. \\ 
\end{longtable}


\ifmatlab

\subsubsection{Factor Creation Utility Functions}

Dimple includes some helper functions to create other built-in factors using a similar syntax to the overloaded MATLAB functions described in section~\ref{sec:overloaded}.  As for other overloaded functions, above, Dimple automatically creates the factors as well as the output variable(s).  Such helper functions are defined for the following built-in distributions:

\begin{itemize}
\item Bernoulli
\item Beta
\item Categorical\footnote{This utility can be used to create either a Categorical or CategoricalUnnormalizedParameters factor, depending on whether the first argument is a RealJoint variable or an array of Real variables.}
\item CategoricalEnergyParameters
\item Dirichlet
\item Gamma
\item InverseGamma
\item LogNormal
\item NegativeExpGamma
\item Normal
\item Rayleigh
\item VonMises
\end{itemize}

For each, the arguments are the parameters of the distribution.  For example:

\begin{lstlisting}
W = Gamma(alpha, beta);
X = Normal(mean, precision);
Y = Categorical(alphaVector);
Z = Rayleigh(sigma);
\end{lstlisting}


The parameters can be variables, constants\footnote{If \emph{all} parameter inputs are constants, then instead of creating both variables and factors, the variables are created, and the Input to each variable is set accordingly.}, or some of each\footnote{For cases where a parameter is a vector of real values, the parameter may be a Real variable array, a cell-arrays that may mix Real variables and constants, or an array of constants.}.

By default, calling one of these functions creates a single output variable, and the factor is added to the most-recently created graph. But, optional arguments allow you to specify the dimensions of the array of output variables, or to specify the factor graph. �These arguments can be in either order after the parameters. For example:

\begin{lstlisting}
W = Gamma(alpha, beta, altGraph);
X = Normal(mean, precision, [100, 1]);
Y = Categorical(alphaVector, [10, 10, 2], aGraph);
Z = Rayleigh(sigma, myGraph, size(somethingElse));
\end{lstlisting}

\fi


Dimple includes some built-in helper functions to create structured graphs, combining smaller factors to form an efficient implementation of a larger subgraph.  Specifically, the following functions are provided:

\ifmatlab
\begin{itemize}
\item getNBitXorDef(n), where n is a positive integer. Returns a nestable graph and an array of n-Bit connector variables. Efficient tree implementation of the XORDelta function.
\item getVXOR(n), where n is a positive integer. Returns a nestable graph and an array of n-Bits connector variables. Constrains exactly one bit to be 1, and all others to be 0.
\item MultiplexerCPD(domains) - See section~\ref{sec:multiplexerCPD}
\item MultiplexerCPD(domain,numZs) - See section~\ref{sec:multiplexerCPD}
\end{itemize}
\fi

\ifjava
\begin{itemize}
\item MultiplexerCPD(domains) - See section~\ref{sec:multiplexerCPD}
\item MultiplexerCPD(domain,numZs) - See section~\ref{sec:multiplexerCPD}
\end{itemize}
\fi



\ifmatlab
\subsection{List of Overloaded MATLAB Operators and Functions}
\label{sec:overloaded}

The following table lists the set of overloaded MATLAB operators that can be used to implicitly create factors.  The table shows the operator, the corresponding built-in factor (as described in section~\ref{sec:builtInFactors}), the valid variable data types of the inputs and outputs (B = Bit, D = Discrete, or R = Real), and wether or not vectorized inputs are supported.  The use of these operators and functions is described in section~\ref{sec:ImplicitFactorCreation}.

\begin{longtable} {l p{3cm} p{1cm} p{1cm} p{1cm} l p{4cm}}
Operator & Factor & Out & In1 & In2 & Vectorized & Description \\
\hline
\endhead
%
$\&$ & And & B & B & B & \checkmark & Logical AND \\
$|$ & Or & B & B & B & \checkmark & Logical OR \\
xor() & Xor & B & B & B & \checkmark & Logical XOR \\
$\sim$ & Not & B & B & - & \checkmark & Logical NOT \\
$+$ & Sum & D,R\footnote{\label{ftn:outReal}If either input is Real, then the output is Real} & D,R & D,R & \checkmark & Plus \\
$-$ & Subtract & D,R\textsuperscript{\ref{ftn:outReal}} & D,R & D,R & \checkmark & Minus \\
$-$ & Negate & D,R\textsuperscript{\ref{ftn:outReal}} & D,R & - & \checkmark & Unary minus \\
$*$ & Product \newline MatrixVectorProduct & D,R\textsuperscript{\ref{ftn:outReal}} & D,R & D,R & \checkmark\footnote{One of the inputs may be a vector as long as the other is a scalar.} & Scalar multiply, or matrix-vector multiply\footnote{If one input is a vector and the other is a matrix of appropriate dimension, then the MatrixVectorProduct factor will be used.  Otherwise the Product factor will be used.} \\
$.*$ & Product & D,R\textsuperscript{\ref{ftn:outReal}} & D,R & D,R & \checkmark & Point-wise multiply \\
$/$ & Divide & D,R\textsuperscript{\ref{ftn:outReal}} & D,R & D,R & \checkmark\footnote{The dividend may be a vector as long as the divisor is a scalar.} & Scalar divide \\
$./$ & Divide & D,R\textsuperscript{\ref{ftn:outReal}} & D,R & D,R & \checkmark & Point-wise divide \\
$\wedge$ & Power & D,R\textsuperscript{\ref{ftn:outReal}} & D,R & D,R & \checkmark\footnote{The base may be a vector as long as the exponent is a scalar.} & Scalar power \\
$.\wedge$ & Power & D,R\textsuperscript{\ref{ftn:outReal}} & D,R & D,R & \checkmark & Point-wise power \\
$<$ & LessThan & B & D,R & D,R & \checkmark & Less than \\
$>$ & GreaterThan & B & D,R & D,R & \checkmark & Greater than \\
$<=$ & GreaterThan\footnote{Uses GreaterThan factor, reversing the order.} & B & D,R & D,R & \checkmark & Less than or equal to \\
$>=$ & LessThan\footnote{Uses LessThan factor, reversing the order.} & B & D,R & D,R & \checkmark & Greater than or equal to \\
Equals() & Equals & B & B,D,R & B,D,R\footnote{\label{ftn:equals}This function is not limited to two inputs, but can take an arbitrary number of inputs} & \checkmark & Equals\footnote{Equivalent to the $==$ operator, but the $==$ operator is not overloaded for this purpose so that it can instead be used to determine whether or not two variables reference the same Dimple variable.} \\
NotEquals() & NotEquals & B & B,D,R & B,D,R$^{\ref{ftn:equals}}$ & \checkmark & Not equals\footnote{Equivalent to the $\sim=$ operator, but the $\sim=$ operator is not overloaded for this purpose so that it can instead be used to determine whether or not two variables reference the same Dimple variable.} \\
mod() & - & D  & D & D & \checkmark & Modulo function\footnote{Currently, the mod() operator supports discrete variables only, and it uses the MATLAB definition of mod on negative numbers.  This may be subject to change in future versions.} \\
abs() & Abs & D,R\textsuperscript{\ref{ftn:outReal}} & D,R & - & \checkmark & Absolute value \\
sqrt() & Sqrt & R & R & - & \checkmark & Square root \\
log() & Log & R & R & - & \checkmark & Natural log \\
exp() & Exp & R & R & - & \checkmark & Exponential function \\
sin() & Sin & R & R & - & \checkmark & Sine \\
cos() & Cos & R & R & - & \checkmark & Cosine \\
tan() & Tan & R & R & - & \checkmark & Tangent \\
asin() & ASin & R & R & - & \checkmark & Arc-sine \\
acos() & ACos & R & R & - & \checkmark & Arc-cosine \\
atan() & ATan & R & R & - & \checkmark & Arc-tangent \\
sinh() & Sinh & R & R & - & \checkmark & Hyperbolic sine \\
cosh() & Cosh & R & R & - & \checkmark & Hyperbolic cosine \\
tanh() & Tanh & R & R & - & \checkmark & Hyperbolic tangent \\
\end{longtable}




\fi

\subsection{Other Top Level Functions}

\subsubsection{setSolver}

\begin{lstlisting}
setSolver('SolverName');
\end{lstlisting}

This function changes the default solver to the solver designated by the argument, which is a string indicating the name of the solver (see section~\ref{sec:FactorGraph.Solver} for the list of valid solver names, and section~\ref{sec:SolversAPI} for a description of each solver).  The solver name is case insensitive.


%\para{getSolverNames}
%
%\begin{lstlisting}
%getSolverNames()
%\end{lstlisting}
%
%
%\para{registerSolver}
%
%\begin{lstlisting}
%registerSolver(solverName,solverConstructor);
%\end{lstlisting}
%
%
%\para{unregisterSolver}
%
%\begin{lstlisting}
%unregisterSolver(solverName);
%\end{lstlisting}
%





\begin{appendices}
\section{A Short Introduction to Factor Graphs}
\label{sec:IntroductionToFactorGraphs}

We introduce factor graphs, a powerful tool for statistical modeling.  
Factor graphs can be used to describe a number of commonly used 
statistical models, such as hidden Markov models, Markov random 
fields, Kalman filters, and Bayes nets. 

Suppose that we are given a set of n discrete random variables: $a_1,...,a_n$.  The random variables have some joint probability distribution:  $ p(a_1,a_2,...,a_n) $.  Suppose that the joint probability distribution factors, in the following sense: there exist subsets  $ S_1,...,S_k \subseteq \lbrace 1,2,...,n\rbrace $ where $ S_j = \lbrace S_1^j,s_2^j,...,s_{t(j)}^j \rbrace $ and such that $p(a_1,a_2,...,a_n) = \prod_{j=1}^k f_j(a_{s_1^j},a_{s_2^j},...,a_{s_{t(j)}^j})$.

For example, if the $a_i$ form a Markov chain, then the joint probability can be factored as
%
\begin{eqnarray}
\label{eqn:markovChain}
p(a_1,a_2,...,a_n) &=& p(a_1) \prod_{j=1}^{n-1}p(a_{j+1}|a_j) \\
		   &=& f_0(a_1) \prod_{j=1}^{n-1} f_j(a_j,a_{j+1})
\end{eqnarray}

The factors above are normalized, in the sense that as the $a_i$  vary, 
the probabilities sum to one.  We will define our factors more 
generally and ask only that they are proportional to the joint 
probability.  So, we call the $f_{j}$ a collection of factors of $p$ if 
%
\[
p(a_1,a_2,...,a_n) \propto \prod_{j=1}^k f_j(a_{s_1^j},a_{s_2^j},...,a_{s_{t(j)}^j}) 
\]

The product of the factors then differs from the joint probability only by 
multiplication by a normalizing constant.

When a probability distribution can be expressed as a product of small factors 
(i.e., $|S_j|$ is small for all j), then if is possible to invoke a host of powerful 
tools for modeling and inference, as we will soon see.

Suppose that we are given a factored representation of a joint probability distribution.  
It is possible to describe the structure of the factors as a graph.  We can represent 
each variable $a_i$  and each function $f_j$  by a node in the graph, and place an (undirected) 
edge between node $a_i$  and node $f_j$ if and only if the variable $a_i$  is an argument in the 
function $f_j$.  These two types of nodes are referred to as factor nodes and variable nodes. 
Because all edges lie between the two disjoint classes of nodes, the resulting graph is bipartite.  
This graph is called a factor graph.

In the remainder of this documentation, we slightly abuse notation and use $f_j$  and $a_i$  
to refer both to the nodes of the factor graph and to the underlying factors and variables 
(i.e., both the graphical representation of these entities and the mathematical entities underlying them).

To understand what factor graphs look like, we will construct several examples.  
First, let us continue with a Markov chain. Equation~\ref{eqn:markovChain} expressed a Markov chain in factored form, where  
%
\[
f_j(a_j,a_{j+1})=p(a_{j+1}|a_j)
\]

We display the corresponding factor graph in the following figure:

\includegraphics{images/FactorGraphExample.png}

Next, let us consider a hidden Markov model (HMM).  We can construct the 
corresponding factor graph by extending the previous example.  An HMM 
contains a Markov chain transiting from state $a_{i}$ to $a_{i+1}$. There is also an 
observation $b_{i}$ made of each state; if we are given $a_{i}$, then $b_{i}$
is conditionally independent of all other variables. We can incorporate 
this probability by using a factor:
%
\[
g_i(a_i) = Pr(b_i|a_i)
\]

The product of our factors is then
%
\begin{eqnarray*}
f_0 \left( \prod_{j=1}^{n-1} f_j(a_j,a_j+1) \right) \prod_{j=1}^n g_j(a_j)  &=& 
    Pr(a_1) \left( \prod_{j=1}^{n-1} Pr(a_{j+1}|a_j)  \right) \prod_{j=1}^n Pr(b_j|a_j)  \\
  &=& Pr(a_1,...,a_n,b_1,...,b_n)
\end{eqnarray*}

Since the $b_i$ observed, then $Pr(b_1,...,b_n)$ is a constant.  Therefore
%
\begin{eqnarray*}
Pr(a_1,...,a_n,b_1,...,b_n) & \propto & \frac{Pr(a_1,...,a_n,b_1,...,b_n)}{Pr(b_1,...,b_n)} \\
			    &=& Pr(a_1,...,a_n|b_1,...,b_n)
\end{eqnarray*}

as desired.

\includegraphics{images/HMMExample.png}

The resulting factor graph takes the following form illustrated in the figure above. Note 
that the variables $b_i$  need not appear explicitly in the factor graph; we have incorporated their effect in the $g_i$  factors.

Generalizing from a Markov chain to an HMM illustrates a very powerful feature of factor graphs.  Complicated mathematical models are often composed of simpler parts.  When these models are expressed as factor graphs, we can frequently reuse the simpler factor graphs to construct the more complicated ones.  This can be done simply in Dimple by using the nested graphs feature (see section~\ref{sec:nestedGraphs} on Nested Graphs).

As a final example, we will construct a factor graph for error correction (for this
 more advanced topic, we will assume the reader is familiar with LDPC codes).  Suppose
 that we receive a codeword from a 4-bit LDPC error-correcting code that has been corrupted by noise.  
The sender wishes to communicate a four-bit codeword $(a_{1},a_{2},a_{3},a_{4})$ satisfying some parity check equations, 
but the receiver only observes the corrupted values $(b_{1},b_{2},b_{3},b_{4})$. (The domain of the $b_{i}$ is determined by the 
communication channel.  For instance, if we have a discrete binary symmetric channel, then the $b_{i}$ will be bits; 
if we have a continuous additive white Gaussian noise channel and some modulation scheme, the $b_{i}$  will be real-valued.) 
Let $H$ be the parity check matrix of the LDPC code used, i.e., the codeword $(a_{1},a_{2},a_{3},a_{4})$ verifies the equation $Ha=0 \textrm{ mod } 2$.

For instance, suppose that $H$ is the following parity check matrix:
%
\[
H = \left( \begin{array}{cccc}
1 & 1 & 0 & 1 \\
1 & 1 & 1 & 0 \\
0 & 1 & 1 & 1 \end{array} \right) 
\]

Suppose that the error is independent, i.e., if we condition on $a_i$, $b_i$ is conditionally independent of the other variables.   Then, the following factor graph represents the decoding task at hand.

\includegraphics{images/DecoderExample.png}




The construction above applies to any linear binary ECC.  However, if every row and column of H is sparse (as would be the case with an LDPC code), then every factor is small, and every node in the factor graph will be of small degree.

Given a factor graph, the objective is often to compute the marginal distribution of the random variables $a_i$ of the graph (this also allows us to find the most likely value taken by each variable, by maximization of the marginal probability). Dimple provides an implementation of belief propagation (BP) (in its sum-product version), in order to approximately compute the marginal distribution of each random variable. 

BP is an iterative message-passing algorithm where messages pass along the edges of the factor graph.  A ``message'' can be viewed as an un-normalized probability distribution.  The algorithm comes in a number of variants depending on the message update rule and the order of the message updates.  

The sum-product form of BP generalizes a number of algorithms, including the ``forward-backward'' algorithm of HMMs, and the BCJR algorithm of coding theory. It always gives the exact answer when the underlying factor graph is a tree (if the graph contains no cycles). Although it is not an exact algorithm for general graphs, BP has been found to give excellent results for a wide variety of factor graphs, and runs particularly fast on sparse factor graphs (i.e., factor graphs of low node degree).


\section{Creating Custom Dimple Extensions \ifmatlab in Java\fi}
\label{sec:userJava}

\subsection{Creating a Custom Factor Function}

\ifmatlab
There are some cases in which it is desirable to add a custom factor function written in Java rather than MATLAB.  Specific cases where this is desirable are:
%
\begin{itemize}
\item A factor function is needed to support continuous variables that is not available as a Dimple built-in factor.
\item A MATLAB factor function runs too slowly when creating a factor table, where a Java implementation may run more quickly.
\end{itemize}
\fi

\ifjava
When a factor function function is needed to support continuous variables that is not available as a Dimple built-in factor, then it is necessary to create a custom factor function.
\fi

To create a custom factor function, you must create a Java class that extends the Dimple \texttt{FactorFunction} class.  When extending the \texttt{FactorFunction} class, the following method must be overwritten:
%
\begin{itemize}
\item \texttt{evalEnergy}: Evaluates a set of input values and returns an energy value (negative log of a weight value).
\end{itemize}


The user may extend other methods, as appropriate:

\begin{itemize}
%
\item Constructor: If a constructor is specified (for example, to pass constructor arguments), it must call the constructor of the super class.
%
\item \texttt{isDirected}: Indicates whether the factor function is directed.  If directed, then there are a set of directed outputs for which the marginal distribution for all possible input values is a constant.  If not overridden, this is assumed false.
%
\item \texttt{getDirectedToIndices}: If a factor function is directed, indicates which edge indices are the directed outputs (numbering from zero), returning an array of integers.  There are two forms of this method, which may be used depending on whether the set of directed outputs depends on the number of edges in the factor that uses this factor function (many factor functions support a variable number of edges).  If \texttt{isDirected} is overridden and can return \texttt{true}, then this method must also be overridden.
%
\item \texttt{isDeterministicDirected}: Indicates whether a factor function is both directed and deterministic.  If deterministic and directed, then it is in the form of a deterministic function such that for all possible settings of the input values there is exactly one output value the results in a non-zero weight (or, equivalently, a non-infinite energy)\footnote{The indication that a factor function is deterministic directed is used by the Gibbs solver, and is necessary for such factor functions to work when using the Gibbs solver.}.  If not overridden, this is assumed false.
%
\item \texttt{evalDeterministic}: If a factor function is directed and deterministic, this method evaluates the values considered the inputs of the deterministic function and returns the resulting values for the corresponding outputs.  Note that these are not the weight or energy values, but the actual values of the associated variables that are considered outputs of the deterministic function.  If \texttt{isDeterministicDirected} is overridden and can return \texttt{true}, then this method must also be overridden.
%
\item \texttt{eval}: Evaluates a set of input values and returns a weight instead of an energy value.  Overriding this method would only be useful if implementing this method can be done significantly more computationally efficiently than the default implementation, which calls evalEnergy and then computes $\exp(-energy)$.
%
\end{itemize}

The following is a very simple example of a custom factor function:

\begin{lstlisting}
import com.analog.lyric.dimple.factorfunctions.core.FactorFunction;

/*
 * This factor enforces equality between all variables and weights
 * elements of the domain proportional to their value
 */
public class BigEquals extends FactorFunction
{	
    @Override
    public final double evalEnergy(Value[] input)
    {
        if (input.length == 0)
            return 0;
	    	
        Value firstVal = input[0];
	    	
        for (int i = 1; i < input.length; i++)
            if (!input[i].valueEquals(firstVal))
                return Double.POSITIVE_INFINITY;
	    	
        return 0;
    }
}
\end{lstlisting}




\subsection{Creating a Custom Proposal Kernel}
\label{sec:CreatingACustomProposalKernel}

In some cases, it may be useful to add a custom proposal kernel when using the Gibbs solver with a Metropolis-Hastings sampler.  In particular, since the \emph{block} Metropolis-Hastings sampler does not have a default proposal kernel, it is necessary to add a custom proposal kernel in this case.

To create a custom proposal kernel, you must create a Java class that implements either the Dimple \texttt{IProposalKernel} interface in the case of a single-variable proposal kernel, or the \texttt{IBlockProposalKernel} interface in the case of a block proposal kernel.

These interfaces define the following methods that must be implemented:
%
\begin{itemize}
%
\item \texttt{next} This method takes the current value(s) of the associated variable(s) along with the corresponding variable domain(s), and returns a proposal.  The proposal object returned (either a \texttt{Proposal} object for a single-variable proposal or a \texttt{BlockProposal}: object for a block proposal) includes both the proposed value(s) as well as the forward and reverse proposal probabilities (the negative log of these probabilities).
%
\item \texttt{setParameters}: Allows the class to include user-specified parameters that can be set to modify the behavior of the kernel.  This method must be implemented but may be empty if no parameters are needed.
%
\item \texttt{getParameters}: Returns the value of any user-specified parameters that have been specified.  This method must be implemented but may return null if no parameters are needed.
%
\end{itemize}


\ifmatlab

\subsection{Compiling Dimple Extensions in Java}

The new class must be compiled to class files, or optionally create a jar file. If using Eclipse, users can simply create a new project, create the new class, and the .class files will be created automatically.


\subsection{Adding Java Binary to MATLAB Path}

In MATLAB, you must use the javaaddpath call to add the Java class or jar files to the javaclasspath.  For example:

\begin{lstlisting}
javaaddpath('<path to my project>/MyFactorFunctions/bin');
\end{lstlisting}

or

\begin{lstlisting}
javaaddpath('<path to the jar>/myjar.jar');
\end{lstlisting}

\fi
\end{appendices}

\end{document}  
