\subsection{Other Top Level Functions}

\subsubsection{setSolver}

\begin{lstlisting}
setSolver('SolverName');
\end{lstlisting}

This function changes the default solver to the solver designated by the argument, which is a string indicating the name of the solver (see section~\ref{sec:FactorGraph.Solver} for the list of valid solver names, and section~\ref{sec:SolversAPI} for a description of each solver).  The solver name is case insensitive.


%\para{getSolverNames}
%
%\begin{lstlisting}
%getSolverNames()
%\end{lstlisting}
%
%
%\para{registerSolver}
%
%\begin{lstlisting}
%registerSolver(solverName,solverConstructor);
%\end{lstlisting}
%
%
%\para{unregisterSolver}
%
%\begin{lstlisting}
%unregisterSolver(solverName);
%\end{lstlisting}
%

\subsubsection{Factor Creation Utility Functions}

Dimple includes some helper functions to create other built-in factors using a similar syntax to the overloaded MATLAB functions described in section~\ref{sec:overloaded}.  As for other overloaded functions, above, Dimple automatically creates the factors as well as the output variable(s).  Such helper functions are defined for the following built-in distributions:

\begin{itemize}
\item Beta
\item Gamma
\item InverseGamma
\item NegativeExpGamma
\item Normal
\item LogNormal
\item VonMises
\item Rayleigh
\item Categorical
\end{itemize}

For each, the arguments are the parameters of the distribution.  For example:

\begin{lstlisting}
W = Gamma(alpha, beta);
X = Normal(mean, precision);
Y = Categorical(alphaVector);
Z = Rayleigh(sigma);
\end{lstlisting}


The parameters can be variables, constants, or some of each.

By default, calling one of these functions creates a single output variable, and the factor is added to the most-recently created graph. �But, optional arguments allow you to specify the dimensions of the array of output variables, or to specify the factor graph. �These arguments can be in either order after the parameters. �For example:

\begin{lstlisting}
W = Gamma(alpha, beta, altGraph);
X = Normal(mean, precision, [100, 1]);
Y = Categorical(alphaVector, [10, 10, 2], aGraph);
Z = Rayleigh(sigma, myGraph, size(somethingElse));
\end{lstlisting}


Dimple also includes some built-in helper functions to create structured graphs, combining smaller factors to form an efficient implementation of a larger subgraph.  Specifically, the following functions are provided:

\begin{itemize}
\item getNBitXorDef(n), where n is a positive integer. Returns a nestable graph and an array of n-Bit connector variables. Efficient tree implementation of the XORDelta function.
\item getVXOR(n), where n is a positive integer. Returns a nestable graph and an array of n-Bits connector variables. Constrains exactly one bit to be 1, and all others to be 0.
\end{itemize}

